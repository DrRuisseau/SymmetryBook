\chapter{Historical remarks}
\label{ch:grouphistory}

Here we briefly sketch some of the history of groups.
See the book by \citeauthor{Wussing-genesis}\footcite{Wussing-genesis}
for a detailed account,
as well as the shorter survey by
\citeauthor{Kleiner-group-survey}\footcite{Kleiner-group-survey}.
There's also the book by \citeauthor{Yaglom1988}\footcite{Yaglom1988}.

Some waypoints we might mention include:
\begin{itemize}
\item Early nineteenth century geometry,
  the rise of projective geometry, Möbius and Plücker
\item Early group theory in number theory,
  forms, power residues, Euler and Gauss.
\item Permutation groups, Lagrange and Cauchy,
  leading (via Ruffini) to Abel and Galois.
\item Liouville and Jordan\footcite{Jordan} ruminating on Galois.
\item Cayley, Klein and the Erlangen Program\footcite{Klein-EP-de}.
\item Lie and differentiation.
\item von~Dyck and Hölder.
\item J.H.C.~Whitehead and crossed modules.
\item Artin and Schreier theory.
\item Algebraic groups (Borel and Chevalley et al.)
\item Feit-Thompson and the classification of finite simple groups.
\item Grothendieck and the homotopy hypothesis.
\item Voevodsky and univalence.
\end{itemize}

%%% Local Variables:
%%% mode: latex
%%% fill-column: 144
%%% latex-block-names: ("lemma" "theorem" "remark" "definition" "corollary" "fact" "properties" "conjecture" "proof" "question" "proposition")
%%% TeX-master: "book"
%%% End:

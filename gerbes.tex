\chapter{Gerbes}
\label{ch:bunches}

We use here the $1$-convention: without further qualifiers, ``group'',
``bunch'', etc.~mean $1$-group, $1$-bunch, etc.

Recall definitions:

$\Group$: A group is a pointed, connected $1$-type

$\Bunch$: A bunch is a connected $1$-type

$\Band\jdeq\Trunc{\Bunch}_1$: a band is an element of the
$1$-truncation of $\Bunch$.

Center: $Z : \Band \to \AbBunch$, $\B^2Z(K) \jdeq \sum_{X : \Bunch}
\band X = K$.

The forgetful map $\AbGroup \to \AbBunch$ is $1$-connected, so induces
an equivalence $\AbGroup \equiv \AbBand$.

\begin{lemma}
  Let $A : \Bunch$ and $x,y:A$. There is a canonical action of
  $\Inn(x)\times\Inn(y)$ on $\Aut(x) = \Aut(y)$ whose orbit set is
  $\Aut(\band A)$.
\end{lemma}
\begin{proof}
  The action takes $\angled{\angled{G,p},\angled{H,q}}$ to $(G=H)$.
  The orbit set $O \defeq (\Aut(x)=\Aut(y)) / \Inn(x)\times\Inn(y)$ is
  \begin{align*}
    O
    &= \Trunc*{\sum_{G:\Group}\Trunc{\bunch G=A}_0 \times
      \sum_{H:\Group}\Trunc{\bunch G=A}_0 \times (G = H)}_0 \\
    &= \Trunc*{\sum_{X:\Bunch}(X=A) \times
      \sum_{Y:\Bunch}(Y=A) \times \sum_{x:X}\sum_{y:Y}\sum_{f:X=Y}(\mathop
      f x = y)}_0 \\
    &= \Trunc*{\sum_{x:A}\sum_{y:A}\sum_{f:A=A}(\mathop f x = y)}_0 \\
    &= \Trunc*{A \times (A = A)}_0 = \Trunc A_0 \times \Trunc{A=A}_0 \\
    &= (\band A = \band A) = \Aut(\band A) \qedhere
  \end{align*}
\end{proof}

\begin{definition}
  The category of \emph{bands} is the homotopy category of the
  $(2,1)$-category of bunches.
\end{definition}
\begin{lemma} % Giraud IV.1.2.3
  The band functor $\Group \Rightarrow \Band$ induces an equivalence
  of categories between the subcategories of abelian groups and
  abelian bands.
\end{lemma}
\begin{proof}
  The functor is fully faithful and essentially surjective on objects.
\end{proof}

The category of bands has a zero object represented by the trivial
group. We have finite products and $\band$ preserves products.

\begin{lemma}
  The category of bands has cokernels and $\band$ preserves cokernels.
\end{lemma}
\begin{lemma}
  The inclusion $\AbGroup \hookrightarrow \Band$ has a left adjoint.
\end{lemma}
(This is abelianization, mapping $L$ to $\pi_2(\Sigma L)$.

\begin{definition}
  Suppose $u : L \to M$ is a morphism of bands. The \emph{centralizer}
  of $u$ is a band $C_u$ and a morphism $c_u : L \times C_u \to M$
  such that for every band $X$ the induced map
  \[
    \Hom(X,C_u) \to \setof{x \in \Hom(L \times X,
      M)}{x\circ\iota_1=u},
  \]
  with $x \mapsto c_u \circ (\id_L\times x)$, is an isomorphism.
\end{definition}
Centralizers exist by picking representatives. (Is there a more direct
definition?)

If $L$ is a band, then the \emph{center} of $L$ is an abelian group
$Z(L)$ with $\B^2Z(L) \defeq \sum_{X : \Bunch} L=\band(X) \jdeq
L\Bunch$. This gives a well-defined abelian group.

\begin{theorem}
  % Giraud IV.2.3.2
  Let $u : L\to M$ be a morphism of bands, $P:L\Bunch$, $Q:M\Bunch$.
  Then $\Hom_u(P,Q)$ is a bunch.

  Furthermore, $\band\Hom_u(P,Q)$ with the band of the application map
  is a centralizer of $u$.
\end{theorem}
\begin{lemma}
  Let $L:\Band$, $P,Q:L\Bunch$. Then $\Hom_{\id_L}(P,Q)$ is banded by
  the center of $L$.
\end{lemma}

\begin{lemma}
  % Giraud IV.2.4.4
  $L:\Band$, $C$ is the center of $L$, $P,Q:L\Bunch$. Then there is a
  canonical equivalence $P \wedge^C \Hom_L(P,Q) \to Q$.
\end{lemma}

\begin{lemma}
  % Giraud IV.2.5.5.5
  Suppose we have a $k : K\to F$, $m:F\to G$ morphisms of bunches lying
  over an exact sequence of band homomorphisms. Then there is a unique
  section $s : G$ such that $m\circ k=s\circ 1$. There is an
  equivalence between $K$ and $\fiber_m(s)$.
\end{lemma}

\begin{definition}
  Let $1\to A\to B\to C\to 1$ be an exact sequence of groups, and let
  $P$ be a $C$-torsor. Then the fiber of $P$ over $B$ is a bunch with
  band ${}^P\band(A)$.
\end{definition}

\begin{definition}
  Let $L : \Band$. Then $\sum_{G:\Group}\band(G)=L$ is a bunch. (The
  bunch of \emph{representatives} of $L$).
\end{definition}

\begin{definition} % Giraud IV.4.1.3
  Let $v : M \to N$ be a map of bands. Then $\Ker(v)$ is a bunch. It
  consists of pairs $\angled{N,u}$ of a band $N$ and a map $u:N\to M$
  giving an exact sequence.
\end{definition}

\section{Bilinear maps and cup products}

Suppose $G,H,A$ are abelian groups and $\mu : G \times H \to A$ is bilinear.
Then we have an induced map $\BG \wedge \BH \to_\pt \B^2A$.

%%% Local Variables:
%%% mode: latex
%%% TeX-master: "book"
%%% End:

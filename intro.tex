%% introduction to the book
\chapter{Introduction to the topic of this book}
\label{ch:intro}

\begin{quote}
  \itshape \foreignlanguage{ngerman}{Poincar\'e sagte gelegentlich,
  dass alle Mathematik eine Gruppengeschichte war.
  Ich erz\"ahlte ihm dann \"uber dein Programm,
  das er nicht kannte.}

  \smallskip

  \noindent Poincar\'e was saying
  that all of mathematics was a tale about groups.
  I then told him about your program,
  which he didn't know about.
\end{quote}
\hfill (Letter from Sophus Lie to Felix Klein, October 1882)

\bigskip

%{\em If this book is about group theory, then here we will explain what's interesting about groups and why one would want to study them.}


Since this book is called ``symmetry'' it is reasonable to hope
that by the time you're reached the end you'll have a clear idea of
what ``symmetry'' means.

Ideally the answer should give a solid foundation for dealing with
questions about symmetries. It should also equip you with language
with which to talk about symmetries, making precise -- but also
reflecting faithfully -- the intuition humans seem to be born with.

So, we should start by talking about how one intuitively can approach the
subject while giving hints about how this intuition can be made into
the solid, workable tool, which is the topic of this book.

When we say that something is ``symmetric'' or possesses many ``symmetries'',
we mean that the thing remains unchanged, even if we ``do things to it.''
The best examples to begin with is if the something is some shape, for
instance this square $\square$. Rotating by 90 degrees doesn’t change it, so we may say that ``rotation by 90 degrees is a symmetry of $\square$''
Of course, rotating by $90$ degrees will move individual points in $\square$, but that
is not of essence -- the shape remains the same.
However, the outcome of rotating by $360$ degree or not at all
is the same - even from the point of view of each individual point in $\square$ -- so it probably feels contrived to count rotations by $0$ and $360$ degrees as different rotations.

It feels reasonable to consider the rotations by $0$\textdegree, $90$\textdegree, $180$\textdegree, and $270$\textdegree{} to be all the (rotational) symmetries of $\square$.  Two thoughts may strike you:
\begin{enumerate}
\item
  are these \emph{all} the symmetries?
\item ``rotation'' indicates a \emph{motion}, through different squares, joining $\square$ with itself via a ``journey in the world of squares''.
\end{enumerate}
\begin{center}
  \begin{tikzpicture}
    \foreach \x/\s in {45/0,35/1,25/2,15/3,5/4,-5/5,-15/6,-25/7,-35/8,-45/9} {
      \begin{scope}[xshift=\s cm]
        \draw (\x:.3) -- (\x+90:.3) -- (\x+180:.3) -- (\x+270:.3) -- cycle;
      \end{scope}
    }
    % stick figure pushing
    \begin{scope}[thick,line cap=round]
      \node[dot] at (-.3,.3) {};
      \draw (-.4,.1) -- (-.212,.1);
      \draw (-.5,-.1) -- (-.35,.2);
      \draw (-.5,-.1) -- (-.35,-.1);
      \draw (-.5,-.1) -- (-.6,-.2);
      \draw (-.35,-.1) -- (-.38,-.3);
      \draw (-.38,-.3) -- (-.33,-.3);
      \draw (-.6,-.2) -- (-.78,-.28);
      \draw (-.78,-.28) -- (-.73,-.3);
    \end{scope}
    % stick figure resting
    \begin{scope}[thick,line cap=round,xshift=9cm]
      \node[dot] at (.5,.35) {};
      \draw (.5,.25) -- (.5,-.05);
      \draw (.5,-.05) -- (.6,-.3);
      \draw (.6,-.3) -- (.65,-.3);
      \draw (.5,-.05) -- (.4,-.3);
      \draw (.4,-.3) -- (.35,-.3);
      \draw (.5,.25) -- (.65,.1);
      \draw (.65,.1) -- (.5,-.02);
      \draw (.5,.25) -- (.35,.15);
      \draw (.35,.15) -- (.212,.212);
    \end{scope}
  \end{tikzpicture}
\end{center}
How is that reconcilable with a precise notion of symmetry?

The answer to the first question clearly depends on the context.  If we allow reflections or even more exotic symmetries the answer is ``no''.  Each context has its own answer to what the symmetries of the square are.

Actually, the two questions should be seen as connected.  If a symmetry of $\square$ is like a journey (loop) in the world (type) of squares, what symmetries are allowed is dependent on how big a ``world of squares'' we consider.  Is it, for instance, big enough to contain a loop representing a reflection?

We argue that in order to pin down the symmetries of a thing (a ``shape''), all you need to do is specify
\begin{enumerate}
\item a type $X$ (of things)
\item the particular thing $x$ (in $X$).
\end{enumerate}
It is (almost) that simple!

Note that this presupposes that our setup is strong enough to support the notion of a ``loop'', which in the real world perhaps could be envisioned as a continuous path starting and ending at the same place.

Different setups have different advantages.  The theory of sets is an absolutely wonderful setup, but carrying out the above in sets requires at the very least developing some notions like ``topology'' and ``homotopy theory'', which (while fun and worthwhile in itself) is something of a detour.

The setup we adopt, ``HoTT'' or ``univalent foundations'', seems custom-built for such an approach.

In practice, one of the most important things is to be able to \emph{compare} symmetries of ``thing 1'' and ``thing 2''.  In our case this amounts to nothing but a function the takes thing 1 to thing 2.
\begin{quote}
  Picture of the function as in the original Goodnotes, putting some names to things.
\end{quote}
While such comparisons of symmetries are traditionally handled by something called a ``group homomorphism'' which is a function satisfying a rather long list of axioms, in our case the only thing we need to know of the function is that it really does take thing 1 to thing 2 -- everything else then follows naturally.

Some important examples have provocatively simple representations in this framework.  For instance, consider the circle
\begin{quote}
  picture of the pointed circle
\end{quote}
Since symmetries are interpreted as loops, you see that you have a loop for every integer -- the number $7$ can be represented by looping seven times counterclockwise.  As we shall see, in our setup any loop is naturally identified with a unique integer (the ``winding number'' if you will).  Everything you can wish to know about the structure of the ``group of integers'' is encoded in the circle.

Another example is the ``free group of words in two letters $a$ and $b$''.  This is represented by
\begin{quote}
  picture of figure eight
\end{quote}
(the word $ab^2a^{-1}$ is represented by looping around circle $a$ and $b$ respectively $1$, $2$ and $-1$ times in succession -- notice that since the $b^2$ is in the middle it prevents the $a$ and the $a^{-1}$ from meeting and cancelling each other out.  If you wanted the ``\emph{abelian} group on the letters $a$ and $b$'' (where $a$ and $b$ are allowed to move past each other), you should instead look at the torus
\begin{quote}
  picture of the torus
\end{quote}
Just why this last example works can remain a puzzle for now.

In some situations, the type $X$ ``of things'' can be more difficult to draw.  For instance, what is the ``type of all squares'' which we discussed earlier, representing all rotational symmetries of $\square$?  While it is harder to draw, you have perhaps already visualized it as the type of all squares in the plane, with $\square$ being the shape the loop must start and stop in.  In addition, we see that any symmetry can be reached by doing the $90$\textdegree-rotation a few of times, together with the fact that taking any loop four times reduces to not doing anything at all: it represents the ``cyclic group of order four''.


{\em From here I want to problematize when two loops are equal, univalence with the point of view that loops in the universe are conceptually simple, while equivalence gives a really compact, local description that is much easier to work with and that they describing the same thing gives a particular forceful setup

  Then I have a couple of topics I want to talk informally about.  In particular, Subgroups, finite groups, Burnside and group actions, counting, Sylow, Noether, Free groups and the more geometric material figuring prominently in the last part of the book (still being developed)..}






OLD

This book is about symmetry and its many manifestations in mathematics.
There are many kinds of symmetry and many ways of studying it.
Euclidean plane geometry is the study of properties that are invariant under rigid motions of the plane.
Other kinds of geometry arise by considering other notions of transformation.
Univalent mathematics gives a new perspective on symmetries:
Motions of the plane are forms of identifying the plane with itself in possibly non-trivial ways.
It may also be useful to consider different presentations of planes
(for instance as embedded in a common three-dimensional space)
and different identifications between them.
For instance, when drawing images in perspective
we identify planes in the scene with the image plane,
not in a rigid Euclidean way, but
rather via a perspectivity (see Fig.~?).
This gives rise to projective geometry.

Does that mean that a plane from the point of view of Euclidean
geometry is not the same as a plane from the point of view of
projective or affine geometry?
Yes.
These are of different types,
because they have different notions of identification,
and thus they have different properties.

Here we follow Quine's dictum: No entity without identity!
To know a type of objects is to know what it means to identify representatives of the type.
The collection of self-identifications (self-transformations) of a given object form a \emph{group}.

% TODO : Propositions, sets, and $1$-types (groupoids). (Here?)

Group theory emerged from many different directions in the latter half of the 19\th century.
Lagrange initiated the study of the invariants under permutations
of the roots of a polynomial equation $f(x)=0$,
which culminated in the celebrated work of Abel and Galois.
In number theory, Gauss had made detailed studies of modular arithmetic,
proving for instance that the group of units of $\ZZ/p\ZZ$ is cyclic.
Klein was bringing order to geometry by considering groups of transformation,
while Lie was applying group theory in analysis to the study of differential equations.

Galois was the first to use the word ``group'' in a technical sense,
speaking of collections of permutations closed under composition.
He realized that the existence of resolvent equation is equivalent
to the existence of a normal subgroup of prime index
in the group of the equation.

Groupoids vs groups.
The type of all squares in a euclidean plane form a groupoid.
It is connected,
because between any two there exist identifications between them.
But there is no canonical identification.

When we say ``the symmetry group of the square'',
we can mean two things:
1) the symmetry group of a particular square;
this is indeed a group,
or 2) the connected groupoid of all squares;
this is a ``group up to conjugation''.

Vector spaces. Constructions and fields. Descartes and cartesian geometry.

Klein's EP:
\begin{quote}
  Given a manifold and a transformation group acting on it,
  to investigate those properties of figures on that manifold
  that are invariant under transformations of that group.
\end{quote}
and
\begin{quote}
  Given a manifold, and a transformation group acting on it,
  to study its \emph{invariants}.
\end{quote}
Invariant theory had previously been introduced in algebra
and studied by Clebsch and Gordan.

(Mention continuity, differentiability, analyticity and Hilbert's 5\th problem?)

Any finite automorphism group of the Riemann sphere is conjugate to a
rotation group (automorphism group of the Euclidean sphere).
[Dependency: diagonalizability] (Any complex representation of a
finite group is conjugate to a unitary representation.)

% Groups up to conjugation: $\Gal(\bar\Q/Q)$?

All of mathematics is a tale, not about groups,
but about $\infty$-groupoids.
However, a lot of the action happens already with groups.

\newpage

\section*{Glossary of coercions}

MOVE TO BETTER PLACE
Throughout this book we will use the following coercions to make the text more readable.
\begin{itemize}[noitemsep]
\item If $X$ is the pointed type $(A,a)$, then $x:X$ means $x:A$.
\item On hold, lacking context: If $p$ and $q$ are paths, then $(p,q)$ means $(p,q)^=$.
\item If $e$ is a pair of a function and a proof, we also use $e$ for the function.
\item If $e$ is an equivalence between types $A$ and $B$, we use $\etop e$ for the
identification of $A$ and $B$ induced by univalence.
\item If $p: A= B$ with $A$ and $B$ types, then we use $\ptoe p$ for the canonical
equivalence from $A$ to $B$ (also only as function).
%\item If $G$ is the group $(A,a,p,q)$, then $g:G$ means $g: a=_A a$. %TODO: El
\item If $X$ is $(A,a,\ldots)$ with $a:A$, then $\pt_X$ and even just $\pt$ mean $a$.
\end{itemize}

\bigskip

\section*{How to read this book}

\ldots

\noindent\emph{A word of warning.}\enspace
We include a lot of figures to make it easier to follow the material.
But like all mathematical writing, you'll get the most out of it,
if you maintain a skeptical attitude:
Do the pictures really accurately represent the formal constructions?
Don't just believe us: Think about it!

The same goes for the proofs: When we say that something \emph{clearly} follows,
it should be \emph{clear to you}.
So clear, in fact, that you could go and convince a proof assistant,
should you so desire.

\section*{Acknowledgement}
The authors acknowledge the support of the Centre for Advanced Study (CAS)
at the Norwegian Academy of Science and Letters
in Oslo, Norway, which funded and hosted the research project Homotopy
Type Theory and Univalent Foundations during the academic year 2018/19.



%%% Local Variables:
%%% mode: latex
%%% fill-column: 144
%%% TeX-master: "book"
%%% End:

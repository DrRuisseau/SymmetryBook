\section{The Limited Principle of Omniscience}

\begin{remark}\label{rem:LPO-solves-halting problem}
We show that LPO is not constructively provable, the reason being that
"either $P$, or $Q$" is encoded constructively as $P\coprod Q$.\marginnote{%
The statement that something is not provable is a statement \emph{about}, 
and not \emph{in}, our theory. Proving such a statement often requires
properties of the theory that cannot be formulated nor proved in the theory itself.
One such `metaproperty' that we use here is \emph{canonicity}.
An example of canonicity is that every natural number that does not depend
on anything is a \emph{numeral}, that is, either $0$ or $S(n)$ for some numeral $n$.
Another example of canonicity is that every element of $P\coprod Q$ that 
does not depend on anything is either of the form $\inl{p}$ for some $p:P$ or 
of the form $\inr{q}$ for some $q:Q$.}

Our argument is based on the Halting Problem: given a Turing machine 
$M$ and an input $n$, determine whether $M$ halts on $n$.
It is known that the Halting Problem cannot be solved by an algorithm.
%that can be implemented on a Turing machine
We use a few more facts from computability theory.
First, Turing machines can be enumerated $M_0,M_1,\ldots$
Second, there exists a function $T(e,n,k)$ such that $T(e,n,k) = 1$
if $M_e$ halts on input $n$ in at most $k$ steps, and $T(e,n,k) = 0$
otherwise. This function $T$ can easily be implemented in our theory.

Towards a contradiction, assume we have a proof $p$ of LPO in our theory,
where $p$ does not depend on anything. In particular, $p$ does not
depend on any axiom.\footnote{It is possible to weaken the notion
of canonicity so that the argument still works even if $p$ uses $\ua$.}
It is clear that $k\mapsto T(e,n,k)$ is a constant function with value $0$
if and only if $M_e$ does not halt on input $n$. Now consider $p(k\mapsto T(e,n,k))$.
This is an element of a type of the form $P\coprod Q$. Let $e$ and $n$ be
arbitrary numerals. Then $p(k\mapsto T(e,n,k))$ does not depend on anything
and has one of the two canonical forms $\inl{p}$, representing the case
of not halting, or $\inr{q}$, halting. Since canonical forms
can be computed we would have an algorithm to solve the Halting Problem
for all $e$ and $n$. Contradiction.
\end{remark}

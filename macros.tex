%% document class
\documentclass[10pt,oneside,leqno,openright]{memoir}
%% packages
\usepackage{microtype}
\usepackage[T2A,T1]{fontenc}
\usepackage[utf8]{inputenc}
\usepackage[osf]{newpxtext}     % largsc
\renewcommand{\rmdefault}{pplx} % use FPL text for better sc
\usepackage[vvarbb]{newpxmath}
\DeclareSymbolFont{largesymbols}  {OMX}{zplm}{m}{n}
\makeatletter
\re@DeclareMathSymbol{\sumop}{\mathop}{largesymbols}{"50}
\re@DeclareMathSymbol{\prodop}{\mathop}{largesymbols}{"51}
\makeatother
\renewcommand*{\coprod}{\amalg}
\usepackage[cal=cm,scr=zapfc,scrscaled=1.2,frak=euler]{mathalfa}
\let\pxopenbox\openbox % pxmath already has openbox but amsthm also defines it
\let\openbox\relax
\usepackage{csquotes}
\usepackage[russian,ngerman,main=english,shorthands=off]{babel}
\usepackage{amsthm}
\renewcommand*{\openbox}{\pxopenbox}
\usepackage{mathtools}          %to get \vcentcolon
\usepackage{thmtools}
\usepackage{xspace}
\usepackage{ragged2e}           % to hyphenate marginals
\usepackage{marginfix}          % to get \mparshift
\usepackage{scalerel}           % to scale \div
\usepackage[all]{xy}
\usepackage{pgfplots,tikz,tikz-cd,tikz-3dplot}
\pgfplotsset{compat=1.11}
\usepackage{enumitem}
\usepackage{xifthen}

\input tikzsetup

%% mathtools does not define an extensible equal sign, so we do so here:
\makeatletter
\def\Equalfill@{\arrowfill@\Relbar\Relbar\Relbar}
\MHInternalSyntaxOn
\providecommand*\xEqual[2][]{%
  \ext@arrow 0055{\Equalfill@}{#1}{#2}}
\MHInternalSyntaxOff
\makeatother
\mathtoolsset{mathic}

% bibliography
\usepackage{hyphenat}
\usepackage[backref=true,
            style=verbose,
            autocite=footnote,
            autolang=hyphen,
            backend=biber,
            bibencoding=utf8]{biblatex}
\addbibresource{papers.bib}
\newcommand*{\citeerror}{\PackageError{cas}{Don't use \protect\cite\space: %
    use \protect\footcite\space instead,\MessageBreak %
    perhaps together with \protect\citeauthor}{}}
\let\cite\citeerror
\DefineBibliographyStrings{english}{%
  backrefpage = {page},% originally "cited on page"
  backrefpages = {pages},% originally "cited on pages"
}
\newcommand*{\arxiv}[1]{Preprint available at arXiv: \href{http://arxiv.org/abs/#1}{\nolinkurl{#1}}}

% hyperref should be the package loaded last
\usepackage[colorlinks,
            bookmarksnumbered=true,
            bookmarksdepth=2,
            bookmarksopenlevel=2,
            citecolor=linkcolor,
            linkcolor=linkcolor,
            urlcolor=linkcolor,
            unicode,
            pdfborderstyle=,
            pdfauthor={CAS},
            pdftitle={Symmetry},
            pdfsubject={Mathematics},
            pdfkeywords={type theory, group theory, univalence axiom}]{hyperref}
% - except for cleveref!
\usepackage[capitalize]{cleveref}

%% Colors: link color, and a red/blue pair that remains contrasting
%% for colorblind individuals (https://davidmathlogic.com/colorblind)
\definecolor{linkcolor}{rgb}{0,0,0.5}
\definecolor{casred}{rgb}{0.863,0.196,0.125}
\definecolor{casblue}{rgb}{0.0,0.353,0.71}

%% Enable synctex
\pdfadjustspacing=1
\brokenpenalty=10000 %%% No hyphenation across page breaks
\synctex=1

%% redefine \em to use \slshape
\makeatletter
\DeclareRobustCommand\em
  {\@nomath\em \ifdim \fontdimen\@ne\font >\z@
    \eminnershape \else \slshape \fi}%
\makeatother

%% macros
\newcommand{\DELETE}[1]{} % no star on purpose: argument may contain \par

%%% Lists (uses enumitem)
\setlist{itemsep=0.5ex}
\setlist[1]{labelindent=\parindent}
\setlist[1]{leftmargin=*}
\setlist[enumerate,1]{label=(\arabic*),ref=(\arabic*)}
\setlist[description]{font=\normalfont\scshape}

%%%%%%%%%%%%%%%%%%%%%%%%%%%%%%%%%%%%%%%%%%%%%%%%%%%%%%%%%%%%%%%%%%%%%%%%%%%%%
%%% THEOREMS
\newcommand*\qefsymbol{\ensuremath{\lrcorner}}
\declaretheoremstyle[headfont=\normalfont\scshape,bodyfont=\slshape]{cas-thm}
\declaretheoremstyle[headfont=\normalfont\scshape,qed=\qefsymbol]{cas-def}
\declaretheorem[sibling=subsection,style=cas-thm]{theorem}
\declaretheorem[sibling=theorem,style=cas-thm]{lemma}
\declaretheorem[sibling=theorem,style=cas-thm]{corollary}
\declaretheorem[sibling=theorem,style=cas-thm]{conjecture}
\declaretheorem[sibling=theorem,style=cas-thm]{axiom}
\declaretheorem[sibling=theorem,style=cas-thm]{construction}
\declaretheorem[sibling=theorem,style=cas-def]{definition}
\declaretheorem[sibling=theorem,style=cas-def]{remark}
\declaretheorem[sibling=theorem,style=cas-def]{example}
\declaretheorem[sibling=theorem,style=cas-def]{exercise}
\declaretheorem[sibling=theorem,style=cas-def,name=Exercise]{xca}
\declaretheorem[sibling=theorem,style=cas-def,name=Principle]{principle}
\def\implementation#1{\proof[Implementation of \cref{#1}]}
\def\endimplementation{\endproof}
\numberwithin{equation}{section}

%% end
%%%%%%%%%%%%%%%%%%%%%%%%%%%%%%%%%%%%%%%%%%%%%%%%%%%%%%%%%%%%%%%%%%%%%%%%%%%%%

%%% Palatino line width
\setlength{\normalrulethickness}{0.59pt}

%%% The occasional Russian
\newcommand{\casrus}[1]{\foreignlanguage{russian}{%
    \fontfamily{Tempora-TLF}\selectfont #1}}

%%% Tufte font size
%%
% Set the font sizes and baselines to match Tufte's books
\makeatletter
\renewcommand\normalsize{%
   \@setfontsize\normalsize\@xpt{14}%
   \abovedisplayskip 10\p@ \@plus2\p@ \@minus5\p@
   \abovedisplayshortskip \z@ \@plus3\p@
   \belowdisplayshortskip 6\p@ \@plus3\p@ \@minus3\p@
   \belowdisplayskip \abovedisplayskip
   \let\@listi\@listI}
\normalbaselineskip=14pt
\normalsize
\renewcommand\small{%
   \@setfontsize\small\@ixpt{12}%
   \abovedisplayskip 8.5\p@ \@plus3\p@ \@minus4\p@
   \abovedisplayshortskip \z@ \@plus2\p@
   \belowdisplayshortskip 4\p@ \@plus2\p@ \@minus2\p@
   \def\@listi{\leftmargin\leftmargini
               \topsep 4\p@ \@plus2\p@ \@minus2\p@
               \parsep 2\p@ \@plus\p@ \@minus\p@
               \itemsep \parsep}%
   \belowdisplayskip \abovedisplayskip
}
\renewcommand\footnotesize{%
   \@setfontsize\footnotesize\@viiipt{10}%
   \abovedisplayskip 6\p@ \@plus2\p@ \@minus4\p@
   \abovedisplayshortskip \z@ \@plus\p@
   \belowdisplayshortskip 3\p@ \@plus\p@ \@minus2\p@
   \def\@listi{\leftmargin\leftmargini
               \topsep 3\p@ \@plus\p@ \@minus\p@
               \parsep 2\p@ \@plus\p@ \@minus\p@
               \itemsep \parsep}%
   \belowdisplayskip \abovedisplayskip
}
\renewcommand\scriptsize{\@setfontsize\scriptsize\@viipt\@viiipt}
\renewcommand\tiny{\@setfontsize\tiny\@vpt\@vipt}
\renewcommand\large{\@setfontsize\large\@xipt{15}}
\renewcommand\Large{\@setfontsize\Large\@xiipt{16}}
\renewcommand\LARGE{\@setfontsize\LARGE\@xivpt{18}}
\renewcommand\huge{\@setfontsize\huge\@xxpt{30}}
\renewcommand\Huge{\@setfontsize\Huge{24}{36}}
\makeatother

%%% Layout
\stockaiv
\pageaiv
\settypeblocksize{*}{26pc}{*}
\setlrmargins{1in}{*}{*}
\setulmarginsandblock{1in}{2in}{*}
\setheadfoot{\baselineskip}{2\baselineskip}
\setheaderspaces{*}{2\baselineskip}{*}
\setmarginnotes{2pc}{11pc}{\baselineskip}
\checkandfixthelayout
\raggedbottom
\setlength{\parindent}{1em}

%%% Tufte margins
\marginparmargin{right}
\footnotesinmargin
\renewcommand*{\thefootnote}{\arabic{footnote}}
\setlength{\footmarkwidth}{-1sp}
\setlength{\footmarksep}{0pt}
\renewcommand{\foottextfont}{\footnotesize\RaggedRight}
\newlength{\extrawidth}
\setlength{\extrawidth}{\marginparsep}
\addtolength{\extrawidth}{\marginparwidth}
\newenvironment{fullwidth}{\begin{adjustwidth}{0mm}{-\extrawidth}%
  \blockmargin}%
  {\unblockmargin\end{adjustwidth}}
\setmpjustification{\RaggedLeft}{\RaggedRight}%
\newcommand*{\marginnote}[2][]{%
  \ifthenelse{\isempty{#1}}{\relax}{\mparshift{#1}}%
  \marginpar{\mpjustification\normalfont\footnotesize #2}}
\setmarginfloatcaptionadjustment{figure}{\captionnamefont{\footnotesize\scshape}%
  \captiontitlefont{\footnotesize}\captionstyle{\mpjustification}}
\sidecapmargin{right}
\setsidecappos{t}
\renewcommand*{\sidecapstyle}{\captionnamefont{\footnotesize\scshape}%
  \captiontitlefont{\footnotesize}\captionstyle{\mpjustification}}

%%% Space after period
\DeclareMathSymbol{.}{\mathpunct}{letters}{"3A}
\DeclareMathSymbol{\decimalperiod}{\mathord}{letters}{"3A}

%%% Title page
\newcommand{\thetitlepage}{{%
    \thispagestyle{empty}
    \noindent{\huge \textls[250]{SYMMETRY}}

    \vspace*{10ex}

    \noindent{\itshape\Large\foreignlanguage{ngerman}{Am Anfang war die Symmetrie}
      --
      In the beginning was symmetry!}\\

    \vspace{2ex}

    \noindent\hfill\parbox{.62\textwidth}{Werner Heisenberg,
      \emph{\foreignlanguage{ngerman}{Der Teil
        und das Ganze: Gespr\"ache im Umkreis der Atomphysik}}, 1969,
    English translation, \emph{Physics and Beyond}, 1971.}

    \vfill
    {\Large Version of \today}
    \vfill\clearpage}}

%%% Table of contents
\renewcommand*{\cftchapterfont}{\itshape\LARGE}
\renewcommand*{\cftchapterleader}{\kern1pc\textperiodcentered}
\renewcommand*{\cftchapterformatpnum}[1]{\kern1pc\normalfont\LARGE #1}
\renewcommand*{\cftchapterafterpnum}{\cftparfillskip}
\renewcommand*{\chapternumberline}[1]{\llap{#1\kern2pc}}
\renewcommand*{\cftsectionfont}{\normalfont}
\renewcommand*{\cftsectionleader}{\enskip\textperiodcentered}
\renewcommand*{\cftsectionformatpnum}[1]{\enskip\normalfont #1}
\renewcommand*{\cftsectionafterpnum}{\cftparfillskip}
\setrmarg{0pt plus 1fil}
\setlength{\cftsectionnumwidth}{3em}
\setlength{\cftsectionindent}{0pt}
\setpnumwidth{2pc}
\maxtocdepth{section}

%%% Headers
\setlength{\headwidth}{\textwidth}
\addtolength{\headwidth}{\marginparsep}
\addtolength{\headwidth}{\marginparwidth}
\makepagestyle{cas}
\makerunningwidth{cas}{\headwidth}
\makeheadposition{cas}{flushright}{flushleft}{}{}
\newcommand*{\headtitle}{\normalfont\scshape\textls{symmetry}}
\makeevenhead{cas}{\thepage\kern2pc\headtitle}{}{}
\makeoddhead{cas}{}{}{\headtitle\kern2pc\thepage}
\pagestyle{cas}

%%% Chapters
\makeatletter
\makechapterstyle{cas}{%
  \setlength{\beforechapskip}{2\onelineskip}%
  \setlength{\midchapskip}{3pt}%
  \setlength{\afterchapskip}{4\onelineskip \@plus .1\onelineskip
    \@minus 0.167\onelineskip}%
  \renewcommand*{\printchaptername}{}%
  \renewcommand*{\chapternamenum}{}%
  \renewcommand*{\chapnumfont}{\raggedright\normalfont\huge\itshape}%
  \renewcommand*{\chaptitlefont}{\raggedright\normalfont\huge\slshape}%
  \renewcommand*{\printchaptertitle}[1]{%
    \begin{adjustwidth}{}{-\extrawidth}
      \chaptitlefont ##1\par\nobreak
    \end{adjustwidth}}%
  \renewcommand*{\printchapternum}{\chapnumfont \thechapter}%
  \renewcommand*{\printchapternonum}{\raggedright}}
\makeatother
\chapterstyle{cas}

%%% Sectioning
\setsecnumdepth{subsection}
\maxsecnumdepth{subsection}
\setsecheadstyle{\normalfont\Large\slshape}
\setsubsecheadstyle{\normalfont\large\slshape}
\setsubsubsecheadstyle{\normalfont\slshape}

%%% Symbol index (Glossary)
\makeglossary
\changeglossnum{\thepage}
\changeglossnumformat{|hyperpage}
\renewcommand*{\glossaryname}{Symbol index}
\renewcommand{\glossitem}[4]{#1\enskip\textperiodcentered\enskip #2, #3 #4\\}
%\twocolglossary

%%% Index
\makeindex

% deprecated!!
\renewcommand*{\sc}{\scshape}
\renewcommand*{\bf}{}

% save old div and equiv
\let\olddiv\div
\renewcommand*{\div}{{\mathchoice%
  {\olddiv}%
  {\olddiv}%
  {\vcenter{\hbox{\fontsize{.5em}{0}\selectfont$\olddiv$}}}%
  {\vcenter{\hbox{\fontsize{.45em}{0}\selectfont $\olddiv$}}}%
}}%

% Meta-macros for:
\newcommand*{\constant}[1]{\mathrm{#1}} % defined constants / functions : roman
\newcommand*{\constructor}[1]{\mathrm{#1}} % constructors : roman
\newcommand*{\typeformer}[1]{\mathrm{#1}} % (uppercase) typeformers : roman
\newcommand*{\var}[1]{\mathit{#1}} % variable : italics
\newcommand*{\UU}{{\mathscr{U}}} % universes (special case) : calligraphic
\newcommand*{\UUp}{{\mathscr{U}_*}}

\newcommand*{\fakeslant}[1]{%
  \pdfliteral{1 0 0.167 1 0 0 cm}#1\pdfliteral{1 0 -0.167 1 0 0 cm}}
\newcommand*{\bn}[1]{{%
  \if!\ifnum9<1#1!\else_\fi%
  \mathbb{#1}\else\fakeslant{\mathbb{#1}}\fi}}

% our version of \operatorname with a font change (so we can use the appropriate meta-macro)
\makeatletter
\newcommand*{\casop}[1]{\mathop{\newmcodes@\kern\z@ #1}\nolimits@}
\makeatother

% Typeformers
\newcommand*{\bool}{\typeformer{Bool}}
\newcommand*{\charstring}{\typeformer{String}}
\newcommand*{\integer}{\typeformer{Int}}
\newcommand*{\real}{\typeformer{Real}}
\newcommand*{\true}{\typeformer{True}}
\newcommand*{\false}{\typeformer{False}}
\newcommand*{\Prop}{\typeformer{Prop}}
\newcommand*{\Set}{\typeformer{Set}}
\newcommand*{\Groupoid}{\typeformer{Groupoid}}
\newcommand*{\FinSet}{\typeformer{FinSet}}
\newcommand*{\InfCycSet}{\typeformer{InfCycSet}}
\newcommand*{\FinCycSet}{\typeformer{FinCycSet}}
\newcommand*{\GSet}[1][G]{\mathord{#1\textrm{-}\Set}}
\newcommand*{\Group}{\typeformer{Group}}
\newcommand*{\AbGroup}{\typeformer{AbGroup}}
\newcommand*{\Bunch}{\typeformer{Bunch}}
\newcommand*{\AbBunch}{\typeformer{AbBunch}}
\newcommand*{\Band}{\typeformer{Band}}
\newcommand*{\AbBand}{\typeformer{AbBand}}
\newcommand*{\Monoid}{\typeformer{Monoid}}
\newcommand*{\Tors}{\typeformer{Torsor}}
\newcommand*{\Copy}{\casop{\typeformer{Copy}}}
\newcommand*{\Heap}{\typeformer{Heap}}

\newcommand*{\zet}{\typeformer{Z}} % the SET of integers
\newcommand*{\QQ}{\mathbb{Q}}
\newcommand*{\ZZ}{\mathbb{Z}}
\newcommand*{\NN}{\mathbb{N}}
\newcommand*{\NNN}{\mathbb{N}^{-}} % negated natural numbers
\newcommand*{\CC}{\mathbb{C}}
\newcommand*{\RR}{\mathbb{R}}
\newcommand*{\ii}{\constant{i}} % unit complex imaginary
\newcommand*{\emptytype}{\emptyset}
\newcommand*{\UUscone}{\UU_\ast^{=1}} % pointed connected groupoids
\newcommand*{\UUsctwo}{\UU_\ast^{=2}} % pointed 1-connected 2-types
\newcommand*{\UUpconn}{\UU_\ast^{>0}} % pointed connected types
\newcommand*{\UUppone}{\UU_{{\ast}{\ast}}^{=1}} % bipointed conn gpds

% Constructors
\newcommand*{\yes}{\constructor{yes}}
\newcommand*{\no}{\constructor{no}}
\newcommand*{\triv}{\constructor{triv}}
\newcommand*{\refl}[1]{\constructor{refl}_{#1}}
\newcommand*{\rrfl}{\constructor{rrfl}}
\newcommand*{\inl}[1]{\casop{\constructor{inl}_{#1}}}
\newcommand*{\inr}[1]{\casop{\constructor{inr}_{#1}}}
\newcommand*{\inc}[1]{\casop{\constructor{in}_{#1}}}
\newcommand*{\zeq}{\constructor{zeq}}

% Functions and defined elements
\newcommand*{\refloi}[1]{\casop{\constant{refl}^{-o}_{#1}}} % exception
\newcommand*{\fact}{\casop{\constant{fact}}}
\newcommand*{\id}{\mathord{\constant{id}}}
\newcommand*{\pt}{\constant{pt}}
\newcommand*{\shape}{\constant{sh}} % the basepoint of the classifying type of a group G, referred to as the designated shape of G.  Note: "\sh" is already in use.
\newcommand*{\ad}{\constant{ad}}
\newcommand*{\symm}{\casop{\constant{symm}}}
\newcommand*{\trans}{\casop{\constant{trans}}}
\newcommand*{\trp}[2][]{\casop{\constant{trp}^{#1}_{#2}}}
\newcommand*{\fst}{\casop{\constant{fst}}}
\newcommand*{\snd}{\casop{\constant{snd}}}
\newcommand*{\zpos}{\casop{\constant{pos}}}
\newcommand*{\zneg}{\casop{\constant{neg}}}
\newcommand*{\zzero}{\casop{\constant{zero}}}
\newcommand*{\preim}{\casop{\constant{preim}}}
\newcommand*{\tot}{\casop{\constant{tot}}}
\newcommand*{\funext}{\casop{\constant{funext}}}
\newcommand*{\ptw}{\casop{\constant{ptw}}}
\newcommand*{\ap}[1]{\casop{\constant{ap}_{#1}}}
\newcommand*{\apd}[1]{\casop{\constant{apd}_{#1}}}
\newcommand*{\apap}[3]{\casop{\constant{apap}_{#1}(#2)(#3)}}
\newcommand*{\apc}{\casop{\constant{ap}{\ct}}}
\newcommand*{\ns}{\casop{\constant{ns}}}
\newcommand*{\ev}{\casop{\constant{ev}}}
\newcommand*{\pow}[1]{\casop{\constant{pow}_{#1}}} % power bundle/map
\newcommand*{\ve}{\casop{\constant{ve}}} % cute: the inverse of \ev
\newcommand*{\out}{\casop{\constant{out}}} % projection from copy
\newcommand*{\permgrp}[1]{\Sigma_{{#1}}}%
\newcommand*{\cast}{\casop{\constant{cast}}}
\newcommand*{\etop}[1]{\bar {#1}}   % equivalence to path
\newcommand*{\overetop}[1]{\overbracket[0.76pt][-1pt]{#1}}
\newcommand*{\casoverline}[1]{\overbracket[0.59pt][-1pt]{#1}}
\newcommand*{\ptoe}[1]{\tilde {#1}} % path to equivalence
\newcommand*{\ua}{\casop{\constant{ua}}} % univalence
\newcommand*{\cst}[1]{\casop{\constant{cst}}_{#1}} % constant function at
\newcommand*{\N}{\constant{N}} % normalizer
\newcommand*{\BAut}{\casop{\constant{BAut}}}
\newcommand*{\Aut}{\casop{\constant{Aut}}}
\newcommand*{\aut}{\Aut}
\newcommand*{\mkaut}{\casop{\underline{\constant{Aut}}}}
\newcommand*{\Out}{\casop{\constant{Out}}}
\newcommand*{\Inn}{\casop{\constant{Inn}}}
\newcommand*{\Ker}{\casop{\constant{Ker}}}
\renewcommand*{\ker}{\casop{\constant{ker}}}
\newcommand{\typekernel}{{\constant{Ker}}}
\newcommand*{\Img}{\casop{\constant{Im}}}
\newcommand*{\img}{\casop{\constant{im}}}
\newcommand*{\image}{\Img}
\newcommand{\prj}{\constant{pr}}
\newcommand{\prjim}{\prj^{\img}}
\newcommand*{\incl}{\constant{in}}%the homomorphism in an element #1 in \Mono_G
\newcommand*{\Coker}{\casop{\constant{Coker}}}
\newcommand*{\coker}{\casop{\constant{coker}}}
\newcommand*{\Hom}{\casop{\constant{Hom}}}
\newcommand*{\Nor}{\constant{Nor}}
\newcommand{\nor}{{\casop{\constant{nor}}}}
\newcommand*{\Sym}{\casop{\constant{Sym}}}
\newcommand*{\USym}{\constant{U}} % underlying symmetries (no space!) formerly \sh_G=\hs_G
\newcommand*{\US}[1]{\USym #1}%changes e.g f^\abstr changed to to Uf
\newcommand*{\ISym}{\constant{I}} % underlying identities
\newcommand*{\Card}{\casop{\constant{Card}}}
\newcommand*{\bunch}{\casop{\constant{bunch}}}
\newcommand*{\Ad}{\casop{\constant{Ad}}}
\newcommand*{\band}{\casop{\constant{band}}}
\newcommand*{\fiber}{\casop{\constant{fiber}}}
\newcommand*{\zs}{\casop{\constant{s}}} % successor on the integers
\newcommand*{\Succ}{\casop{\constant{succ}}}
\newcommand*{\Pred}{\casop{\constant{pred}}}
\newcommand*{\fin}{\casop{\constant{fin}}} % This is the operator, not the type, which is \FinSet
\newcommand*{\El}{\casop{\constant{El}}}
\newcommand*{\clf}{\constant{B}} % classifying space operator (no space)
\newcommand*{\B}{\constant{B}}            % without extra space
\newcommand*{\BB}{\B^2}
\newcommand*{\grpcenter}{\casop{\constant{Z}}}
\newcommand*{\im}{\casop{\constant{im}}}
\newcommand*{\iscontr}{\casop{\mathrm{isContr}}}
\newcommand*{\isprop}{\casop{\mathrm{isProp}}}
\newcommand*{\isset}{\casop{\mathrm{isSet}}}
\newcommand*{\isgrpd}{\casop{\mathrm{isGrpd}}}
\newcommand*{\isfinset}{\casop{\mathrm{isFinSet}}}
\newcommand*{\isEq}{\casop{\mathrm{isEquiv}}}
\newcommand*{\isonetype}{\casop{\mathrm{is1Type}}}
\newcommand*{\isconn}{\casop{\mathrm{isConn}}}
\newcommand*{\istrans}{\casop{\mathrm{isTrans}}}
\newcommand*{\isinj}{\casop{\mathrm{isInj}}}
\newcommand*{\ismono}{\casop{\mathrm{isMono}}}
\newcommand*{\isepi}{\casop{\mathrm{isEpi}}}


\newcommand*{\fundgrp}{\casop{\pi_1}}
\newcommand*{\fundgrpd}{\casop{\Pi_1}}

\newcommand*{\dg}[1]{\casop{\delta_{#1}}} % degree m map on circle
\newcommand*{\cdg}[1]{\casop{\rho_{#1}}} % degree m map on inf.cyc.sets (root)
\newcommand*{\mathdegree}{\textup{\textdegree}}

% Common multi-letter variables
% A bit of a hack. If we use \var{...} we get a big box
% with the wrong sub-/super-script positions
\newcommand*{\BG}{\mathit{B}\mathit{G}}
\newcommand*{\BH}{\mathit{B}\mathit{H}}
\newcommand*{\BK}{\mathit{B}\mathit{K}}
\newcommand*{\BL}{\mathit{B}\mathit{L}}
\newcommand*{\BM}{\mathit{B}\mathit{M}}
\newcommand*{\BN}{\mathit{B}\mathit{N}}
\newcommand*{\BW}{\mathit{B}\mathit{W}}
\newcommand*{\Bf}{\mathit{B}\mathit{f}}
\newcommand*{\Bg}{\mathit{B}\mathit{g}}
\newcommand*{\Bh}{\mathit{B}\mathit{h}}
\newcommand*{\Bi}{\mathit{B}\mathit{i}}
\newcommand*{\Bj}{\mathit{B}\mathit{j}}
\newcommand*{\Bk}{\mathit{B}\mathit{k}}
\newcommand*{\Bp}{\mathit{B}\mathit{p}}

% Latin abbreviations and initialisms
\newcommand*{\cf}{cf.~}
\newcommand*{\ie}{i.e., }
\newcommand*{\eg}{e.g., }
\newcommand*{\viz}{viz., }

% disputed phrases
\newcommand*{\covering}{set bundle\xspace}
\newcommand*{\coverings}{set bundles\xspace}
\newcommand*{\Covering}{Set bundle\xspace}
\newcommand*{\Coverings}{Set bundles\xspace}
\newcommand*{\inftygp}{$\infty$-group\xspace}
\newcommand*{\aninftygp}{an $\infty$-group\xspace}
\newcommand*{\inftygps}{$\infty$-groups\xspace}
\newcommand*{\gporder}{cardinality\xspace}
\newcommand*{\gporders}{cardinalities\xspace}
\renewcommand*{\th}{\textsuperscript{th}\xspace}
\newcommand*{\st}{\textsuperscript{st}\xspace}
\newcommand*{\nd}{\textsuperscript{nd}\xspace}
\newcommand*{\rd}{\textsuperscript{rd}\xspace}

% relations
\let\oldequiv\equiv
\renewcommand*{\equiv}{\simeq}
\newcommand*{\weq}{\simeq}
\newcommand*{\we}{\xrightarrow{\sim}}
\newcommand*{\isom}{\cong}
\newcommand*{\liff}{\weq}
\newcommand*{\jdeq}{\oldequiv}
\newcommand*{\defeq}{\mathrel{\vcentcolon\jdeq}}
\newcommand*{\defequi}{\defeq} % definitionally equal
\newcommand*{\defis}{\mathrel{\vcentcolon=}}

% binary operations
\newcommand*{\ct}{*}
\newcommand*{\cto}{*_{\constant{o}}}
\newcommand*{\dblslash}{\mathbin{/\kern-3pt/}}

% paired delimiters (can change size via optional argument)
\DeclarePairedDelimiter\Trunc{\lVert}{\rVert} % truncated type
\DeclarePairedDelimiter\trunc{\lvert}{\rvert} % truncation constructor
\DeclarePairedDelimiter\merely{\lVert}{\rVert_{-1}}
\DeclarePairedDelimiter\angled{\langle}{\rangle}
\DeclarePairedDelimiter\Fin[]
\DeclarePairedDelimiter\set\lbrace\rbrace
\DeclarePairedDelimiterX\setof[2]\lbrace\rbrace{#1 \mid #2}

% special truncations and truncation constructors
\newcommand*{\nonempty}[1]{\Trunc{#1}}
\newcommand*{\setTrunc}[1]{{\Trunc{#1}}_0}
\newcommand*{\settrunc}[1]{{\trunc{#1}}_0}
\newcommand*{\grpdTrunc}[1]{{\Trunc{#1}}_1}
\newcommand*{\grpdtrunc}[1]{{\trunc{#1}}_1}

% footnote without number, no star: may contain \par
\newcommand\blfootnote[1]{%
  \begingroup
  \renewcommand\thefootnote{}\footnote{#1}%
  \addtocounter{footnote}{-1}%
  \endgroup}

% deprecated macros
\newcommand*{\princ}[1]{\constant{Pr}_{{#1}}}
\newcommand*{\pathsp}[1]{\constant{P}_{\!#1}} % NB negative thin space
\newcommand*{\uc}[1]{{\pathsp{#1}}}%universal set bundle
\newcommand*{\abstr}{\constant{abs}}
\newcommand*{\agp}[1]{\mathcal #1} %generic abstract group
\newcommand*{\grpcenterinc}[1]{\mathrm z_{{#1}}} %

\newcommand*{\pre}{\constant{pre}}%these may be open for discussion
\newcommand*{\preinv}{\constant{preinv}}
\newcommand*{\post}{\constant{post}}
\newcommand*{\adjoint}{\ad}
\newcommand*{\concr}{\constant{concr}}

%some special types
\newcommand*{\Gtorsor}{\Tors_G}
\newcommand*{\Xtorsor}[1]{\Tors_{#1}}%      added by MAB
\newcommand*{\Ztorsor}{\Xtorsor{\zet}}
\newcommand*{\TorZ}{{\constant{T}\ZZ}} % {\mathbf{TorZor}}% alternative: BZ.
\newcommand*{\SetBundle}{\constant{SetBundle}}             % end MAB
\newcommand*{\typegroup}{\Group}
\newcommand*{\typeabgroup}{\AbGroup}
\newcommand*{\typesubgroup}{\constant{Sub}}%"gp" removed - is evident from the type of the subscript G
\newcommand*{\typemono}{\constant{Mono}}%monomorphisms into a group (subscript)
\newcommand*{\typeepi}{\constant{Epi}}%epimorphisms out of a group (subscript)
\newcommand*{\typenormal}{\constant{Nor}}
\newcommand*{\typeset}{\Set}
\newcommand*{\typeinftygp}{{\infty}\Group}
\newcommand*{\typemonoid}{\Monoid}
\newcommand*{\typetorsor}{\Tors}
\newcommand*{\pttype}{\UUp}
\newcommand*{\typeabsgp}{\Group_{\textup{Abstract}}}
\newcommand*{\BSigma}{\B\Sigma}%previously \Set_{(S)} - the component of S:\Set
\newcommand*{\twist}{\constant{twist}}%loop in BC_2
\newcommand*{\Sc}{{\typeformer{S}^1}}%the circle
%\newcommand{\sbt}{\begin{picture}(-1,1)(-1,-3)\circle*{2}\end{picture}}%
\newcommand*{\sbt}{\mathchoice%
  {\vcenter{\hbox{\scriptsize\textbullet}}}%
  {\vcenter{\hbox{\scriptsize\textbullet}}}%
  {\vcenter{\hbox{\tiny\textbullet}}}%
  {\vcenter{\hbox{\fontsize{4}{5}\selectfont\textbullet}}}}% {\mathop{\bullet}}
%{\tikz[anchor=base,baseline]{\node[scale=.7,inner
%    sep=0, outer sep=0, circle]%
%    {$\bullet$};}}%
\newcommand*{\base}{{\sbt}}%point in circle
\newcommand*{\Zloop}{\circlearrowleft} %was \mathrm{loop}}%MAB: loop TorZor
\newcommand*{\Sloop}{\circlearrowleft}%loop in circle

\newcommand*{\conncomp}[2]{{{#1}_{\left(#2\right)}}}%
\newcommand*{\univcover}[2]{{{#1}^0_{\left(#2\right)}}}%

\newcommand*{\blank}{\_}%
\newcommand*{\inv}[1]{#1^{-1}}%
\newcommand*{\invo}[1]{#1^{-1\mathop{\constant{o}}}}%mathop deliberately to center the o
\newcommand*{\ptdto}{\to_\ast}%
\newcommand*{\ptdweq}{\weq_\ast}%
\newcommand*{\mono}{\hookrightarrow}%
\newcommand*{\epi}{\twoheadrightarrow}%
\newcommand*{\loops}[1][\null]{\Omega^{#1}} % no space
\newcommand*{\mkgroup}[1][\null]{\underline{\Omega}^{#1}}
\newcommand*{\mkhom}[1][\null]{\underline{\Omega}^{#1}}
\newcommand*{\mkheap}{\underline{\constant{I}}}
\newcommand*{\cdoto}{\cdot^{\constant{o}}} % concat in pathover

%% paths over paths
\newcommand*{\pathover}[4]{\mathchoice{{#1} \xEqual[#3]{#2} #4}%
  {#1 =^{#2}_{#3} #4}%
  {#1 =^{#2}_{#3} #4}%
  {#1 =^{#2}_{#3} #4}}
\newcommand*{\po}{\casop{\constant{po}}}
\newcommand*{\pair}{\casop{\constructor{pair}}}
\newcommand*{\rec}{\casop{\constant{rec}}}
\newcommand*{\ind}{\casop{\constant{ind}}}
\newcommand*\pathpair[2]{\overline{({#1},{#2})}}

%% Euclidean geometry

\newcommand*{\EE}{\mathbb E}
\newcommand*{\VV}{\mathbb V}
\newcommand*{\ES}{{\tilde \EE}}
\newcommand*{\EucObj}{\casop{\constant{EucObj}}}
\newcommand*{\OS}{{\tilde \VV}}
\newcommand*{\OrthGp}[1]{\constant{O}(#1)}
\newcommand*{\EucGp}[1]{\constant{E}(#1)}
\newcommand*{\Vectors}{\casop{\constant{Vec}}}
\newcommand*{\Points}{\casop{\constant{Pts}}}
\newcommand*{\typeRealVectorSpace}{\constant{Vect}_{\RR}}

%%%%%%%%%%%%%%%%%%%%%%%%%%%%%%%%%%%%%%%%%%%%%%%%%%%%%%%%%%%%%%%%%%%%%%%%%%%%

% Peter & Benedikt's macros for referring to coqdoc
% d2c4e86
% see https://tex.stackexchange.com/a/35314/ for help understanding the following:

\newcommand{\longhash}{e47ce20acce953129e34e021a10976ed27948a39}
\newcommand{\shorthash}{e47ce20}

%fragile, better to freeze with stable hash
\newcommand{\coqdocbasebaseurl}{https://unimath.github.io/doc/UniMath/\shorthash/}

%\coqident call are relative to this long path
\newcommand{\coqdocbaseurl}{\coqdocbasebaseurl UniMath.}
\newcommand{\urlhash}{\#}

\newcommand{\coqdocurl}[2]{\coqdocbaseurl #1.html\urlhash #2}

%nolinkurl from url or hyperref package
\newcommand{\nolinkcoqident}[1]{\nolinkurl{#1}} % TODO: give better def for this?
\makeatletter
\newcommand{\coqident}{\begingroup\@makeother\#\@coqident}
\newcommand{\@coqident}[3][]{% empty default first and optional argument
  \ifthenelse{\isempty{#2}}%
  {\nolinkcoqident{#3}}%           [optional]{}{printme}
  {\ifthenelse{\isempty{#1}}%
  {\href{\coqdocurl{#2}{#3}}{\nolinkcoqident{#3}}}% []{file}{identifier+printed}
  {\href{\coqdocurl{#2}{#3}}{\nolinkcoqident{#1}}}}% [printme]{file}{identifier}
\endgroup}
% optional argument allows link text to differ from link url
\newcommand{\coqfile}[2]{%
  \ifthenelse{\isempty{#1}}%
  {\href{\coqdocbaseurl #2.html}{#2.v}}%
  {\href{\coqdocbaseurl #1.#2.html}{#2.v}}}
\makeatother

%sususe is a replacement for subsubsection.  sususe is the same as subsubsection but numbers correctly bid
%\newenvironment{sususe}[1]{\refstepcounter{theorem}%
%\vspace{.5\baselineskip}\par\medskip\noindent%
%{\normalfont\normalsize\bfseries{\thetheorem. #1}}%
%\vspace{.5\baselineskip}\newline}
\let\sususe\subsubsection



% Pierre Cagne macros
  \newcommand{\loopspace}[1][]{\constructor{Aut}^2_{#1}}

%%% Local Variables:
%%% mode: latex
%%% TeX-master: "book"
%%% End:

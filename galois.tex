\chapter{Galois theory}%
\label{chap:galois-theory}%



\section{Covering spaces and fields extensions}
\label{sec:cover-spac-fields}

\def\fieldstype{\mathbf{Fields}}%
\def\Gal{\mathrm{Gal}}%
\def\fieldshom{\hom_{\fieldstype}}%
\def\isHom{\mathrm{isHom}}%
\def\iso{\mathrm{Iso}}%
\newcommand\fieldsext[1]{#1\backslash\fieldstype}%
Recall that a field extension is simply a morphism of fields $e: K\to L$ from a
field $K$ to a field $L$. Given a fixed field $K$, the type of fields
extensions of $K$ is defined as
\begin{displaymath}
  \fieldsext K \defequi \sum_{L:\fieldstype}\fieldshom(K,L)
\end{displaymath}

\begin{definition}
  The Galois group of an extension $(L,e)$ of a field $K$, denoted $\Gal(L/K)$ when $e$ is clear from context,
  is the group
  $\Aut_{\fieldsext K}{(L,e)}$.
  \label{def:galois-group}
\end{definition}

\begin{remark}
  \label{rem:sip-univalence}
  The Structure Identity Principle holds for fields, which means that for
  $K,L:\fieldstype$, one has
  \begin{displaymath}
    (K = L) \weq \iso(K,L)
  \end{displaymath}
  where $\iso(K,L)$ denotes the type of these equivalences that are
  homomorphisms of fields. Indeed, if one uses $K$ and $L$ also for the carrier
  types of the fields, one gets:
  \begin{displaymath}
    \begin{split}
      (K = L) \weq \sum_{p:K=_\UU L}  (\trp p (+_K) = +_L)
      \times (\trp p (\cdot_K) = \cdot_L)
      \\ \times (\trp p (0_K) = 0_L)
      \times (\trp p (1_K) = 1_L) 
    \end{split}
  \end{displaymath}
  Any $p : K =\UU L$ is the image under univalence of an equivalence $\phi: K \weq L$, and then:
  \begin{align*}
    \trp p (+_K) &= (x,y) \mapsto \phi( \inv\phi(x) +_K \inv \phi(y)) \\
    \trp p (\cdot_K) &= (x,y) \mapsto \phi( \inv\phi(x) \cdot_K \inv \phi(y)) \\
    \trp p (0_K) &=\phi( 0_K ) \\
    \trp p (1_K) &=\phi( 1_K )
  \end{align*}
  It follows that:
  \begin{displaymath}
    \begin{split}
      (K=L) \weq \sum_{\phi: K \weq L} (\phi(x +_K y) = \phi (x) +_L \phi (y)) \\
      \times (\phi(x\cdot_K y) = \phi (x) \cdot_L \phi (y)) \\
      \times (\phi(0_K) = 0_L)
        \times (\phi(1_K) = 1_L)
    \end{split}
  \end{displaymath}
  The type on the right handside is exactly the definition of $\iso(K,L)$.

  In particular, given an extension $(L,e)$ of $K$:
  \begin{align*}
    \US \Gal(L/K) \weq \sum_{p:L=L} \trp p e = e \weq \sum_{\phi:\iso(L,L)} \phi \circ e = e
  \end{align*}
  This is how the Galois group of an extension is defined in ordinary mathematics.
\end{remark}

\begin{definition}
  A field extension $e:K\to L$ is saif finite when $L$ as a
  $K$-vector space, the structure of which is given by $e$, is of finite dimension.
  In that case, the dimension is called the degree of $e$, denoted $[L:K]$ when $e$ is clear from context.
  \label{defn:degree-field-extension}
\end{definition} 

\section{Intermediate extensions and subgroups}
%
Given two extensions $e: k \to K$ and $f: K \to L$, the resulting extension
$fe:k \to L$ is of special interest.  In particular, there is a pointed map
\begin{displaymath}
  e^\ast: \B\Gal(L/K) \to \B \Gal(L/k),\quad x\mapsto x\circ e.
\end{displaymath}
\begin{lemma}
  The map $e^\ast$ is a set-bundle.
  \label{lem:intermediate-ext-to-subgroup}
\end{lemma}
\begin{proof}
  By connectedness, one can afford to only show that the fiber at $fe$ is a
  set. An element in this fiber consists of an extension $x:K \to X$ together
  with a path $p:L=X$ such that $pfe=xe$. Given another element given by $x':K
  \to X'$ and $p:L=X'$, one wants to prove that the identity type
  $(x,p)=(x',p')$ in the fiber of $e^\ast$ over $fe$ is a proposition.
  
  An element of $(x,p)=(x',p')$ is a path from $x$ to $x'$ as extensions of
  $K$, the transport of $p$ over which is $p'$. In other words, it is the data
  of $\alpha: X=X'$ such that $\alpha x = x'$ and $\alpha p = p'$. This last
  condition shows that $\alpha$ is necessarily equal to $p'\inv p$. Hence,
  every two elements of $(x,p)=(x',p')$ are equal.
\end{proof}

Then, the set-bundle $e^\ast$ presents the group $\Gal(L/K)$ as a subgroup of $\Gal(L/k)$.

Conversely, given an extension $e:k \to L$ and a connected set-bundle $f:B \to \B \Gal(L/k)$, one considers the type
\begin{displaymath}
  \sum_{k:}
\end{displaymath}

\section{separable/normal/etc.}
\label{sec:cover-spac-fields-1}

\section{fundamental theorem}
\label{sec:fundamental-theorem}




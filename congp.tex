\chapter{Constructing groups}
\label{ch:congp}

\section{Brief overview of the chapter}

\section{Semidirect products}
\label{sec:Semidirect-products}{\color{red} just moved without proofreading BID 211116. Problem with $\USym$ followed by composite MAB 240815.}

In this section we describe a generalization of the product of two group, called the {\em semidirect} product, which can be constructed from an
action of a group on a group.  Like the product, it consists of pairs, both at the level of concrete groups and of abstract groups, as we shall
see.

We start with some preliminaries on paths between pairs.
Lemma \cref{lem:isEq-pair=} above takes a simpler form when $y$ and $y'$ are values of a family $x \mapsto f(x)$
of elements of the family $x \mapsto Y(x)$, as the following lemma shows.

\begin{lemma}\label{lem:pathpairsection}
  Suppose we are given a type $X$ and a family of types $Y(x)$ parametrized by the elements $x$ of $X$.
  Suppose we are also given a function $f : \prod_{x:X} Y(x)$.
  For any elements $x$ and $x'$ of $X$,
  there is an equivalence of type
  $$\left ( (x,f(x)) = (x',f(x')) \right ) \weq (x=x') \times (f(x) = f(x)),$$
  where the identity type on the left side is between elements of $\sum_{x:X} Y(x)$.
\end{lemma}

\begin{proof}
  By \cref{lem:isEq-pair=} and by composition of equivalences, it suffices to establish an equivalence of type
  $$\left( \sum_{p:x=x'} \pathover {f(x)} Y p {f(x')} \right) \weq (x=x') \times (f(x) = f(x)).$$
  Rewriting the right hand side as a sum over a constant family, it suffices to find an equivalence of type
  $$\left( \sum_{p:x=x'} \pathover {f(x)} Y p {f(x')} \right) \weq \sum_{p:x=x'} (f(x) = f(x)).$$
  By \cref{lem:fiberwise} it suffices to establish an equivalence of type
  $$ \left( \pathover {f(x)} Y p {f(x')} \right) \weq (f(x) = f(x))$$
  for each $p:x=x'$.  By induction on $x'$ and $p$ we reduce to the case where $x'$ is $x$ and $p$ is $\refl x$, and it suffices to establish an
  equivalence of type
  $$ \left( \pathover {f(x)} Y {\refl x} {f(x)} \right) \weq (f(x) = f(x)).$$
  Now the two sides are equal by definition, so the identity equivalence provides what we need.
\end{proof}

The lemma above shows how to rewrite certain paths between pairs as pairs of paths.  Now we wish to establish the formula for composition of
paths, rewritten in terms of pairs of paths, but first we introduce a convenient definition for the transport of loops in $Y(x)$ along paths in
$X$.

\begin{definition}\label{def:pathsectionaction}
  Suppose we are given a type $X$ and a family of types $Y(x)$ parametrized by the elements $x$ of $X$.
  Suppose we are also given a function $f : \prod_{x:X} Y(x)$.
  For any elements $x$ and $x'$ of $X$ and for any identity $p : x = x'$, define a function $(f(x') = f(x')) \to (f(x) = f(x))$, to be denoted
  by $q' \mapsto {q'} ^ p$, by induction on $p$ and $x'$, reducing to the case where $x'$ is $x$ and $p$ is $\refl x$, allowing us to
  set ${q'} ^{ \refl x } \defeq q'$.
\end{definition}

We turn now to associativity for the operation just defined.

\begin{lemma}\label{def:pathsectionactionassoc}
  Suppose we are given a type $X$ and a family of types $Y(x)$ parametrized by the elements $x$ of $X$.
  Suppose we are also given a function $f : \prod_{x:X} Y(x)$.
  For any elements $x$, $x'$, and $x''$ of $X$, for any identities $p : x = x'$ and $p' : x' = x''$,
  and for any $q : f x'' = f x''$,
  there is an identification of type $ ( q ^{ p' }) ^ p = q ^{( p' \cdot p )}$.
\end{lemma}

\begin{proof}
  By induction on $p$ and $p'$, it suffices to show that $ ( q ^{ \refl y }) ^ { \refl y } = q ^{( \refl y \cdot \refl y )}$, in which both sides are
  equal to $q$ by definition.
\end{proof}

Observe that the operation depends on $f$, but $f$ is not included as part of the notation.

The next lemma contains the formula we are seeking.

\begin{lemma}\label{lem:pathpairsectionmult}
  Suppose we are given a type $X$ and a family of types $Y(x)$ parametrized by the elements $x$ of $X$.
  Suppose we are also given a function $f : \prod_{x:X} Y(x)$.
  For any elements $x$, $x'$, and $x''$ of $X$, and for any two identities $e : (x,f(x)) = (x',f(x'))$ and $e' : (x',f(x')) = (x'',f(x''))$,
  if $e$ corresponds to the pair $(p,q)$ with $p : x = x'$ and $q : f x = f x$ under the equivalence of \cref{lem:pathpairsection},
  and $e'$ corresponds to the pair $(p',q')$ with $p' : x' = x''$ and $q' : f x' = f x'$,
  then $e' \cdot e$ corresponds to the pair $(p' \cdot p , ({q'} ^ p) \cdot q)$.
\end{lemma}

\begin{proof}
  By induction on $p$ and $p'$ we reduce to the case where $x'$ and $x''$ are $x$ and $p$ and $p'$ are $\refl x$.
  It now suffices to show that $e' \cdot e$ corresponds to the pair $(\refl x , q' \cdot q)$.
  Applying the definition of the map $\Phi$ in the proof of \cref{lem:isEq-pair=} to our three pairs, we see that it suffices to show that
  $\left( \apap g {\refl x} {q'} \right) \cdot \left( \apap g {\refl x} {q} \right) = \apap g {\refl x} {q' \cdot q}$, with $g$, as there, being the function $ g(x)(y) \defeq (x,y)$.
  By \cref{def:applfun2comp} it suffices to show that $\left( \ap {g(x)} {q'} \right) \cdot \left( \ap {g(x)} {q} \right) = \ap {g(x)} {(q' \cdot q)}$, which follows from
  compatibility of $\ap {g(x)}$ with composition, as in~\cref{lem:apcomp}.
\end{proof}

The lemma above will be applied mostly in the case where $x'$ and $x''$ are $x$, but if it had been stated only for that case, we would not have
been able to argue by induction on $p$ and $p'$.

\begin{definition}\label{def:semidirect-product}
  Given a group $G$ and an action $\tilde H : \BG \to \typegroup$ on a group $H \defeq \tilde H(\shape_G)$, we define a group called the {\em
    semidirect product} as follows.
  $$G \ltimes \tilde H \defeq \mkgroup { \sum_{t:\BG} \B \tilde H(t) }$$
  Here the basepoint of the sum is taken to be the point $(\shape_G,\shape_H)$.
  (We deduce from \cref{lem:level-n-utils}, \cref{level-n-utils-sum}, that $\sum_{t:\BG} \B \tilde H(t)$ is a groupoid.
  See \cref{xca:connected-trivia-1} for a proof that
  $\sum_{t:\BG} \B \tilde H(t)$ is connected.)
\end{definition}

Observe that if the action of $G$ on $H$ is trivial, then $\tilde H(t) \jdeq H$ for all $t$ and $G \ltimes \tilde H \jdeq G \times H$.

Projection onto the first factor gives a homomorphism $p \defeq \mkgroup \fst : G \ltimes \tilde H \to G$.
Moreover, there is a homomorphism $s : G \to G \ltimes \tilde H$ defined by
$ s \defeq \mkgroup {\left( t \mapsto (t,\shape_{\tilde H(t)}) \right) }$, for $t : \B G$.
The two maps are homomorphisms because they are made from basepoint-preserving maps.
The map $s$ is a \emph{section} of $p$ in the sense the $p \circ s = \id_G$.
There is also a homomorphism $j : H \to G \ltimes \tilde H$ defined by $j \defeq \mkgroup { \left( u \mapsto (\shape_G,u) \right) }$, for $u : \B H$.

\begin{lemma}
  The homomorphism $j$ above is a monomorphism, and it gives the same (normal) subgroup of $G \ltimes \tilde H$ as the kernel $\ker p$ of $p$.
\end{lemma}
\footnote{{\color{red}MUST BE MOVED TO THE SUBGROUP CHAPTER}}

\begin{proof}
  See \ref{def:kernel} for the definition of kernel.  According to \cref{lem:fst-fiber(a)=B(a)}, the map $\B H \to (\B p)^{-1}(\shape_G)$ defined by
  $ u \mapsto ((\shape_G,u), \refl{\shape_G}) $ is an equivalence.  This establishes that the fiber $(\B p)^{-1}(\shape_G)$ is connected and thus serves as
  the classifying type of $\ker p$.  Pointing out that the composite map $H \xrightarrow{\isom} \ker p \to G \ltimes \tilde H$ is $j$ and using
  univalence to promote the equivalence to an identity gives the result.
\end{proof}

Our next goal is to present the explicit formula for the multiplication operation in $\USym { G \ltimes \tilde H }$.
First we apply \cref{lem:pathpairsection} to get a bijection $\USym { G \ltimes \tilde H } \weq \USymG \times \USymH$.
Now use that to transport the multiplication operation of the group $\USym { G \ltimes \tilde H }$ to the set $\USymG \times \USymH$.
Now \cref{lem:pathpairsectionmult} tells us the formula for that transported operation is given as follows.
$$ (p',q') \cdot (p,q) = (p' \cdot p , ({q'} ^ p) \cdot q) $$
In a traditional algebra course dealing with abstract groups, this formula is used as the definition of the multiplication operation
on the set $\USymG \times \USymH$, but then one must prove that the operation satisfies the properties of \cref{def:abstractgroup}.
The advantage of our approach is that the formula emerges from the underlying logic that governs how composition of paths works.


\section{The pullback}
\label{sec:pullback}
Given two functions $f:B\to D$ and $g:C\to D$ with common target, the ``pullback'' which we will now define should be thought about as the type of all pairs of elements $(b,c):B\times C$ so that $f(b)=g(c)$.  This construction is important in many situations also beyond group theory.

\begin{definition}
  \label{def:pullback}
  Let $B, C, D$ be types and let $f:B\to D$ and $g:C\to D$ be two maps.
The \emph{pullback}\index{pullback} of $f$ and $g$ is the type
$$\prod(f,g)\defequi\sum_{(b,c):B\times C}(f(b)=_Dg(c))$$
together with the two projections $\prj_B:\prod(f,g)\to B$ and $\prj_C:\prod(f,g)\to C$ sending $(b,c,p):\prod(f,g)$ to $b:B$ or $c:C$.  If $f$ and $g$ are clear from the context, we may write $B\times_DC$ instead of $\prod(f,g)$ and summarize the situation by the diagram
$$\xymatrix{B\times_DC\ar[r]^{\prj_C}\ar[d]^{\prj_B}&C\ar[d]^g\\B\ar[r]^f&\,D.}$$
\end{definition}
\begin{xca}
  \marginnote{Illustrating the exercise: if the solid diagram commutes there is a unique dotted arrow so that the resulting diagram commutes:
    $$\xymatrix{A\ar@{.>}[dr]\ar[drr]\ar[ddr]&&\\
    &B\times_DC\ar[r]\ar[d]&C\ar[d]\\&B\ar[r]&D}
  $$}
  \label{xca:univpropofpullback}
  Let $f:B\to D$ and $g:C\to D$ be two maps with common target.  If $A$ is a type show that
  \begin{align*}
    (A\to B)\times_{(A\to D)}(A\to C)\to &(A\to B\times_DC)\\
(\beta,\gamma,p:f\beta=g\gamma)\,\mapsto\,&(a\mapsto (f(a),g(a),p(a):f\beta(a)=g\gamma(a)))
  \end{align*}
 is an equivalence.
\end{xca}
In view of \cref{xca:univpropofpullback} we will say that we have a \emph{pullback diagram}\index{pullback diagram}
$$\xymatrix{A\ar[d]^{g'}\ar[r]^{f'}&C\ar[d]^g\\B\ar[r]^f&D}$$
to indicate that we have an element in $(A\to B)\times_{(A\to D)}(A\to C)$ such that the resulting map $A\to B\times_DC$ is an equivalence.

\begin{example}
  \marginnote{Preimage as a pullback: $$\xymatrix{f^{-1}(d)\ar[d]\ar[r]&\bn1\ar[d]^d\\B\ar[r]^f&D}$$}
  If $g:\bn 1\to D$ has value $d:D$ and $f:B\to D$ is any map, then $\prod(f,g)\oldequiv B\times_D\bn 1$ is equivalent to the preimage $f^{-1}(d)\defequi\sum_{b:B}d=f(b)$.
\end{example}
\begin{example}
  \label{ex:pullbackandgcd}
  Much group theory is hidden in the pullback.  For instance, the greatest common divisor $\gcd(a,b)$ of $a,b:\NN$ is another name for the number of components you get if you pull back the $a$-fold and the $b$-fold \coverings of the circle: we % will see in \cref{lem:iso2} we
  have a pullback
$$\xymatrix{S^1\times\B C_{\gcd(a,b)}\ar[d]\ar[r]& S^1\ar[d]^{(-)^b}\\
S^1\ar[r]^{(-)^a}&\,S^1}
$$
(where $C_n$ was the cyclic group of order $n$).
To get a geometric idea, think of the circle as the unit circle in the complex numbers so that the $a$-fold \covering is simply taking the $a$-fold power.  With this setup, the pullback should consist of pairs $(z_1,z_2)$ of unit length complex numbers with the property that $z_1^a=z_2^b$.  Let $a=a'\gcd(a,b)$ and $b=b'\gcd(a,b)$. Taking an arbitrary unit length complex number $z$, then the pair $(z^{b'},z^{a'})$ is in the pull back (since $a'b=ab'$).  But so is $(\zeta z^{b'},z^{a'})$, where $\zeta$ is any $\gcd(a,b)$\th root of unity.  Each of the $\gcd(a,b)$-choices of $\zeta$ contributes in this way to a component of the pullback.  In more detail: identifying the cyclic group $C_{\gcd(a,b)}$ of order $\gcd(a,b)$ with the group of $g$\th roots of unity, the top horizontal map $S^1\times C_{\gcd(a,b)}\to S^1$ sends $(z,\zeta)$ to $z^{a'}$ and the left vertical map sends $(z,\zeta)$ to the product $\zeta z^{b'}$.

Also the least common multiple $\lcm(a,b)=a'b$ is hidden in the pullback; in the present example it is demonstrated that the map(s) across the diagram makes each component of the pullback a copy of the $\lcm(a,b)$-fold \covering.
\end{example}


\begin{definition}
  \label{def:intersectionand unionofsets}
  Let $S$ be a set and consider two subsets $A$ and $B$ of $S$ given by two families of propositions (for $s:S$) $P(s)$ and $Q(s)$.  The \emph{intersection}\index{intersection! of sets} $A\cap B$ of the two subsets is given by the family of propositions $P(s)\times Q(s)$.  The \emph{union}\index{union of sets} $A\cup B$ is given by the set family of propositions $A(s)+B(s)$.
\end{definition}
\begin{xca}
  \label{xca:intersectionpullbackofsets}
  Given two subsets $A$, $B$ of a set $S$, prove that
  \begin{enumerate}
  \item The pullback $A\times_SB$ maps by an equivalence to the intersection $A\cap B$,
  \item\label{xca:cardinalityintersectionunion}
    If $S$ is finite, then the sum of the cardinalities of $A$ and $B$ is equal to the sum of the cardinalities of $A\cup B$ and $A\cap B$.\qedhere
  \end{enumerate}
\end{xca}

\begin{definition}
  \label{def:intersectionofgroups}
  Let $f:\Hom(H,G)$ and $f':\Hom(H',G)$ be two homomorphisms with common target.  The \emph{pullback}\index{pullback!of groups} $H\times_GH'$ is the group obtained as the (pointed) component of
$$\pt_{H\times_GH'}\defequi(\shape_H,\pt_{H'},p_{f'}p_f^{-1})$$ of the pullback $\BH\times_{\BG}\BH'$ (where $p_f:\shape_G=f(\shape_H)$ is the name we chose for the data displaying $f$ as a pointed map, so that $p_{f'}p_f^{-1}:f(\shape_H)=f'(\pt_{H'})$).

If $(H,f,!)$ and $(H',f',!)$ are monomorphisms into $G$, then the pullback is called the \emph{intersection}\index{intersection! of monomorphisms} and if the context is clear denoted simply $H\cap H'$.
\end{definition}
\begin{example}
  If $a,b:\NN$ are natural number with least common multiple $L$, then $L\ZZ$ is the intersection $a\ZZ\cap b\ZZ$ of the subgroups $a\ZZ$ and $b\ZZ$ of $\ZZ$.
\end{example}
% \begin{example}this came out wrong DELETE June
%   If $H,K:\typemono_G$ with $X,Y:\BG\to\Set$ being the corresponding transitive $G$-sets under the equivalence $E$, then the intersection of $H$ and $K$ corresponds to the $G$-set $X\times Y:\BG\to\Set$ (with $(X\times Y)(x)\defequi X(x)\times Y(x)$).
% \end{example}

\begin{xca}
  Prove that if $f:\Hom(H,G)$ and $f':\Hom(H',G)$ are homomorphisms,
  then the pointed version of \cref{xca:univpropofpullback} induces an equivalence
  $$
  \Hom(K,H)\times_{\Hom(K,G)}\Hom(K,H')\simeq \Hom(K,H\times_GH')
  $$
  for all groups $K$ and an equivalence
  \[
    \USymH \times_{\USymG} \USymH'
    \simeq (\shape_{H\times_GH'}=\shape_{H\times_GH'}).\text{\footnotemark}
  \]
  Elevate the last equivalence to a statement about abstract groups.\footnotetext{%
    Hint: set $A\defequi \Sc$, $B\defequi \BH$, $C\defequi \BH'$ and $D\defequi \BG$.}
\end{xca}

\begin{remark}
  The pullback is an example of when a construction of types \emph{not} preserving connectivity can be used profitably also for groups.
  We get the pullback of groups by restricting to a pointed component, but also the other components have group theoretic importance.
  We will return to this when discussing subgroups.
\end{remark}

\section{Pushouts of types}
\label{sec:pushouts}

(TBW)

\section{Sums of groups}
\label{sec:coprod}
We have seen how the group of integers $\ZZ=(S^1,\base)$ synthesizes the notion of one symmetry with no relations: every symmetry in the circle is of the form $\Sloop^n$ for some unique $n$.  Also, given any group $G=\Aut_A(a)$, the set $a=a$ of symmetries of $a$ corresponds to the set of homomorphisms $\ZZ\to G$, \ie to pointed functions $(S^1,\base)\to_*(A,a)$ by evaluation at $\Sloop$.  What happens if we want to study more than one symmetry at the time?

For instance, is there a group $\ZZ\vee\ZZ$ so that for any group $G=\Aut_A(a)$ a homomorphism $\ZZ\vee\ZZ\to G$ corresponds to \emph{two} symmetries of $a$?
At the very least, $\ZZ\vee\ZZ$ itself would have to have two symmetries and these two can't have any relation, since in a general group $G=\Aut_A(a)$ there is a priori no telling what the relation between the symmetries of $a$ might be.
Now, \emph{one} symmetry is given by a pointed function $(S^1,\base)\to_*(A,a)$ and so a \emph{pair} of symmetries is given by a function $f:S^1+S^1\to A$ with the property that $f$ sends each of the base points of the circles to $a$.  But $S^1+S^1$ is not connected, and so not a group.  To fix this we take the clue from the requirement that both the base points were to be sent to a common base point and \emph{define} $S^1\vee S^1$ to be what we get from $S^1+S^1$ when we \emph{insert an identity} between the two basepoints.
\marginnote{$S^1\vee S^1$ if formed from $S^1+S^1$ by inserting an identity$$\xymatrix{\base\ar@(ul,dl)[]|{\Sloop}\ar@{.>}[rr]^{\text{identify!}}&&\base\ar@(ur,dr)[]|{\Sloop}}
  $$}

The amazing thing is that this works -- an enormous simplification of the classical construction of the ``free products'' or ``amalgamated sum'' of groups.  We need to show that the ``wedge'' $S^1\vee S^1$ is indeed a group, and this proof simultaneously unpacks the classical description.

We start by giving a definition of the wedge construction which is important for pointed types in general and then prove that the wedge of two groups is a group whose symmetries are arbitrary ``words'' in the original symmetries.

\begin{definition}
  \label{def:wedge}
  Let $(A_1,a_1)$ and $(A_2,a_2)$ be pointed types.  The \emph{wedge}\index{wedge of pointed types} is the pointed type $(A_1\vee A_2,a_{12})$ given as a higher inductive type by
  \begin{enumerate}
  \item functions $i_1:A_1\to A_1\vee A_2$ and $i_2:A_2\to A_1\vee A_2$
  \item an identity $g:i_1a_1=i_2a_2$.
  \end{enumerate}
We point this type at $a_{12}\defequi i_1a_1$.
  The function
$$i^g_2:(a_2=_{A_2}a_2)\to(a_{12}=_{A_1\vee A_2}a_{12})$$
is defined by $i^g_2(p)\defequi g^{-1}i_2(p)g$, whereas (for notational consistency only) we set $i_1^g\defequi i_1:(a_1=_{A_1}a_1)\to(a_{12}=_{A_1\vee A_2}a_{12})$.
Simplifying by writing $i:A_1+A_2\to A_1\vee A_2$ for the function given by $i_1$ and $i_2$ (with basepoints systematically left out of the notation),
the induction principle is
$$\prod_{C:(A_1\vee A_2)\to\UU}\sum_{s:\prod_{a:A_1+A_2}Ci(a)}
((s(a_1)=C(g^{-1})s(a_2))\,\to\,\prod_{x:(A_1\vee A_2)}C(x)).$$
\end{definition}

Unraveling the induction principle we see that if $B$ is a pointed type, then a  pointed function $f:A_1\vee A_2\to_* B$ is given by providing pointed functions $f_1:A_1\to_* B$ and $f_2:A_2\to_* B$  -- the identity $f_1(a_1)=f_2(a_2)$ which seems to be missing is provided by the requirement of the functions being pointed.  For the record
\begin{lemma}
  \label{lem:univvee}
  If $B$ is a pointed type, then the function
  $$i^*:(A_1\vee A_2\to_*B)\to(A_1\to_*B)\times(A_2\to_*B),\qquad i^*(f)=(fi_1,fi_2)
$$
is an equivalence.
\end{lemma}



To the right you see a picture of $i_2^g(p)$: \marginnote{$$
\xy (-20,20)*+{};(-20,20)*+{}
**\crv{(15,20)&(18,20)&(-10,35)&(10,45)&(25,30)&(20,19)&(0,20)}
%?>*\dir{>}
?(0)*{} *!LD!/^-20pt/{i_1A_1}
?(.45)*{} *!LD!/^2pt/{>}
?(.95)*{} *!LD!/^-15pt/{g^{-1}}
?(.03)*{} *!LD!/^-5pt/{g}
?(.55)*{} *!LD!/^-7pt/{i_2p}
?(.65)*{} *!LD!/^-30pt/{i_2A_2}
?(.87)*{} *!LD!/^-12pt/{i_2a_2}
?(.86)*{} *!LD!/^-2pt/{\bullet}
?(1)*{} *!LD!/^-2pt/{\bullet}
?(1)*{} *!LD!/^-12pt/{a_{12}}
\endxy
$$}it is the symmetry of the base point $a_{12}\defequi i_1a_1$ you get by \emph{first} moving to $i_2a_2$ with $g$, \emph{then} travel around with $p$ ($i_2p$, really) and finally go home to the basepoint with the inverse of $g$.

\marginnote{The idea is that an identity in $a_{12}=x$ can be factored into a string of identities, each lying solely in $A_1$ or in $A_2$.  We define a family of sets consisting of exactly such strings of identities --  it is a set since $A_1$ and $A_2$ are groupoids -- and prove that it is equivalent to the family $P(x)\defequi(a_{12}=_{A_1\vee A_2}x)$ which consequently must be a family of sets.
We need to be able to determine whether a symmetry is reflexivity or not, but once we know that, the symmetries of the base point in the wedge are then given by ``words $p_0p_1\dots p_n$'' where the $p_j$ alternate between being symmetries in the first or the second group, and none of the $p_j$ for positive $j$ are allowed to be reflexivity.  Note that there order of the $p_j$s is not negotiable: if I shuffle them I get a new symmetry.}


\begin{definition}
  \label{def:sumofgroup}
  If $G_1=\Aut_{A_1}(a_1)$ and $G_2=\Aut_{A_2}(a_2)$ are groups, then their \emph{sum}\index{sum of groups} is defined as
  $$G_1\vee G_2\defequi \Aut_{A_1\vee A_2}(a_{12}).$$
  The homomorphisms $i_1:G_1\to G_1\vee G_2$ and $i_2:G_2\to G_1\vee G_2$ induced from the structure maps  $i_1:A_1\to A_1\vee A_2$ and  $i_2:A_2\to A_1\vee A_2$ are also referred to as \emph{structure maps}.
\end{definition}
\begin{lemma}
  \label{lem:sumofgroupsISsum} If $G_1$, $G_2$ and $G$ are groups, then the function
  $$\Hom(G_1\vee G_2,G)\to\Hom(G_1,G)\times\Hom(G_2,G)$$
given by restriction along the structure maps is an equivalence.
\end{lemma}
\begin{proof}
  This is a special case of \cref{lem:univvee}.
\end{proof}
Specializing further, we return to our initial motivation and see that mapping out of a wedge of two circles \emph{exactly} captures the information of two independent symmetries:
\begin{corollary}
  \label{cor:ZplusZuniv}
  If $G$ is a group, then the functions
  $$\Hom(\ZZ\vee\ZZ,G)\to \Hom(\ZZ,G)\times\Hom(\ZZ,G)\simeq \USymG\times \USymG$$
  is an equivalence.
\end{corollary}

\begin{xca}\label{xca:whatAREabeliangroups}
  This leads to the following characterization of abelian groups\marginnote{%
    \wip{Do we know at this point that this extension problem is a proposition,
      \ie that the wedge inclusion is an epimorphism?
      Check re pointedness.}}
  formulated purely in terms of pointed connected groupoids
  (with no direct reference to identity types).
  A group $G$ is abelian if and only if the canonical map\marginnote{%
    \begin{tikzcd}[ampersand replacement=\&]
      \BG\vee\BG\ar[r,"\fold"]\ar[d,"\text{inclusion}"'] \& \BG \\
       \BG\times\BG\ar[ur,dashed]
    \end{tikzcd}}
  \[
    \fold :\BG\vee \BG\ptdto \BG
  \]
  (given via \cref{lem:sumofgroupsISsum} by $\id_G:G\to G$)
  extends over the inclusion
  \[
    i:\BG\vee \BG\ptdto \BG\times \BG
  \]
  (given by the inclusions $\incl_1,\incl_2:G\to G\times G$).

  As a cute aside, one can see that the required map $\BG\times\BG\ptdto\BG$
  actually doesn't need to be pointed:
  factoring
  $\fold:\BG\vee\BG\to\BG$ over $i:\BG\vee\BG\to\BG\times\BG$
  -- even in an unpointed way -- kills all ``commutators''
  $ghg^{-1}h^{-1}:\USym(G\vee G)$.
  (\wip{is this still a proposition, or do we need to truncate?})
\end{xca}

We end the section by proving that wedges of decidable groups are decidable groups and that they can be given the classical description in terms of words.


\begin{lemma}
  \label{lem:wedgeofgpoidisgpoid}
  Let $G_1\defequi\Aut_{A_1}(a_1)$ and $G_2\defequi\Aut_{A_2}(a_2)$ be decidable groups, then the wedge sum $G_1\vee G_2\defequi\Aut_{A_1\vee A_2}(a_{12})$ is a decidable group.

Let $C_1$ be the set of strings $(p_0,n,p_1,\dots,p_n)$ with $n:\NN$ and, for $0\leq j\leq n$
\begin{itemize}
\item $p_{j}:\USymG_1%\defequi (a_1=a_1)
  $  for even $j$
\item $p_{j}:\USymG_2%\defequi (a_2=a_2)
  $ for odd $j$ and
\item $p_j$ is not reflexivity for $j$ positive
\end{itemize}
(the last requirement makes sense and is a proposition since our groups are decidable).

  Then the function given by composition in $\USymG_{12}\defequi(a_{12}=a_{12})$
  $$\beta:C_1\to\USymG_{12}%\defequi(a_{12}=a_{12})
  ,\qquad\beta(p_0,n,p_1,\dots p_n)\defequi i_1^gp_0i_2^gp_1i_1^gp_2\dots i_?^gp_n$$
(where $i_?^gp_n$  is $i_1^gp_n$ or $i_2^gp_n$ according to whether $n$ is even or odd) is an equivalence.
\end{lemma}
\begin{proof}
  That the wedge is connected follows by transitivity of identifications,
  if necessary passing through the identification $g:i_1a_1=i_2a_2$ in the wedge.

We must prove that the wedge is a groupoid, \ie that all identity types are sets, which we do by giving an explicit description of the universal \covering.

 We use the notation of \cref{def:wedge} freely, and for ease of notation, let $a_{2k+i}\defequi a_i$ and $i_{2k+i}^g\defequi i_i^g$ for $i=1,2$, $k:\NN$.
Define families of sets
$$C_i:A_i\to\Set,\qquad i=1,2$$
by
$$
C_i(x)\defequi(a_i=_{A_i}x)\times
\sum_{n:\NN}\prod_{1\leq k\leq n}\sum_{p_k:a_{i+k}= a_{i+k}}(p_k\neq\refl {a_{i+k}})
$$
when $x:A_i$.  Note that $p_k\neq\refl{a_{i+k}}$  is a proposition; we leave it out when naming elements. Hence, an element in $C_1(a)$ is a tuple
$(p_0,n,p_1,\dots,p_n)$ where $p_0:a_1=_{A_1}a$, $p_1:a_2=_{A_2}a_2$, $p_2:a_1=_{A_1}a_1$, and so on -- alternating between symmetries of $a_1$ and $a_2$, and where $p_0$ is the only identity allowed to be $\refl{}$. Define $C_{12}:C_1(a_1)\to C_2(a_2)$ by
$$C_{12}(p_0,n,p_1\dots,p_n)=
\begin{cases}
  (\refl{a_2}0,)&\text{ if }p_0=\refl{a_1}, n=0,\\
  (p_1,n-1,p_2\dots,p_n)& \text{ if }p_0=\refl{a_1},n\neq0,\\
  (\refl{a_2},n+1,p_0,\dots,p_n)& \text{ if }p_0\neq\refl{a_1}.
\end{cases}
$$
It is perhaps instructive to see a table of the values $C_{12}(p_0,n,p_1,\dots,p_n)$ for $n<3$:
\begin{center}
  \begin{tabular}{r|c cc}
    &$(p_0,0)$&$(p_0,1,p_1)$&$(p_0,2,p_1,p_2)$\\
    \hline
    $p_0=\refl{a_1}$&$(\refl{a_2},0)$&$(p_1,0)$&$(p_1,1,p_2)$\\
    $p_0\neq\refl{a_1}$&$(\refl{a_2},1,p_0)$&$(\refl{a_2},2,p_0,p_1)$&$(\refl{a_2},3,p_0,p_1,p_2)$
  \end{tabular}
\end{center}
Since $C_{12}$ is an equivalence, the triple $(C_1,C_2,C_{12})$ defines a family
$$C:A_1\vee A_2\to\Set.$$
In particular, $C(a_{12})\defequi C_1(a_1)$.
For $x:A_1$ we let $i^C_1:C_1(x)\to C(i_1(x))$ be the induced equivalence, and likewise for $i^C_2$.
We will show that $C$ is equivalent to $P\defequi \pathsp{a_{12}}$, where $P(x)\defequi(a_{12}=x)$, and so that the identity types in the wedge are equal to the sets provided by $C$.

One direction is by transport in $C$; more precisely,
$$\alpha:\prod_{x:A_1\vee A_2}(P(x)\to C(x))$$ is given by transport with $\alpha(a_{12})(\refl{a_{12}})\defequi(\refl{a_{1}},0):C(a_{12})$.
The other way,
$$\beta:\prod_{x:A_1\vee A_2}(C(x)\to P(x))$$ is given by composing identities, using the glue $g$ to make their ends meet:
$$\beta(i_1a)(p_0,n,p_1,\dots,p_n)\defequi i_1(p_0)i_2^g(p_1)i_3^g(p_2) \dots i_{n+1}^g(p_n)$$
(here the definition $\dots i_3^g\defequi i_1^g\defequi i_1$ proves handy since we don't need to distinguish the odd and even cases)
and likewise
$$\beta(i_2a)(p_0,n,p_1,\dots,p_n)\defequi i_2(p_0)g\,i_1^g(p_1)i_2^g(p_2) \dots i_{n}^g(p_n)$$ and compatibility with the glue $C_{12}$ is clear since the composite $\refl{x}p$ is equal to $p$.

For notational convenience, we hide the $x$ in $\alpha(x)(p)$ and $\beta(x)(p)$ from now on.

That $\beta\alpha(p)=p$ follows by path induction: it is enough to prove it for $x=a_{12}$ and
$p\defequi\refl{a_{12}}$:
$$\beta\alpha(\refl{a_{12}})=\beta(\refl{a_1},0)=i_1^g\refl{a_1}=\refl{a_{12}}.$$

That $\alpha\beta(p_0,n,p_1\dots,p_n)=(p_0,n,p_1,\dots,p_n)$ follows by induction on $n$ and $p_0$.  For $n=0$ it is enough to consider  $x=a_{12}$ and $p_0=\refl{a_1}$, and then
$\alpha\beta(\refl{a_1},0)\defequi\alpha(\refl{a_{12}})\defequi(\refl{a_1},0)$.  In general, (for $n>0$)
\begin{align*}
  \alpha\beta(p_0,n,p_1\dots,p_n)
=&\trp[C]{i_1(p_0)i_2^g(p_1)i_1^g(p_2) \dots i_{n+1}^g(p_n)}(\refl{a_1,0})\\
=&\trp[C]{i_1(p_0)}\dots\trp[C]{i_{n+1}^g(p_n)}(\refl{a_1,0}).
\end{align*}
  The induction step is as follows: let $0< k\leq n$, then
\begin{align*}
  &\trp[C]{i_k^gp_{k-1}}i^C_{k-1}(p_k,n-k-1,p_{k+1},\dots,p_n)\\
  =&\trp[C]{i_k^gp_{k-1}}i^C_k(\refl{a_{k-1}},n-k,p_k,\dots,p_n)\\
  =&i^C_k\trp[C_k]{p_{k-1}}(\refl{a_{k-1}},n-k,p_k,\dots,p_n)\\
  =&(p_{k-1},n-k,p_k,\dots,p_n).
\end{align*}
%((please see whether this makes sense to anybody but yvt))
\end{proof}

\section{Free groups}
\label{sec:freegroups}

We have seen in~\cref{ex:Zinitial} that the group of integers $\ZZ$
is the free group on one generator in the sense that the set of homomorphisms
from $\ZZ$ to any group $G$ is equivalent (by evaluation at the loop)
to the underlying set of symmetries in $G$, $\USymG$.
This set is of course equivalent (by evaluation at the unique element)
to the set of maps $(\bn 1 \to \USymG)$.

Likewise, we have seen in~\cref{cor:ZplusZuniv} that the binary sum $\ZZ \vee \ZZ$
is the free group on two generators, corresponding to the left and right summands.

In general, a free group on a set of generators $S$ is a group $\FG_S$ with
specified elements $\iota_s:\UFG_S$ labeled by $s:S$,
such that evaluation gives an equivalence $\Hom(\FG_S,G) \equivto (S \to \USymG)$
for each group $G$.

We now give a definition of the classifying type of a free group
as a higher inductive type that is very much like that of the circle,
except that instead of having a single generating loop,
it has a loop $\Sloop_s$ for each element $s:S$.
\begin{definition}
  \label{def:bfree}
  Fix a set $S$.
  The classifying type of the free group on $S$, $\BFG_S$,%
  \index{free group}%
  \glossary(FS){$\protect\FG_S$}{free group on a decidable set of generators}
  is a type with a point $\base : \BFG_S$ and
  a constructor ${\Sloop_\blank} : S \to \base\eqto\base$.

  Let $A(x)$ be a type for every element $x:\BFG_S$.
  The induction principle for $\BFG_S$ states that,
  in order to define an element of $A(x)$ for every $x:\BFG_S$,
  it suffices to give an element $a$ of $A(\base)$ together
  with an identification $l_s:\pathover{a}{A}{\Sloop_s}{a}$
  for every $s:S$.
  The function $f$ thus defined satisfies $f(\base)\jdeq a$
  and we are provided identifications $\apd{f}(\Sloop_s)\eqto l_s$ for each $s:S$.

  We define the \emph{free group} on $S$ as $\FG_S \defeq \mkgroup(\BFG_S,\base)$.
\end{definition}

A priori, $\FG_S$ is only an \inftygp.
Nevertheless, we get immediately from the induction principle
that evaluation at the elements of $S$
gives an equivalence $\Hom(\FG_S,G) \equivto (S \to \USymG)$ for each
\inftygp $G$.

In order to see that $\FG_S$ is a group,
we need to know that $\BFG_S$ is a groupoid.
This follows from a general theorem on identifications in pushouts due to~\textcite{Warn2023}.
Here we restrict our discussion to decidable sets $S$,
where we can give a more concrete proof.

We can follow that same strategy as in~\cref{lem:univisexp}
and~\cref{lem:wedgeofgpoidisgpoid}
and show this by giving a description of $\FG_S$ as an \emph{abstract} group.
To see what this should be, think about what symmetries of $\base$
we can write using the constructors $\Sloop_s$ for $s:S$.
We can compose these out of $\Sloop_s$ and $\Sloop_s^{-1}$
with various generators $s$.
However, if we at any point have $\Sloop_s\Sloop_s^{-1}$ or $\Sloop_s^{-1}\Sloop_s$,
then these cancel.
This motivates the following definitions.

\begin{definition}
  Fix a decidable set $S$. Let $\tilde S \defeq S + S$%
  \index{signed set}%
  \glossary(1tildeS){$\tilde S$}{signed version of the set $S$}
  be the (decidable) set of \emph{signed} letters from $S$.
  Also, let $\bar{\blank} : \tilde S \to \tilde S$
  be the equivalence that swaps the two copies of $S$.
  This map is an involution called \emph{complementation}.
\end{definition}

If $a:S$, we'll also write $a:\tilde S$ for the left inclusion,
and we'll write $A\defeq\bar a:\tilde S$ for the right inclusion,
so that $\bar a \jdeq A$ and $\bar A \jdeq a$, \ie $a$ and $A$ are complementary.

\begin{definition}
  For any set $T$, let $T^*$ be the set of finite lists of elements of $T$.
  This is the inductive type with constructors $\varepsilon : T^*$
  (\emph{the empty list})\glossary(1Tstar){$T^*$}{list of elements of $T$}%
  \index{list}
  and \emph{concatenation} of type $T \to T^* \to T^*$,
  taking an element $t:T$ and a list $\ell$ to the extended list $t\ell$
  consisting of $t$ followed by the elements of $\ell$.
\end{definition}
Instead of ``lists'' we often speak about ``words'' formed from ``letters''
taking from the set $T$, which is thus a kind of ``alphabet''.

If we take $T \defeq \tilde S$ we get the set of words in the signed letters from $S$. If we have $a,b:S$, we find among the elements of $\tilde S^*$ the following:
\[
  \varepsilon,a,b,A,B,aa,ab,aA,aB,ba,bb,bA,bB,Aa,Ab,AA,AB,\ldots
\]
When we interpret these as symmetries in $\BFG_S$, \ie as elements in $\UFG_S$,
the words $aA$ and $Bb$, etc., become trivial.
\begin{definition}
  A word $w : \tilde S^*$ is called \emph{reduced}
  if it doesn't contain any consecutive pairs of complementary letters.
  The map $\rho_S : \tilde S^* \to \tilde S^*$ maps a word to its \emph{reduction},
  which is obtained by repeatedly deleting consecutive pairs of complementary
  letters until none remain.
\end{definition}

\begin{xca}
  Complete the definition of $\rho_S$ by nested induction on words.\footnote{%
    Hint: This is precisely the point where we need $S$ to have decidable equality.}
\end{xca}

\begin{definition}
  We define $\mathcal R_S$ to be the image of $\rho_S$ in $\tilde S^*$,
  whose elements are the \emph{reduced words}.
  We define $\mathcal D_S$ to be the fiber of $\rho_S$ at the empty word,
  $\rho_S^{-1}(\varepsilon)$,
  whose elements are called \emph{Dyck words}.\footnote{%
    Considered as a set of words,
    $\mathcal D_S$ is called the \emph{2-sided Dyck language}.
    Perhaps the \emph{1-sided Dyck language} is more familiar
    in language theory: Here, $S$ is considered as a set of
    `opening parentheses', while the complementary elements
    are `closing parentheses'.
    For example, the 1-sided Dyck language for $\tilde S=\set{\textup{(},\textup{)}}$
    consists of all \emph{balanced} words of opening and closing parentheses,
    \eg (), (()), ()(), etc., while our $\mathcal D_S$ in this case
    also has words like )( and ))(()(.}
\end{definition}

\begin{remark}
  Like any map, $\rho_S$ induces an equivalence relation $\sim$
  on the set $\tilde S^*$
  where two words $u,v$ are related
  if and only if they map to the same reduced word,
  in other words,
  $u \sim v$ if and only if $\rho_S(u) = \rho_S(v)$.
  Thus, $\rho_S$ induces an equivalence $\tilde S^*/{\sim} \equivto \mathcal R_S$.
\end{remark}

We are now ready to prove that set $\mathcal R_S$ of reduced words
is equivalent to $\UFG_S$.
We'll do this be defining an interpretation function from words
to elements of the free group.

\begin{definition}
  We define $\sem\blank : \tilde S^* \to \UFG_S$
  by induction on words by setting
  \begin{align*}
    \sem\varepsilon             &\defeq \refl\base & & \\
    \sem{aw}                    &\defeq \Sloop_a \cdot \sem w,
                                &\text{for $a:S$,}& \\
    \sem{\bar aw}\jdeq \sem{Aw} &\defeq \Sloop_a^{-1} \cdot \sem w,
                                &\text{for $a:S$.}&\qedhere
  \end{align*}
\end{definition}

\begin{theorem}\label{thm:free-group-elements}
  Fix a decidable set $S$.
  The interpretation map $\sem\blank$ restricts to
  an equivalence, denoted the same way,
  $\sem\blank : \mathcal R_S \to \UFG_S$.
\end{theorem}

\begin{proof}
  We extend $\mathcal R_S$ to an $\FG_S$-set, $\mathcal R_S : \BFG_S \to \Set$,
  where we define $\mathcal R_S(x)$ by induction on $x : \BFG_S$, with
  \[
    \mathcal R_S(\base) \defeq \mathcal R_S,\quad\text{and}\quad
    \mathcal R_S(\Sloop_a) \defis \casoverline{\zs_a},\quad\text{for $a:S$.}
  \]
  Here $\zs_a : \mathcal R_S \equivto \mathcal R_S$ is the equivalence
  sending a word $w$ to $\rho_S(aw)$, whose inverse sends
  $w$ to $\rho_S(Aw)$. These operations are indeed mutual inverses,
  since $aAw \sim w \sim Aaw$.\footnote{%
    The set $\mathcal R_S$ is very much like $\zet$, but instead of having
    only one successor equivalence $\zs$, it has one for each element of $S$.}

  Our goal now is to extend the definition of $\sem\blank$ to
  $\sem\blank_x : \mathcal R_S(x) \to \pathsp\base$,
  where $\pathsp\base(x) \jdeq (\base \eqto x)$, for $x : \BFG_S$,
  so that this is an inverse to the map given by transport of $\varepsilon$,
  $\tau_x : (\base \eqto x) \to \mathcal R_S(x)$,
  with $\tau_x(p) \defeq \trp[\mathcal R_S]{p}(\varepsilon)$.
  Thinking back to~\cref{def:gRtoP},
  we define $\sem\blank_x$ by induction on $x$ with
  $\sem\blank_\base \defeq \sem\blank$ and using
  $\sem{aw} \jdeq \Sloop_a \cdot \sem w$.\footnote{%
    In a picture, the case for $\Sloop_a$ should prove that it does not matter what
    path you take around the square
    \[
      \begin{tikzcd}[row sep=large,column sep=huge,ampersand replacement=\&]
        \mathcal R_S\ar[r,"{\sem\blank}"]\ar[d,eqr,"\zs_a"] \&
        (\base\eqto\base)\ar[d,eqr,"\Sloop_a\cdot\blank"] \\
        \mathcal R_S\ar[r,"{\sem\blank}"] \& (\base\eqto\base).
      \end{tikzcd}
    \]}

  We get an identification $\sem{\blank}_x \circ \tau_x \eqto \id$ by path induction,
  since $\sem{\varepsilon} \jdeq \refl\base$.

  To prove the proposition $\tau_x(\sem w_x) = w$ for all $x:\BFG_S$
  and $w : \mathcal R_S(x)$,
  it suffices to consider the case $x \jdeq \base$, since $\BFG_S$ is connected.
  We prove that $\tau_\base(\sem w) \sim w$ holds for \emph{all} words
  $w:\tilde S^*$ by induction on $w$,
  because then it follows that $\tau_\base(\sem w) = w$ for \emph{reduced} words $w$.
  The case $w \jdeq \varepsilon$ is trivial.
  In the step case for adding $a:S$, we calculate,
  \[
    \tau_\base(\sem{aw}) \jdeq \trp[\mathcal R_S]{\Sloop_a \cdot \sem w}(\varepsilon)
    = \trp[\mathcal R_S]{\Sloop_a}(\tau_\base(\sem w))
    = \zs_a(w) = \rho_S(aw) \sim aw,
  \]
  as desired, the complementary case being similar.
\end{proof}

\begin{xca}
  Construct an equivalence $\mathcal R_{\bn 1} \equivto \zet$
  sending $\varepsilon$ to $0$ such that $\zs_*$ corresponds to $\zs$,
  where $*:\bn 1$ is the unique element.
  This gives us two more options to add to the list in~\cref{ft:many-integers}
  on~\cpageref{ft:many-integers}: $\tilde{\bn 1}^*/{\sim}$ and $\mathcal R_{\bn 1}$!
\end{xca}

\begin{xca}
  Construct an equivalence $\FG_{\bn n \amalg \true} \equivto \FG_{\bn n}
  \vee \ZZ$ for each $n : \NN$ using the universal properties.
  As a result, give identifications
  \[
    \FG_{\bn n} \eqto \bigl((\ZZ \vee \ZZ) \vee \cdots\bigr) \vee \ZZ,
  \]
  for $n:\NN$, where there are $n$ copies of $\ZZ$ on the right-hand side.
\end{xca}

%%% Local Variables:
%%% mode: LaTeX
%%% fill-column: 144
%%% latex-block-names: ("lemma" "theorem" "remark" "definition" "corollary" "fact" "properties" "conjecture" "proof" "question" "proposition" "exercise")
%%% TeX-master: "book"
%%% TeX-command-extra-options: "-fmt=macros"
%%% compile-command: "make book.pdf"
%%% End:

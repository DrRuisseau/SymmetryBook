\chapter{Finitely generated groups}
\label{ch:fggroups}

\section{Brief overview of the chapter}
\label{sec:fggroups-overview}

TODO:
\begin{itemize}
\item Make a separate chapter on combinatorics? Actions and Burnside and counting colorings?
\item Cayley actions: $G$ acts on $\Gamma(G,S)$: Action on vertices is the left action of $G$ on itself: $t \mapsto (t=_{\BG} \pt)$, on vertices, for $s : S$, have edge $t = \pt$ to $t = \pt$
\item Recall universal property of free groups: If we have a map $\varphi : S \to H$, then we get a homomorphism $\bar\varphi : F(S) \to H$, represented by $BF(S) \to_\pt \BH$ defined by induction, sending $\pt$ to $\pt$ and $s$ to $\varphi(s)$.
\item define different types of graphs ($S$-digraphs, $\tilde S$-graphs,
  (partial) functional graphs, graph homomorphisms, quotients of graphs)
\item define (left/right) Cayley graphs of f.g.~groups
  -- $\Aut(\Gamma_G) = G$ (include $\alpha : F(S) \to G$ in notation?)
  -- Cayley graphs are vertex transitive
\item Cayley graphs and products, semi-direct products, homomorphisms
\item Some isomorphisms involving semi-direct products
  -- Exceptional automorphism of $\Sigma_6$:
  -- Exotic map $\Sigma_5 \to \Sigma_6$.
  (Conjugation action of $\Sigma_5$ on $6$ $5$-Sylow subgroups.)
  A set bundle $X : \BSigma_6 \to \BSigma_6$.
\item \url{https://math.ucr.edu/home/baez/six.html}
  Relating $\Sigma_6$ to the icosahedron.
  The icosahedron has $6$ axes. Two axes determines a golden rectangle (also known as a \emph{duad},\footnote{%
    These names come from Sylvester.}
  so there are $15$ such. A symmetry of the icosahedron can be described
  by knowing where a fixed rectangle goes, and a symmetry of that rectangle.
  Picking three rectangles not sharing a diagonal gives a \emph{syntheme}:
  three golden rectangles whose vertices make up the icosahedron.
  Some synthemes (known as \emph{true crosses}
  have the rectangles orthogonal to each other, as in
  \cref{fig:true-cross}.
  Fact: The symmetries of the icosahedron form the alternating symmetries of the $5$ true crosses.
  Of course, we get an action on the $6$ axes, thus a homomorphism $\constant{A}_5 \to \Sigma_6$.
  Every golden rectangle lies in one true cross and two skew crosses.
  The combinatorics of duads, synthemes, and synthematic totals are illustrated
  in the Cremona-Richardson configuration and the resulting Tutte-Coxeter graph.
  The automorphism group of the latter is in fact $\Aut(\Sigma_6)$.
  If we color the vertices according to duad/syntheme, we get $\Sigma_6$ itself.
\begin{marginfigure}
  \tdplotsetmaincoords{45}{135}
  \begin{tikzpicture}[tdplot_main_coords,scale=1.1]
    \begin{scope}[opacity=0.6]
      \draw (-1.00000, -1.61803, 0.00000) -- (-0.00000, -1.00000, -1.61803);
      \draw (-1.61803, 0.00000, -1.00000) -- (-0.00000, -1.00000, -1.61803);
      \draw (-1.00000, -1.61803, 0.00000) -- (-1.61803, -0.00000, -1.00000);
      \draw (1.00000, -1.61803, 0.00000) -- (-0.00000, -1.00000, -1.61803);
      \draw (1.00000, -1.61803, 0.00000) -- (-1.00000, -1.61803, 0.00000);
      \draw (-1.61803, 0.00000, 1.00000) -- (-1.00000, -1.61803, -0.00000);
      \draw (-1.61803, 0.00000, -1.00000) -- (0.00000, 1.00000, -1.61803);
      \draw (-1.61803, 0.00000, 1.00000) -- (-1.61803, 0.00000, -1.00000);
      \draw (0.00000, 1.00000, -1.61803) -- (0.00000, -1.00000, -1.61803);
      \fill[gray] (0.00000, 0.00000, 0.00000) -- (0.00000, -1.61803, 0.00000) -- (-1.00000, -1.61803, 0.00000) -- (-1.00000, 0.00000, 0.00000) -- cycle;
      \fill[casblue] (0.00000, 0.00000, 0.00000) -- (0.00000, 0.00000, -1.61803) -- (0.00000, -1.00000, -1.61803) -- (0.00000, -1.00000, 0.00000) -- cycle;
      \fill[casred] (0.00000, 0.00000, 0.00000) -- (-1.61803, 0.00000, 0.00000) -- (-1.61803, 0.00000, -1.00000) -- (0.00000, 0.00000, -1.00000) -- cycle;
      \draw (0.00000, -1.00000, 1.61803) -- (-1.00000, -1.61803, 0.00000);
      \draw (1.61803, 0.00000, -1.00000) -- (0.00000, -1.00000, -1.61803);
      \draw (-1.00000, 1.61803, 0.00000) -- (-1.61803, 0.00000, -1.00000);
      \fill[casred] (0.00000, 0.00000, 0.00000) -- (-1.61803, 0.00000, 0.00000) -- (-1.61803, 0.00000, 1.00000) -- (0.00000, 0.00000, 1.00000) -- cycle;
      \fill[casblue] (0.00000, 0.00000, 0.00000) -- (0.00000, 0.00000, -1.61803) -- (0.00000, 1.00000, -1.61803) -- (0.00000, 1.00000, 0.00000) -- cycle;
      \fill[gray] (0.00000, 0.00000, 0.00000) -- (0.00000, -1.61803, 0.00000) -- (1.00000, -1.61803, 0.00000) -- (1.00000, 0.00000, 0.00000) -- cycle;
      \draw (0.00000, -1.00000, 1.61803) -- (1.00000, -1.61803, 0.00000);
      \draw (0.00000, -1.00000, 1.61803) -- (-1.61803, 0.00000, 1.00000);
      \draw (1.61803, 0.00000, -1.00000) -- (1.00000, -1.61803, 0.00000);
      \draw (-1.61803, 0.00000, 1.00000) -- (-1.00000, 1.61803, 0.00000);
      \draw (0.00000, 1.00000, -1.61803) -- (1.61803, 0.00000, -1.00000);
      \draw (-1.00000, 1.61803, 0.00000) -- (0.00000, 1.00000, -1.61803);
      \fill[casred] (0.00000, 0.00000, 0.00000) -- (1.61803, 0.00000, 0.00000) -- (1.61803, 0.00000, -1.00000) -- (0.00000, 0.00000, -1.00000) -- cycle;
      \fill[gray] (0.00000, 0.00000, 0.00000) -- (0.00000, 1.61803, 0.00000) -- (-1.00000, 1.61803, 0.00000) -- (-1.00000, 0.00000, 0.00000) -- cycle;
      \fill[casblue] (0.00000, 0.00000, 0.00000) -- (0.00000, 0.00000, 1.61803) -- (0.00000, -1.00000, 1.61803) -- (0.00000, -1.00000, 0.00000) -- cycle;
      \draw (0.00000, 1.00000, 1.61803) -- (-1.61803, 0.00000, 1.00000);
      \draw (1.61803, 0.00000, 1.00000) -- (1.00000, -1.61803, 0.00000);
      \draw (1.00000, 1.61803, 0.00000) -- (0.00000, 1.00000, -1.61803);
      \fill[gray] (0.00000, 0.00000, 0.00000) -- (0.00000, 1.61803, 0.00000) -- (1.00000, 1.61803, 0.00000) -- (1.00000, 0.00000, 0.00000) -- cycle;
      \fill[casblue] (0.00000, 0.00000, 0.00000) -- (0.00000, 0.00000, 1.61803) -- (0.00000, 1.00000, 1.61803) -- (0.00000, 1.00000, 0.00000) -- cycle;
      \fill[casred] (0.00000, 0.00000, 0.00000) -- (1.61803, 0.00000, 0.00000) -- (1.61803, 0.00000, 1.00000) -- (0.00000, 0.00000, 1.00000) -- cycle;
      \draw (0.00000, -1.00000, 1.61803) -- (1.61803, 0.00000, 1.00000);
      \draw (0.00000, 1.00000, 1.61803) -- (0.00000, -1.00000, 1.61803);
      \draw (1.61803, 0.00000, 1.00000) -- (1.61803, 0.00000, -1.00000);
      \draw (1.00000, 1.61803, 0.00000) -- (1.61803, 0.00000, -1.00000);
      \draw (0.00000, 1.00000, 1.61803) -- (-1.00000, 1.61803, 0.00000);
      \draw (-1.00000, 1.61803, 0.00000) -- (1.00000, 1.61803, 0.00000);
      \draw (0.00000, 1.00000, 1.61803) -- (1.61803, 0.00000, 1.00000);
      \draw (0.00000, 1.00000, 1.61803) -- (1.00000, 1.61803, 0.00000);
      \draw (1.61803, 0.00000, 1.00000) -- (1.00000, 1.61803, 0.00000);
    \end{scope}
  \end{tikzpicture}
  \caption{Icosahedron with an inscribed true cross}
  \label{fig:true-cross}
\end{marginfigure}
\item define (left/right) presentation complex of group presentation
\item define Stallings folding
\item deduce Nielsen--Schreier and Nielsen basis
\item deduce algorithms for generalized word problem, conjugation, etc.
\item deduce Howson's theorem
\item think about 2-cell replacement for folding; better proofs in HoTT?
\item move decidability results to main flow
\item include undecidability of word problem in general
  -- doesn't depend on presentation (for classes closed under inverse images of monoid homomorphisms)
\item describe $F(S)/H$ in the case where $H$ has infinite index
\item describe normal closure of $R$ in $F(S)$ -- still f.g.? -- get Cayley graph of $F(S)/\langle R\rangle$. -- Todd-Coxeter algorithm?
\item in good cases we can recognize $\mathcal{S}(R)$ as a ``fundamental domain'' in Cayley graph of $\langle S\mid R\rangle$.
\end{itemize}

\begin{remark}
  In this chapter, we use letters from the
  beginning of the alphabet $a,b,c,\dots$
  to denote generators,
  and we use the corresponding capital letters
  $A,B,C$ to denote their inverses,
  so, e.g., $aA=Aa=1$.
  This cleans up the notational clutter significantly.
\end{remark}

Do we fix $S$, a finite set $S=\{a,b,\ldots\}$?
Mostly $F$ will denote the free group on $S$.
And for almost all examples, we take $S = \{a,b\}$.

\section{Free groups}
\label{sec:freegroups}

We have seen in~\cref{ex:Zinitial} that the group of integers $\ZZ$
is the free group on one generator in the sense that the set of homomorphisms
from $\ZZ$ to any group $G$ is equivalent (by evaluation at the loop)
to the underlying set of elements of $G$, $\USymG$.
This set is of course equivalent (by evaluation at the unique element)
to the set of maps $(\bn 1 \to \USymG)$.

Likewise, we have seen in~\cref{cor:ZplusZuniv} that the binary sum $\ZZ \vee \ZZ$
is the free group on two generators, corresponding to the left and right summands.

In general, a free group on a set of generators $S$ is a group $\FG_S$ with
specified elements $\iota_s:\UFG_S$ labeled by $s:S$,
such that evaluation gives an equivalence $\Hom(\FG_S,G) \equivto (S \to \USymG)$
for each group $G$.

We now give a definition of the classifying type of a free group
as a higher inductive type that is very much like that of the circle,
except that instead of having a single generating loop,
it has a loop $\Sloop_s$ for each element $s:S$.
For technical reasons, we restrict our discussion to decidable sets $S$.\footnote{%
  It is an open problem whether our theory proves that type $\BFG_S$
  is a groupoid for all sets $S$.
  We can still define the free group on an arbitrary set,
  but we shall not need this generality.}
\begin{definition}
  \label{def:bfree}
  Fix a decidable set $S$.
  The classifying type of the free group on $S$, $\BFG_S$,%
  \index{free group}%
  \glossary(FS){$\protect\FG_S$}{free group on a decidable set of generators}
  is a type with a point $\base : \BFG_S$ and
  a constructor ${\Sloop_\blank} : S \to \base\eqto\base$.

  Let $A(x)$ be a type for every element $x:\BFG_S$.
  The induction principle for $\BFG_S$ states that,
  in order to define an element of $A(x)$ for every $x:\BFG_S$,
  it suffices to give an element $a$ of $A(\base)$ together
  with an identification $l_s:\pathover{a}{A}{\Sloop_s}{a}$
  for every $s:S$.
  The function $f$ thus defined satisfies $f(\base)\jdeq a$
  and we are provided identifications $\apd{f}(\Sloop_s)\eqto l_s$ for each $s:S$.

  We define the \emph{free group} on $S$ as $\FG_S \defeq \mkgroup(\BFG_S,\base)$.
\end{definition}

A priori, $\FG_S$ is only an \inftygp.
Nevertheless, we get immediately from the induction principle
that evaluation at the elements of $S$
gives an equivalence $\Hom(\FG_S,G) \equivto (S \to \USymG)$ for each
\inftygp $G$.

In order to see that $\FG_S$ is group,
we need to know that $\BFG_S$ is a groupoid.
We can follow that same strategy as in~\cref{lem:univisexp}
and~\cref{lem:wedgeofgpoidisgpoid}
and show this by giving a description of $\FG_S$ as an \emph{abstract} group.
To see what this should be, think about what symmetries of $\base$
we can write using the constructors $\Sloop_s$ for $s:S$.
We can compose these out of $\Sloop_s$ and $\Sloop_s^{-1}$
with various generators $s$.
However, if we at any point have $\Sloop_s\Sloop_s^{-1}$ or $\Sloop_s^{-1}\Sloop_s$,
then these cancel.
This motivates the following definitions.

\begin{definition}
  Fix a decidable set $S$. Let $\tilde S \defeq S + S$%
  \index{signed set}%
  \glossary(1tildeS){$\tilde S$}{signed version of the set $S$}
  be the (decidable) set of \emph{signed} letters from $S$.
  Also, let $\bar{\blank} : \tilde S \to \tilde S$
  be the equivalence that swaps the two copies of $S$.
  This map is an involution called \emph{complementation}.
\end{definition}

If $a:S$, we'll also write $a:\tilde S$ for the left inclusion,
and we'll write $A\defeq\bar a:\tilde S$ for the right inclusion,
so that $\bar a \jdeq A$ and $\bar A \jdeq a$, \ie $a$ and $A$ are complementary.

\begin{definition}
  For any set $T$, let $T^*$ be the set of finite lists of elements of $T$.
  This is the inductive type with constructors $\varepsilon : T^*$
  (\emph{the empty list})\glossary(1Tstar){$T^*$}{list of elements of $T$}%
  \index{list}
  and \emph{concatenation} of type $T \to T^* \to T^*$,
  taking an element $t:T$ and a list $\ell$ to the extended list $t\ell$
  consisting of $t$ followed by the elements of $\ell$.
\end{definition}
Instead of ``lists'' we often speak about ``words'' formed from ``letters''
taking from the set $T$, which is thus a kind of ``alphabet''.

If we take $T \defeq \tilde S$ we get the set of words in the signed letters from $S$. If we have $a,b:S$, we find among the elements of $\tilde S^*$ the following:
\[
  \varepsilon,a,b,A,B,aa,ab,aA,aB,ba,bb,bA,bB,Aa,Ab,AA,AB,\ldots
\]
When we interpret these as symmetries in $\BFG_S$, \ie as elements in $\UFG_S$,
the words $aA$ and $Bb$, etc., become trivial.
\begin{definition}
  A word $w : \tilde S^*$ is called \emph{reduced}
  if it doesn't contain any consecutive pairs of complementary letters.
  The map $\rho_S : \tilde S^* \to \tilde S^*$ maps a word to its \emph{reduction},
  which is obtained by repeatedly deleting consecutive pairs of complementary
  letters until none remain.
\end{definition}

\begin{xca}
  Complete the definition of $\rho_S$ by nested induction on words.\footnote{%
    Hint: This is precisely the point where we need $S$ to have decidable equality.}
\end{xca}

\begin{definition}
  We define $\mathcal R_S$ to be the image of $\rho_S$ in $\tilde S^*$,
  whose elements are the \emph{reduced words}.
  We define $\mathcal D_S$ to be the fiber of $\rho_S$ at the empty word,
  $\rho_S^{-1}(\varepsilon)$,
  whose elements are called \emph{Dyck words}.\footnote{%
    Considered as a set of words,
    $\mathcal D_S$ is called the \emph{2-sided Dyck language}.
    Perhaps the \emph{1-sided Dyck language} is more familiar
    in language theory: Here, $S$ is considered as a set of
    `opening parentheses', while the complementary elements
    are `closing parentheses'.
    For example, the 1-sided Dyck language for $\tilde S=\set{\textup{(},\textup{)}}$
    consists of all \emph{balanced} words of opening and closing parentheses,
    \eg (), (()), ()(), etc., while our $\mathcal D_S$ in this case
    also has words like )( and ))(()(.}
\end{definition}

\begin{remark}
  Like any map, $\rho_S$ induces an equivalence relation $\sim$
  on the set $\tilde S^*$
  where two words $u,v$ are related
  if and only if they map to the same reduced word,
  in other words,
  $u \sim v$ if and only if $\rho_S(u) = \rho_S(v)$.
  Thus, $\rho_S$ induces an equivalence $\tilde S^*/{\sim} \equivto \mathcal R_S$.
\end{remark}

We are now ready to prove that set $\mathcal R_S$ of reduced words
is equivalent to $\UFG_S$.
We'll do this be defining an interpretation function from words
to elements of the free group.

\begin{definition}
  We define $\sem\blank : \tilde S^* \to \UFG_S$
  by induction on words by setting
  \begin{align*}
    \sem\varepsilon             &\defeq \refl\base & & \\
    \sem{aw}                    &\defeq \Sloop_a \cdot \sem w,
                                &\text{for $a:S$,}& \\
    \sem{\bar aw}\jdeq \sem{Aw} &\defeq \Sloop_a^{-1} \cdot \sem w,
                                &\text{for $a:S$.}&\qedhere
  \end{align*}
\end{definition}

\begin{theorem}\label{thm:free-group-elements}
  Fix a decidable set $S$.
  The interpretation map $\sem\blank$ restricts to
  an equivalence, denoted the same way,
  $\sem\blank : \mathcal R_S \to \UFG_S$.
\end{theorem}

\begin{proof}
  We extend $\mathcal R_S$ to an $\FG_S$-set, $\mathcal R_S : \BFG_S \to \Set$,
  where we define $\mathcal R_S(x)$ by induction on $x : \BFG_S$, with
  \[
    \mathcal R_S(\base) \defeq \mathcal R_S,\quad\text{and}\quad
    \mathcal R_S(\Sloop_a) \defis \casoverline{\zs_a},\quad\text{for $a:S$.}
  \]
  Here $\zs_a : \mathcal R_S \equivto \mathcal R_S$ is the equivalence
  sending a word $w$ to $\rho_S(aw)$, whose inverse sends
  $w$ to $\rho_S(Aw)$. These operations are indeed mutual inverses,
  since $aAw \sim w \sim Aaw$.\footnote{%
    The set $\mathcal R_S$ is very much like $\zet$, but instead of having
    only one successor equivalence $\zs$, it has one for each element of $S$.}

  Our goal now is to extend the definition of $\sem\blank$ to
  $\sem\blank_x : \mathcal R_S(x) \to \pathsp\base$,
  where $\pathsp\base(x) \jdeq (\base \eqto x)$, for $x : \BFG_S$,
  so that this is an inverse to the map given by transport of $\varepsilon$,
  $\tau_x : (\base \eqto x) \to \mathcal R_S(x)$,
  with $\tau_x(p) \defeq \trp[\mathcal R_S]{p}(\varepsilon)$.
  Thinking back to~\cref{def:gRtoP},
  we define $\sem\blank_x$ by induction on $x$ with
  $\sem\blank_\base \defeq \sem\blank$ and using
  $\sem{aw} \jdeq \Sloop_a \cdot \sem w$.\footnote{%
    In a picture, the case for $\Sloop_a$ should prove that it does not matter what
    path you take around the square
    \[
      \begin{tikzcd}[row sep=large,column sep=huge,ampersand replacement=\&]
        \mathcal R_S\ar[r,"{\sem\blank}"]\ar[d,eqr,"\zs_a"] \&
        (\base\eqto\base)\ar[d,eqr,"\Sloop_a\cdot\blank"] \\
        \mathcal R_S\ar[r,"{\sem\blank}"] \& (\base\eqto\base).
      \end{tikzcd}
    \]}

  We get an identification $\sem{\blank}_x \circ \tau_x \eqto \id$ by path induction,
  since $\sem{\varepsilon} \jdeq \refl\base$.

  To prove the proposition $\tau_x(\sem w_x) = w$ for all $x:\BFG_S$
  and $w : \mathcal R_S(x)$,
  it suffices to consider the case $x \jdeq \base$, since $\BFG_S$ is connected.
  We prove that $\tau_\base(\sem w) \sim w$ holds for \emph{all} words
  $w:\tilde S^*$ by induction on $w$,
  because then it follows that $\tau_\base(\sem w) = w$ for \emph{reduced} words $w$.
  The case $w \jdeq \varepsilon$ is trivial.
  In the step case for adding $a:S$, we calculate,
  \[
    \tau_\base(\sem{aw}) \jdeq \trp[\mathcal R_S]{\Sloop_a \cdot \sem w}(\varepsilon)
    = \trp[\mathcal R_S]{\Sloop_a}(\tau_\base(\sem w))
    = \zs_a(w) = \rho_S(aw) \sim aw,
  \]
  as desired, the complementary case being similar.
\end{proof}

\begin{xca}
  Construct an equivalence $\mathcal R_{\bn 1} \equivto \zet$
  sending $\varepsilon$ to $0$ such that $\zs_*$ corresponds to $\zs$,
  where $*:\bn 1$ is the unique element.
  This gives us two more options to add to the list in~\cref{ft:many-integers}
  on~\cpageref{ft:many-integers}: $\tilde{\bn 1}^*/{\sim}$ and $\mathcal R_{\bn 1}$!
\end{xca}

\begin{xca}
  Construct an equivalence $\FG_{\bn n \amalg \true} \equivto \FG_{\bn n}
  \vee \ZZ$ for each $n : \NN$ using the universal properties.
  As a result, give identifications
  \[
    \FG_{\bn n} \eqto \bigl((\ZZ \vee \ZZ) \vee \cdots\bigr) \vee \ZZ,
  \]
  for $n:\NN$, where there are $n$ copies of $\ZZ$ on the right-hand side.
\end{xca}

\section{Graphs and Cayley graphs}
\label{sec:cayley-graphs}

We have seen in the previous chapter how cyclic groups
(those generated by a single generator)
have neatly described types of torsors.
Indeed, $\BCG_n \jdeq \Cyc_n$, where $\Cyc_n$ is the type of $n$-cycles,
and the classifying type of the integers, $\B\ZZ\jdeq\Sc$, \ie the circle,
is equivalent to the type of infinite cycles, $\Cyc_0$.
In \cref{cha:circle}, we defined the types of (finite or infinite)
cycles as certain components of $\sum_{X:\UU}(X\eqto X)$,
but we can equivalently consider components of $\sum_{X:\UU}(X\to X)$,
since the former is a subtype of the latter.
By thinking of functions in terms of their graphs,
we might as well look at components of $\sum_{X:\UU}(X \to X \to \UU)$.

In this section we shall generalize this story
to groups $G$ generated by a
(finite or just decidable)
set of generators $S$.

First recall from Cayley's \cref{lem:allgpsarepermutationgps}
that any group $G$ can be realized as a subgroup of the permutation group
on the underlying set of elements of $G$, $\USymG$.
In this description, a $G$-shape is a set $X$ equipped a torsorial $G$-action,
which in turn can be expressed as the structure of a map $\alpha:\USymG \to X \to X$
satisfying certain properties.

It may happen that already $\alpha$ restricted to a subset $S$ of $\USymG$
suffices to specify the action.
In that case we say that $S$ generates $G$, though we'll take the following
as the official definition.
\begin{definition}\label{def:gens-gp}
  Let $G$ be a group and $S$ be a subset of $\USymG$, given by an inclusion
  $\iota : S \to \USymG$. We say that \emph{$S$ generates $G$} if the induced
  homomorphism
  \[
    \FG_S \to G
  \]
  is an epimorphism.
\end{definition}
\begin{lemma}\label{lem:gens-gp-iff}
  Let $G$ be a group and $\iota : S \to \USymG$ the inclusion of a subset of the elements of $G$.
  Then $S$ generates $G$ if and only if the map
  \[
    \rho_S : \BG \to \sum_{X:\UU}(S \to X \to X) ,
    \quad
    \rho_S(t) \defeq \bigl(t \eqto \sh_G, s \mapsto \iota(s) \cdot \blank\bigr)
  \]
  is an embedding.\footnote{We use $t \eqto \sh_G$ rather than the equivalent
    $\sh_G \eqto t$ in order to conform to the representation from Cayley's theorem.}
\end{lemma}
In this case, then, $G$ can be identified with the automorphism group of $\rho_S(\sh_G)$
in the type $\sum_{X:\UU}(S \to X \to X)$, or even in the larger type (of which it's a subtype), $\sum_{X:\UU}(S \to X \to X \to \UU)$.


\tikzset{vertex/.style={circle,fill=black,inner sep=0pt,minimum size=4pt}}
\tikzset{gena/.style={draw=casblue,-stealth}}
\tikzset{genb/.style={draw=casred,-stealth}}

\begin{figure}
  \begin{sidecaption}%
    {Cayley graph for {$\protect\SG_3$} with respect to $S = \{(1\;2),(2\;3)\}$.}[fig:cayley-s3]
  \centering
  \begin{tikzpicture}
    \pgfmathsetmacro{\len}{2}
    \node[vertex,label=30:$(1\;3)$]   (n13)  at (30:\len)  {};
    \node[vertex,label=90:$(1\;3\;2)$]  (n132) at (90:\len)  {};
    \node[vertex,label=150:$(1\;2)$]  (n12)  at (150:\len) {};
    \node[vertex,label=210:$e$]     (ne)   at (210:\len) {};
    \node[vertex,label=270:$(2\;3)$]  (n23)  at (270:\len) {};
    \node[vertex,label=330:$(1\;2\;3)$] (n123) at (330:\len) {};
    \begin{scope}[every to/.style={bend left=22}]
      % generator a is (12)
      \draw[gena] (ne)   to (n12);
      \draw[gena] (n12)  to (ne);
      \draw[gena] (n13)  to (n132);
      \draw[gena] (n132) to (n13);
      \draw[gena] (n123) to (n23);
      \draw[gena] (n23)  to (n123);
      % generator b is (23)
      \draw[genb] (ne)   to (n23);
      \draw[genb] (n23)  to (ne);
      \draw[genb] (n13)  to (n123);
      \draw[genb] (n123) to (n13);
      \draw[genb] (n12)  to (n132);
      \draw[genb] (n132) to (n12);
    \end{scope}
  \end{tikzpicture}
  \end{sidecaption}
\end{figure}

Also note that $S$ generates $G$ if and only if the map on elements
$\UFG_S \to \USymG$ is surjective, meaning every element of $g$ can be expressed
as a product of the letters in a (reduced) word from $\mathcal R_S$, interpreted
according to the inclusion of $S$ into $\USymG$.
This is the case for example for $S$ consisting of the transpositions
$(1\;2)$, $(2\;3)$ in $\SG_3$, as illustrated in~\cref{fig:cayley-s3},
where the \textcolor{casblue}{blue} color represents $(1\;2)$
and the \textcolor{casred}{red} color represents $(2\;3)$.

Before we give the proof of~\cref{lem:gens-gp-iff}, let us study these types more closely.
\begin{definition}
  An $S$-labeled graph is an element $(V,E)$ of the type
  $\sum_{V:\UU}(S \to V \to V \to \UU)$.%
  \index{graph!labeled}
  The first component $V$ is called the type of \emph{vertices} of the graph,
  and the type $E(s,x,y)$ is called the type of $s$-colored \emph{edges}
  from $x$ (the source) to $y$ (the target).
\end{definition}
If for every vertex $x:V$ and every color $s:S$ there is unique $s$-colored edge out of $x$, \ie the type $\sum_{y:V}E(s,x,y)$ is contractible, then we say that the graph
is \emph{functional}. This means that it lives in the subtype $\sum_{V:\UU}(S \to V \to V)$,
as is the case for the graph $\rho_S(\sh_G)$ for a group $G$.
This graph is called the \emph{Cayley graph} of $G$ with respect to the set $S$.%
\index{graph!Cayley graph}

If $S$ is contractible (so there's only one color), then we just say \emph{graph}.
Of course, every $S$-labeled graph $(V,E)$ gives rise to such an unlabeled label
by summing over the colors, \ie the type of edges from $x$ to $y$ in this graph
is $\sum_{s:S}E(s,x,y)$.

Another way to represent a graph is to sum over all the sources and targets (and colors),
via~\cref{lem:typefamiliesandfibrations},
\ie as a tuple $(V,E,s,t,c)$, where $V:\UU$ is the type of vertices,
$E$ is the (total) type of edges,
$s,t : E \to V$ give the source and target of an edge,
while $c: E \to S$ gives the color (if we're talking about $S$-colored graphs).
In this description, to get the unlabeled graph we simply drop the last component.

Every graph $(V,E)$ (and thus every labeled graph) gives rise to a type
by ``gluing the edges to the vertices'' defined as follows.
\begin{definition}
  Fix an unlabeled graph $(V,E)$. The \emph{graph quotient}\footnote{%
    If the graph is represented by source and target maps
    $s,t: E \rightrightarrows V$, then the graph quotient is
  also called the \emph{coequalizer} of $s$ and $t$.} $V/E$ is
  the higher inductive type with constructors:
  \begin{enumerate}
  \item For every vertex $x : V$ a point $[x] : V/E$.
  \item For every edge $e : E(x,y)$ an identification $\qedge_e : [x] \eqto [y]$.
  \end{enumerate}

  Let $A(z)$ be a type for every element $z:V/E$. The induction principle
  for $V/E$ states that, in order to define an element of $A(z)$ for every $z:V/E$,
  it suffices to give elements $a_x : A([x])$ for every vertex $x:V$
  together with
  identifications $q_e : \pathover{a_x}{A}{\qedge_e}{a_y}$
  for every $e:E(x,y)$.
  The function $f$ thus defined satisfies $f([x])\jdeq a_x$ for $x:V$
  and we are provided identifications $\apd{f}(\qedge_e)\eqto q_e$ for each $e:E(x,y)$.
\end{definition}
\begin{remark}
  Note the similarity with the classifying type of a free group,
  \cf~\cref{def:bfree}. Indeed, if we form the (unlabeled!)
  graph $(\bn 1,S)$
  on one vertex with $S$ edges, then $\bn 1/S$ is essentially the same as $\BFG_S$.
\end{remark}
\begin{xca}
  An equivalence relation $R : A \to A \to \Prop$ on a set $A$
  can be regarded as a graph $(A,R)$.
  Construct an equivalence between the graph quotient $A/R$
  and the set quotient $A/R$ from~\cref{def:quotient-set} in this case.
  (Thus, the apparent notational clash is not a source of real ambiguity.)
\end{xca}
While we're building up to the proof of~\cref{lem:gens-gp-iff} we need
a description of a sum type over a graph quotient.
By the above remark, this applies also to sum types over $\BFG_S$.
\begin{construction}\label{def:graph-quotient-flattening}
  Given a graph $(V,E)$ and a family of types $X : V/E \to \UU$.
  Define $V' \defeq \sum_{v:V}X([v])$ and
  $E'((v,x),(w,y)) \defeq \sum_{e:E(v,w)}\pathover{x}{X}{\qedge_e}{y}$.
  Then we have an equivalence\footnote{%
    This is often called the \emph{flattening construction} (or flattening lemma),
    as it ``flattens'' a sum over a graph quotient into a single graph quotient.}
  \[
    V'/E' \equivto \sum_{z:V/E}X(z)
  \]
\end{construction}
\begin{implementation}{def:graph-quotient-flattening}
  We define functions $\varphi : V'/E' \to \sum_{z:V/E}X(z)$
  and $\tilde\psi : \prod_{z:V/E}\bigl(X(z) \to V'/E'\bigr)$
  using the induction principles:
  \begin{alignat*}2
    \varphi([(v,x)]) &\defeq ([v],e) &\quad
    \tilde\psi([v]) &\defeq (x \mapsto [(v,x)]) \\
    \ap\varphi(\qedge_{(e,q)}) &\defis \pathpair{\qedge_e}{q} &\quad
    \apd{\tilde\psi}(\qedge_e) &\defis h,
  \end{alignat*}
  where we need to construct
  $h : \pathover{(x \mapsto [(v,x)])}{z \mapsto X(z) \to V'/E'}{\qedge_e}
  {(y \mapsto [(w,y)])}$ for all $e : E(v,w)$.
  By~\cref{lem:trp-in-function-type}, it suffices to give an identification
  $[(v,x)] \eqto [(w,\trp[X]{\qedge_e}(x))]$ for all $x : X([v])$.
  We get this from the identification constructor $\qedge$ for $V'/E'$
  by applying~\cref{def:pathover-trp} to the reflexivity path at
  $\trp[X]{\qedge_e}(x)$.
\end{implementation}
\begin{xca}
  Complete the implementation by giving identifications $\psi\circ\phi\eqto\id$
  and $\phi\circ\psi\eqto\id$, where $\psi : \bigl(\sum_{z:V/E}X(z)\bigr) \to V'/E'$
  is defined by $\psi((z,x)) \defeq \tilde\psi(z)(x)$.
\end{xca}

Later on we'll need also need the following results about graph quotients.
\begin{xca}\label{xca:graph-quotient-in-steps}
  Suppose the edges $E$ of a graph $(V,E)$ are expressed as a binary sum $E_0 \amalg E_1$.
  (Here, it doesn't matter whether $E$ is expressed as a type family $E : V \to V \to \UU$,
  in which case we have a family of equivalences $E(v,w) \equivto E_0(v,w) \amalg E_1(v,w)$,
  or $E$ is the total type of edges.)

  Then we can obtain the graph quotient $V/E$ by first gluing in the edges from $E_0$,
  and then gluing in the edges from $E_1$ to the resulting type $V/E_0$.
  Using the description of graphs with a total type of edges $E \equivto E_0 \amalg E_1$,
  we have corresponding source and target maps expressed as compositions:
  \[
    E_1 \hookrightarrow E_0\amalg E_1 \equivto E \rightrightarrows V \to V/E_0.
  \]
  Construct an equivalence $V/E \equivto V/(E_0 \amalg E_1) \equivto (V/E_0)/E1$.
\end{xca}

\begin{xca}\label{xca:graph-quotient-whisker}
  Suppose we have any type $X$ with an element $x:X$.
  We can form a graph $(X\amalg\bn 1,\bn 1)$ with vertex type $X\amalg\bn 1$
  and a single edge from $\inl x$ to $\inr 0$.
  Construct an equivalence $X \equivto (X\amalg\bn 1)/\bn 1$.\footnote{%
    This equivalence can be visualized as follows, where $X$ ``grows a whisker''
    along the single edge.\\
    \begin{tikzpicture}
      \draw (0,0) ellipse (1 and 1.2);
      \node (X) at (0,1.5) {$X$};
      \node (t) at (1.7,0.6) {$\bn 1$};
      \node[dot,label=below:$x$] (x) at (.6,0.3) {};
      \node[dot] (pt) at (1.5,.4) {};
      \draw[->,bend left] (x) -- (pt) {};
    \end{tikzpicture}}
\end{xca}
\section{Examples}
\label{sec:fg-examples}



\begin{proof}[Proof of~\cref{lem:gens-gp-iff}]
  TBD (perhaps put in graph quotients first)
\end{proof}


\begin{figure}
  \begin{sidecaption}%
    {Cayley graph for $A_5$ with respect to $S = \{a,b\}$,
      where $a$ is a $1/5$-rotation about a vertex and
      $b$ is a $1/2$-rotation about an edge in an icosahedron.}[fig:cayley-a5]
  \centering
\tdplotsetmaincoords{45}{135}
\begin{tikzpicture}[tdplot_main_coords,scale=2]
  % cayley icosahedron
  \begin{scope}[fill=casblue,opacity=.2]
\fill (0.00000, 1.00000, 1.61803) -- (0.00000, -1.00000, 1.61803) -- (-1.61803, 0.00000, 1.00000) -- cycle;
\fill (-0.00000, 1.00000, 1.61803) -- (1.61803, 0.00000, 1.00000) -- (-0.00000, -1.00000, 1.61803) -- cycle;
\fill (0.00000, 1.00000, 1.61803) -- (-1.61803, 0.00000, 1.00000) -- (-1.00000, 1.61803, 0.00000) -- cycle;
\fill (-0.00000, 1.00000, 1.61803) -- (1.00000, 1.61803, 0.00000) -- (1.61803, 0.00000, 1.00000) -- cycle;
\fill (-0.00000, -1.00000, 1.61803) -- (1.61803, -0.00000, 1.00000) -- (1.00000, -1.61803, 0.00000) -- cycle;
\fill (0.00000, 1.00000, 1.61803) -- (-1.00000, 1.61803, 0.00000) -- (1.00000, 1.61803, 0.00000) -- cycle;
\fill (-0.00000, -1.00000, 1.61803) -- (-1.00000, -1.61803, 0.00000) -- (-1.61803, 0.00000, 1.00000) -- cycle;
\fill (1.61803, -0.00000, 1.00000) -- (1.00000, 1.61803, -0.00000) -- (1.61803, -0.00000, -1.00000) -- cycle;
\fill (-0.00000, -1.00000, 1.61803) -- (1.00000, -1.61803, 0.00000) -- (-1.00000, -1.61803, 0.00000) -- cycle;
\fill (-1.61803, 0.00000, 1.00000) -- (-1.61803, 0.00000, -1.00000) -- (-1.00000, 1.61803, 0.00000) -- cycle;
\fill (1.00000, 1.61803, 0.00000) -- (-1.00000, 1.61803, -0.00000) -- (0.00000, 1.00000, -1.61803) -- cycle;
\fill (-1.61803, 0.00000, 1.00000) -- (-1.00000, -1.61803, -0.00000) -- (-1.61803, 0.00000, -1.00000) -- cycle;
\fill (1.61803, 0.00000, 1.00000) -- (1.61803, -0.00000, -1.00000) -- (1.00000, -1.61803, 0.00000) -- cycle;
\fill (-1.00000, -1.61803, 0.00000) -- (1.00000, -1.61803, -0.00000) -- (-0.00000, -1.00000, -1.61803) -- cycle;
\fill (-1.00000, 1.61803, 0.00000) -- (-1.61803, 0.00000, -1.00000) -- (0.00000, 1.00000, -1.61803) -- cycle;
\fill (1.00000, 1.61803, 0.00000) -- (0.00000, 1.00000, -1.61803) -- (1.61803, 0.00000, -1.00000) -- cycle;
\fill (1.00000, -1.61803, 0.00000) -- (1.61803, -0.00000, -1.00000) -- (-0.00000, -1.00000, -1.61803) -- cycle;
\fill (-1.61803, 0.00000, -1.00000) -- (-1.00000, -1.61803, -0.00000) -- (-0.00000, -1.00000, -1.61803) -- cycle;
\fill (1.61803, 0.00000, -1.00000) -- (0.00000, 1.00000, -1.61803) -- (0.00000, -1.00000, -1.61803) -- cycle;
\fill (0.00000, 1.00000, -1.61803) -- (-1.61803, 0.00000, -1.00000) -- (0.00000, -1.00000, -1.61803) -- cycle;
\end{scope}
\node[vertex] (n) at (0.00000, 0.50000, 1.61803) {};
\node[vertex] (na) at (0.40451, 0.75000, 1.46353) {};
\node[vertex] (nb) at (-0.00000, -0.50000, 1.61803) {};
\node[vertex] (nA) at (-0.40451, 0.75000, 1.46353) {};
\node[vertex] (naa) at (0.25000, 1.15451, 1.21353) {};
\node[vertex] (nba) at (-0.40451, -0.75000, 1.46353) {};
\node[vertex] (nab) at (1.21353, 0.25000, 1.15451) {};
\node[vertex] (nAb) at (-1.21353, 0.25000, 1.15451) {};
\node[vertex] (nbA) at (0.40451, -0.75000, 1.46353) {};
\node[vertex] (nAA) at (-0.25000, 1.15451, 1.21353) {};
\node[vertex] (nbaa) at (-0.25000, -1.15451, 1.21353) {};
\node[vertex] (naba) at (1.21353, -0.25000, 1.15451) {};
\node[vertex] (nAba) at (-1.46353, 0.40451, 0.75000) {};
\node[vertex] (naab) at (0.75000, 1.46353, 0.40451) {};
\node[vertex] (nbab) at (-1.21353, -0.25000, 1.15451) {};
\node[vertex] (nAAb) at (-0.75000, 1.46353, 0.40451) {};
\node[vertex] (nabA) at (1.46353, 0.40451, 0.75000) {};
\node[vertex] (nbAA) at (0.25000, -1.15451, 1.21353) {};
\node[vertex] (nabaa) at (1.46353, -0.40451, 0.75000) {};
\node[vertex] (nAbaa) at (-1.61803, 0.00000, 0.50000) {};
\node[vertex] (naaba) at (1.15451, 1.21353, 0.25000) {};
\node[vertex] (nAAba) at (-0.50000, 1.61803, 0.00000) {};
\node[vertex] (nbaab) at (-0.75000, -1.46353, 0.40451) {};
\node[vertex] (nAbab) at (-1.15451, 1.21353, 0.25000) {};
\node[vertex] (nbAAb) at (0.75000, -1.46353, 0.40451) {};
\node[vertex] (naabA) at (0.50000, 1.61803, 0.00000) {};
\node[vertex] (nbabA) at (-1.46353, -0.40451, 0.75000) {};
\node[vertex] (nabAA) at (1.61803, -0.00000, 0.50000) {};
\node[vertex] (naabaa) at (1.15451, 1.21353, -0.25000) {};
\node[vertex] (nAAbaa) at (-0.75000, 1.46353, -0.40451) {};
\node[vertex] (nbaaba) at (-1.15451, -1.21353, 0.25000) {};
\node[vertex] (nbAAba) at (0.50000, -1.61803, 0.00000) {};
\node[vertex] (nabaab) at (1.15451, -1.21353, 0.25000) {};
\node[vertex] (nAbaab) at (-1.61803, 0.00000, -0.50000) {};
\node[vertex] (nabAAb) at (1.61803, -0.00000, -0.50000) {};
\node[vertex] (nbaabA) at (-0.50000, -1.61803, 0.00000) {};
\node[vertex] (nAbabA) at (-1.15451, 1.21353, -0.25000) {};
\node[vertex] (naabAA) at (0.75000, 1.46353, -0.40451) {};
\node[vertex] (nbaabaa) at (-1.15451, -1.21353, -0.25000) {};
\node[vertex] (nbAAbaa) at (0.75000, -1.46353, -0.40451) {};
\node[vertex] (nAbaaba) at (-1.46353, 0.40451, -0.75000) {};
\node[vertex] (nabAAba) at (1.46353, -0.40451, -0.75000) {};
\node[vertex] (naabaab) at (1.46353, 0.40451, -0.75000) {};
\node[vertex] (nAAbaab) at (-0.25000, 1.15451, -1.21353) {};
\node[vertex] (naabAAb) at (0.25000, 1.15451, -1.21353) {};
\node[vertex] (nabaabA) at (1.15451, -1.21353, -0.25000) {};
\node[vertex] (nAbaabA) at (-1.46353, -0.40451, -0.75000) {};
\node[vertex] (nbaabAA) at (-0.75000, -1.46353, -0.40451) {};
\node[vertex] (nAbaabaa) at (-1.21353, 0.25000, -1.15451) {};
\node[vertex] (nabAAbaa) at (1.21353, -0.25000, -1.15451) {};
\node[vertex] (naabAAba) at (0.40451, 0.75000, -1.46353) {};
\node[vertex] (nbAAbaab) at (0.25000, -1.15451, -1.21353) {};
\node[vertex] (nbaabAAb) at (-0.25000, -1.15451, -1.21353) {};
\node[vertex] (naabaabA) at (1.21353, 0.25000, -1.15451) {};
\node[vertex] (nAAbaabA) at (-0.40451, 0.75000, -1.46353) {};
\node[vertex] (nAbaabAA) at (-1.21353, -0.25000, -1.15451) {};
\node[vertex] (naabAAbaa) at (0.00000, 0.50000, -1.61803) {};
\node[vertex] (nbaabAAba) at (-0.40451, -0.75000, -1.46353) {};
\node[vertex] (nabAAbaab) at (0.40451, -0.75000, -1.46353) {};
\node[vertex] (nbaabAAbaa) at (-0.00000, -0.50000, -1.61803) {};
\draw[gena] (n) -- (nA);
\draw[genb] (n) -- (nb);
\draw[gena] (na) -- (n);
\draw[genb] (na) -- (nab);
\draw[gena] (nb) -- (nbA);
\draw[genb] (nb) -- (n);
\draw[gena] (nA) -- (nAA);
\draw[genb] (nA) -- (nAb);
\draw[gena] (naa) -- (na);
\draw[genb] (naa) -- (naab);
\draw[gena] (nba) -- (nb);
\draw[genb] (nba) -- (nbab);
\draw[gena] (nab) -- (nabA);
\draw[genb] (nab) -- (na);
\draw[gena] (nAb) -- (nbab);
\draw[genb] (nAb) -- (nA);
\draw[gena] (nbA) -- (nbAA);
\draw[genb] (nbA) -- (naba);
\draw[gena] (nAA) -- (naa);
\draw[genb] (nAA) -- (nAAb);
\draw[gena] (nbaa) -- (nba);
\draw[genb] (nbaa) -- (nbaab);
\draw[gena] (naba) -- (nab);
\draw[genb] (naba) -- (nbA);
\draw[gena] (nAba) -- (nAb);
\draw[genb] (nAba) -- (nAbab);
\draw[gena] (naab) -- (naabA);
\draw[genb] (naab) -- (naa);
\draw[gena] (nbab) -- (nbabA);
\draw[genb] (nbab) -- (nba);
\draw[gena] (nAAb) -- (nAbab);
\draw[genb] (nAAb) -- (nAA);
\draw[gena] (nabA) -- (nabAA);
\draw[genb] (nabA) -- (naaba);
\draw[gena] (nbAA) -- (nbaa);
\draw[genb] (nbAA) -- (nbAAb);
\draw[gena] (nabaa) -- (naba);
\draw[genb] (nabaa) -- (nabaab);
\draw[gena] (nAbaa) -- (nAba);
\draw[genb] (nAbaa) -- (nAbaab);
\draw[gena] (naaba) -- (naab);
\draw[genb] (naaba) -- (nabA);
\draw[gena] (nAAba) -- (nAAb);
\draw[genb] (nAAba) -- (naabA);
\draw[gena] (nbaab) -- (nbaabA);
\draw[genb] (nbaab) -- (nbaa);
\draw[gena] (nAbab) -- (nAbabA);
\draw[genb] (nAbab) -- (nAba);
\draw[gena] (nbAAb) -- (nabaab);
\draw[genb] (nbAAb) -- (nbAA);
\draw[gena] (naabA) -- (naabAA);
\draw[genb] (naabA) -- (nAAba);
\draw[gena] (nbabA) -- (nAbaa);
\draw[genb] (nbabA) -- (nbaaba);
\draw[gena] (nabAA) -- (nabaa);
\draw[genb] (nabAA) -- (nabAAb);
\draw[gena] (naabaa) -- (naaba);
\draw[genb] (naabaa) -- (naabaab);
\draw[gena] (nAAbaa) -- (nAAba);
\draw[genb] (nAAbaa) -- (nAAbaab);
\draw[gena] (nbaaba) -- (nbaab);
\draw[genb] (nbaaba) -- (nbabA);
\draw[gena] (nbAAba) -- (nbAAb);
\draw[genb] (nbAAba) -- (nbaabA);
\draw[gena] (nabaab) -- (nabaabA);
\draw[genb] (nabaab) -- (nabaa);
\draw[gena] (nAbaab) -- (nAbaabA);
\draw[genb] (nAbaab) -- (nAbaa);
\draw[gena] (nabAAb) -- (naabaab);
\draw[genb] (nabAAb) -- (nabAA);
\draw[gena] (nbaabA) -- (nbaabAA);
\draw[genb] (nbaabA) -- (nbAAba);
\draw[gena] (nAbabA) -- (nAAbaa);
\draw[genb] (nAbabA) -- (nAbaaba);
\draw[gena] (naabAA) -- (naabaa);
\draw[genb] (naabAA) -- (naabAAb);
\draw[gena] (nbaabaa) -- (nbaaba);
\draw[genb] (nbaabaa) -- (nAbaabA);
\draw[gena] (nbAAbaa) -- (nbAAba);
\draw[genb] (nbAAbaa) -- (nbAAbaab);
\draw[gena] (nAbaaba) -- (nAbaab);
\draw[genb] (nAbaaba) -- (nAbabA);
\draw[gena] (nabAAba) -- (nabAAb);
\draw[genb] (nabAAba) -- (nabaabA);
\draw[gena] (naabaab) -- (naabaabA);
\draw[genb] (naabaab) -- (naabaa);
\draw[gena] (nAAbaab) -- (nAAbaabA);
\draw[genb] (nAAbaab) -- (nAAbaa);
\draw[gena] (naabAAb) -- (nAAbaab);
\draw[genb] (naabAAb) -- (naabAA);
\draw[gena] (nabaabA) -- (nbAAbaa);
\draw[genb] (nabaabA) -- (nabAAba);
\draw[gena] (nAbaabA) -- (nAbaabAA);
\draw[genb] (nAbaabA) -- (nbaabaa);
\draw[gena] (nbaabAA) -- (nbaabaa);
\draw[genb] (nbaabAA) -- (nbaabAAb);
\draw[gena] (nAbaabaa) -- (nAbaaba);
\draw[genb] (nAbaabaa) -- (nAAbaabA);
\draw[gena] (nabAAbaa) -- (nabAAba);
\draw[genb] (nabAAbaa) -- (nabAAbaab);
\draw[gena] (naabAAba) -- (naabAAb);
\draw[genb] (naabAAba) -- (naabaabA);
\draw[gena] (nbAAbaab) -- (nabAAbaab);
\draw[genb] (nbAAbaab) -- (nbAAbaa);
\draw[gena] (nbaabAAb) -- (nbAAbaab);
\draw[genb] (nbaabAAb) -- (nbaabAA);
\draw[gena] (naabaabA) -- (nabAAbaa);
\draw[genb] (naabaabA) -- (naabAAba);
\draw[gena] (nAAbaabA) -- (naabAAbaa);
\draw[genb] (nAAbaabA) -- (nAbaabaa);
\draw[gena] (nAbaabAA) -- (nAbaabaa);
\draw[genb] (nAbaabAA) -- (nbaabAAba);
\draw[gena] (naabAAbaa) -- (naabAAba);
\draw[genb] (naabAAbaa) -- (nbaabAAbaa);
\draw[gena] (nbaabAAba) -- (nbaabAAb);
\draw[genb] (nbaabAAba) -- (nAbaabAA);
\draw[gena] (nabAAbaab) -- (nbaabAAbaa);
\draw[genb] (nabAAbaab) -- (nabAAbaa);
\draw[gena] (nbaabAAbaa) -- (nbaabAAba);
\draw[genb] (nbaabAAbaa) -- (naabAAbaa);
\end{tikzpicture}
  \end{sidecaption}
\end{figure}

\section{Subgroups of free groups}
\label{sec:subgroups-free}

We now study subgroups of free groups.\marginnote{%
  Our discussion follows the work of
  \citeauthor{Swan2022}\footnotemark{}.}\footcitetext{Swan2022}
We'll eventually prove the Nielsen--Schreier theorem,
which states that a finite index subgroup $H$ of a free group $\FG_S$ is itself a free group.
Furthermore, when $S$ is finite, the set of free generators of $H$ is itself finite.

% Should this be in subgroup chapter?
\begin{definition}\label{def:finite-index}
  A subgroup of a group $G$ has \emph{finite index} $m$ if, when represented as a
  transitive $G$-set $X : \BG \to \Set$, $X(\sh_G)$ is a finite set of cardinality $m$.
\end{definition}
In that case, of course, all the twisted sets $X(t)$ for $t:\BG$ are also finite sets
of cardinality $m$.

Recall from~\cref{def:set-of-subgroups} that the underlying classifying type of the subgroup is the total type $\sum_{t:\BG}X(t)$.
Thus we'll use the Flattening~\cref{def:graph-quotient-flattening} to analyze
this in case $G \jdeq \FG_S$, so we need to show that the resulting graph
has a quotient equivalent to $\bn 1/T$ for some set $T$.

We do this by finding a ``spanning tree'' in the graph.

\begin{definition}
  A graph $(V,E)$ is \emph{connected}\index{connected!graph} if $V/E$ is a connected type
  and it's a \emph{tree} if $V/E$ is contractible.
\end{definition}

\begin{definition}
  A \emph{subgraph} of a graph $(V,E)$ consists of a subtype $h : U \hookrightarrow V$
  of the vertices along with, for every pair of vertices $v,w$ in $U$,
  a subtype $D(v,w)$ of the edges $E(v,w)$.
\end{definition}
If we represent graphs by source and target maps, then this amounts to embeddings
$h : U \hookrightarrow V$ and $k : D \hookrightarrow E$ along with witnesses that the
following squares commute:
\[
  \begin{tikzcd}
    D \ar[r,hook,"k"]\ar[d,"s"'] & E \ar[d,"s"] \\
    U \ar[r,hook,"h"'] & V
  \end{tikzcd}\quad
  \begin{tikzcd}
    D \ar[r,hook,"k"]\ar[d,"t"'] & E \ar[d,"t"] \\
    U \ar[r,hook,"h"'] & V
  \end{tikzcd}
\]
\begin{definition}
  A \emph{spanning tree} in a graph $(V,E)$ is a subgraph $(U,D)$
  such that $(U,D)$ is a tree, and the embedding of the vertices
  $U \hookrightarrow V$ is an equivalence.
\end{definition}
Equivalently, it's given by subtypes of the edges (leaving the vertices alone)
such that the underlying graph is a tree.
Very often we'll require that the edge embeddings are decidable,
\ie we can decide whether a given edge $e : E(v,w)$ is part of the tree.

\begin{marginfigure}
  \begin{tikzpicture}
    \draw (-.25,0.9) ellipse (.55 and 1.2);
    \node (X) at (0,2.4) {$V_0$};
    \draw (1.2,1.1) ellipse (.7 and 1);
    \node (Y) at (1,2.3) {$V_1$};
    \node[dot,label=below:$v_0$] (x) at (-.1,0.4) {};
    \node[dot,label=above:$v_1$] (y) at (1.6,1.3) {};
    \node[dot,label=below:$u_0$] (s) at (-.1,1.4) {};
    \node[dot,label=below:$u_1$] (t) at (1.0,1.3) {};
    \draw[casblue,dotted,<->] (s) -- (t) node[midway,label=above:$e$] {};
    \draw[dashed] (0.5,1.1) ellipse (1.5 and 1.9);
    \node (XY) at (-0.75,3.05) {$V_0 \amalg V_1$};
  \end{tikzpicture}
  \caption{A connected graph with a crossing edge}
  \label{fig:crossing-edge}
\end{marginfigure}
\begin{lemma}\label{lem:crossing-edge}
  Suppose we have a connected graph $(V,E)$ whose type of vertices
  decomposes as a binary sum $V \equivto V_0 \amalg V_1$
  and we have $v_0:V_0$ and $v_1:V_1$.
  Then there merely exists an edge $e$ either with source in $V_0$ and target in $V_1$
  or vice versa.
\end{lemma}
The situation is illustrated in~\cref{fig:crossing-edge}, where we assume
there is an edge relation on the binary sum that gives a connected graph,
and hence there must be a ``crossing edge'' $e$, going either from $V_0$ to $V_1$
or the other way.
\begin{proof}
  We may assume $V \jdeq V_0 \amalg V_1$ by path induction.
  The idea is then to define a family of propositions $P : V/E \to \Prop$
  that, one the one hand, is trivially true over $V_0$, and on the other hand
  expresses our desired goal, the existence of a ``crossing edge'', over $V_1$.

  We define $P(z)$, for $z:V/E$, by the induction principle for the graph quotient $V/E$,
  with a subsidiary case distinction for the vertex $v : V_0 \amalg V_1$.
  We set $([\inl v]) \defeq \true$ for $v : V_0$ and
  \[
    P([\inr v]) \defeq \Trunc*{\sum_{u_0:V_0}\sum_{u_1:V_1}
      \bigl(E(u_0,u_1) \amalg E(u_1,u_0)\bigr)}
  \]
  for $v : V_1$.
  We must then prove that the propositions $P(v)$ and $P(v')$ are equivalent
  whenever there's an edge from $v$ to $v'$.
  This is trivially the case when $v,v'$ lie in the same summand,
  and it's also the case when they lie in different summands, since then
  we get a witness for the truth over $V_1$.

  Since $V/E$ is connected, $P$ must have a constant truth value,
  and since $P([\inl{v_0}])\jdeq\true$, every $P(z)$ is true.
  Hence also $P([\inr{v_1}])$ is true, which is exactly what we wanted.
\end{proof}

\begin{lemma}\label{lem:spanning-tree-step}
  Fix a connected graph $(V,E)$ where $V$ has decidable equality and $E$ is a family of sets.
  For any subgraph $(U,D)$, where the embedding $U\hookrightarrow V$ is decidable,
  and with vertices $u \in U$ and $v \in V \setminus U$,
  there merely exists a larger subgraph with exactly one more vertex and one more edge,
  $(U\amalg\bn 1, D\amalg\bn 1)$ such that the induced map on graph quotients
  $U/D \to (U\amalg\bn 1)/(D\amalg\bn 1)$ is an equivalence.\marginnote{%
    \begin{tikzcd}[sep=small,ampersand replacement=\&]
      D \ar[r,hook]\ar[d,shift left]\ar[d,shift right] \&
      D\amalg\bn 1 \ar[r,hook]\ar[d,shift left]\ar[d,shift right] \&
      E\ar[d,shift left]\ar[d,shift right] \\
      U \ar[r,hook]\ar[d] \&
      U\amalg\bn 1 \ar[r,hook]\ar[d] \&
      V \ar[d] \\
      U/D \ar[r,"\equiv"] \&
      (U\amalg\bn 1)/(D\amalg\bn 1)\ar[r] \&
      V/E
    \end{tikzcd}}
\end{lemma}
\begin{proof}
  Since the embedding $U\hookrightarrow V$ is decidable,
  we can write $V$ as the binary sum $U \amalg (V\setminus U)$.
  Apply~\cref{lem:crossing-edge} to find a ``crossing edge'' $e$,
  and form the new subgraph $(U\amalg\bn 1, D\amalg\bn 1)$ by adding
  the incident vertex not in $U$ as well as the edge $e$ itself.
  The embedding $U\amalg\bn 1 \to V$ is still decidable, since $V$ has decidable equality.
  Finally, we have
  \[
    (U\amalg\bn 1) / (D\amalg\bn 1) \equivto \bigl((U\amalg\bn 1)/\bn 1\bigr)/D
    \equivto U/D,
  \]
  using~\cref{xca:graph-quotient-in-steps,xca:graph-quotient-whisker}, as desired.
\end{proof}

\begin{lemma}\label{lem:spanning-tree}
  Let $(V,E)$ be a connected graph where $V$ is an $n$-element set,
  and $E$ is a family of decidable sets. Then the graph merely has a spanning tree
  with exactly $n-1$ edges.
\end{lemma}

\begin{marginfigure}
  \begin{tikzpicture}
    \node[dot] (v1) at (0,0) {};
    \node[dot] (v2) at (1,0) {};
    \node[dot] (v3) at (-.4,1) {};
    \node[dot] (v4) at (.7,.9) {};
    \node[dot] (v5) at (.5,1.8) {};
    \node[dot] (v6) at (1.5,1.6) {};
    \draw[->,casred] (v1) -- (v2);
    \draw[->,casred] (v1) to[out=135,in=270] (v3);
    \draw[->,casred] (v3) -- (v4);
    \draw[->] (v4) -- (v2);
    \draw[->,casred] (v6) to[out=200,in=80] (v3);
    \draw[->] (v2) to[out=45,in=270] (v6);
    \draw[->] (v6) -- (v4);
    \draw[->,casred] (v6) to[out=90,in=45] (v5);
  \end{tikzpicture}
  \caption{A connected graph on $6$ vertices with a spanning tree indicated
    in~\textcolor{casred}{red}.}
  \label{fig:spanning-tree-example}
\end{marginfigure}

\begin{proof}
  We show by induction on $k$, with $1\le k\le n$, that there merely exists
  a subgraph $(U,D)$ with $k$ vertices, $k-1$ edges, and $U/D$ contractible,
  \ie the graph $(U,D)$ is a tree.

  For $k\jdeq 1$, we use that $V/E$ is connected to get that $V$ merely has a vertex $v$.
  This then defines the desired subgraph on one vertex with no edges,
  and this is clearly a tree.

  Suppose we have such a desired subgraph $(U,D)$
  with $k$ vertices and $k-1$ edges and $k<n$.
  Since $V$ is finite, there exists vertices $u \in U$ and $v \in V \setminus U$.
  Now apply~\cref{lem:spanning-tree-step} to get the next subgraph.

  Finally, the subgraph $(U,D)$ with $n$ vertices and $n-1$ edges gives
  the desired spanning tree, and any embedding of an $n$-element set
  in another $n$-element set is an equivalence.\footnote{%
    Assuming the Axiom of Choice, we can show the mere existence of a spanning tree
    in any graph $(V,E)$ with a sets of vertices and edges.
    See the above work by~\citeauthor{Swan2022}.}
\end{proof}

\begin{theorem}[Nielsen--Schreier Theorem]
  Suppose that $S$ is a set with decidable equality and
  $X : \BFG_S \to \Set$ defines a finite index subgroup.
  Then $\sum_{z:\BFG_S}X(z)$ is merely equivalent to the classifying type of a free group.

  Moreover, if $S$ is a finite set of cardinality $n$ and the subgroup has index $m$,
  then the free group can be taken on a finite set of cardinality $m(n-1)+1$.
\end{theorem}
\begin{proof}
  By the Flattening~\cref{def:graph-quotient-flattening},
  we have an equivalence $\sum_{z:\BFG_S}X(z) \equivto X(\base)/E$,
  where $E(x,y) \jdeq \sum_{s:S}(\pathover{x}{X}{\Sloop_s}{y})$.
  By the finite index assumption, $X(\base)$ is a finite set,
  say, of cardinality $m$,
  and since both $S$ and $X(\base)$ are decidable, so is $E$.

  By~\cref{lem:spanning-tree}, there merely exists a spanning tree with
  $m-1$ edges $E_0$. Then we have,
  using \cref{xca:graph-quotient-in-steps}:
  \[
    V/E \equivto V/(E_0\amalg E_1) \equivto (V/E_0)/E_1
    \equivto \bn 1/E_1 \equivto \BFG_{E_1}
  \]
  This establishes the first claim.

  If furthermore $S$ has cardinality $n$, then the graph $(X(\base),E)$
  has $mn$ edges, as there are precisely $n$ outgoing edges from each vertex.
  Since $E_0$ has $m-1$ edges,
  that leaves $mn - (m-1) = mn-m+1 = m(n-1)+1$ edges in $E_1$, as desired.
\end{proof}
(This also has an automata theoretic proof, see below.)

\section{Intersecting subgroups}
\label{sec:intersecting-subgroups}

Stallings folding\footcite{Stallings1991}.

\begin{theorem}
  Let $H$ be a finitely generated subgroup of $F(S)$ and let $u\in\tilde S^*$
  be a reduced word. Then $u$ represents an element of $H$ if and only if
  $u$ is recognized by the Stallings automaton $\mathcal{S}(H)$.
\end{theorem}
\begin{theorem}
  Let $H$ be a finitely generated subgroup of $F(S)$.
  Then $H$ has finite index if and only if $\mathcal{S}(H)$ is total.

  Furthermore, in this case the index equals the number of vertices of
  $\mathcal{S}(H)$.
\end{theorem}
\begin{corollary}
  If $H$ has index $n$ in $F(S)$, then $\casop{\constant{rk}} H = 1 + n(\casop{\constant{card}} S-1)$.
\end{corollary}

\begin{theorem}\label{thm:howson-neumann}
  Suppose $H_1,H_2$ are two subgroups of $F$ with finite indices $h_1,h_2$.
  Then the intersection $H_1\cap H_2$ has finite index at most $h_1h_2$.
\end{theorem}
\marginnote{The qualitative part of \cref{thm:howson-neumann}
  is known as \emph{Howson's theorem}, while the inequality
  is known as \emph{Hanna Neumann's inequality}.
  Hanna's son, Walter Neumann, conjectured that the $2$ could be removed,
  and this was later proved independently by Joel Friedman
  and Igor Mineyev.}

\section{Connections with automata (*)}

($S$ is still a fixed finite set.)

Let $\iota : F(S) \to \tilde S^*$ map an element of the free group to the corresponding reduced word.
The kernel of $\iota$ is the \emph{2-sided Dyck language} $\mathcal D_S$.

The following theorem is due to Benois.
\begin{theorem}
  A subset $X$ of $F(S)$ is rational if and only if
  $\iota(X)\subseteq \tilde S^*$ is a regular language.
\end{theorem}

\begin{lemma}
  Let $\rho : \tilde S^* \to \tilde S^*$ map a word to its reduction.
  Then $\rho$ maps regular languages to regular languages.
\end{lemma}

The following is due to Sénizergues:
\begin{theorem}
  A rational subset of $F(S)$ is either disjunctive or recognizable.
\end{theorem}

Given a surjective monoid homomorphism $\alpha : S^* \to G$,
we define the corresponding \emph{matched homomorphism} $\tilde\alpha : \tilde S^* \to G$ by $(\tilde\alpha(a^{-1}) \defeq \alpha(a)^{-1}$.
\begin{theorem}[?]
  Consider a f.g.\ group $G$ with a surjective homomorphism
  $\alpha : F(S) \to G$. A subset $X$ of $G$ is recognisable by a finite $G$-action
  if and only if $\tilde\alpha^{-1}(X) \subseteq \tilde S^*$ is rational (i.e., regular).
\end{theorem}
\begin{theorem}[Chomsky--Schützenberger]
  A language $L \subseteq T^*$ is context-free if and only if
  $L = h(R \cap D_S)$ for some finite $S$,
  where $h : T^* \to \tilde S^*$ is a homomorphism,
  $R \subseteq \tilde S^*$ is a regular language, and
  $D_S$ is the Dyck language for $S$.\footnote{%
    References TODO. The theorem is also true if we replace $D_S$
    by its one-sided variant, but in this case it reduces to
    the well-known equivalence between context-free languages
    and languages recognizable by pushdown automata.}
\end{theorem}
\begin{theorem}[Muller--Schupp, ?]
  Suppose $\tilde\alpha : \tilde S^* \to G$ is a surjective matched homomorphism
  onto a group $G$. Then $G$ is virtually free (i.e., $G$ has a normal free subgroup of finite index) if and only if $\ker(\tilde\alpha)$ is a context-free language.
\end{theorem}
\begin{theorem}

\end{theorem}
The Stallings automaton is an \emph{inverse automaton}:
it's deterministic,
and there's an edge $(p,a,q)$ if and only if there's one $(q,A,p)$.
We can always think of the latter as the \emph{reverse} edge.
(It's then also deterministic in the reverse direction.)

Two vertices $p,q$ get identified in the Stallings graph/automaton
if and only if there is a run from $p$ to $q$ with a word $w$
whose reduction is $1$. (So a word like $aAAaBBbb$.)

\begin{theorem}
  Let $X \subseteq F(S)$. Then $Y$ is a coset $Hw$ with $H$
  a finitely generated subgroup,
  if and only if
  there is a finite state inverse automaton whose language (after reduction)
  is $Y$.
\end{theorem}

\begin{corollary}
  The generalized word problem in $F(S)$ is solvable:
  Given a finitely generated subgroup $H$, and a word $u : \tilde S^*$,
  we can decide whether $u$ represents an element of $H$.
\end{corollary}
\marginnote{%
  The Stallings automaton for $H$ can be constructed in time
  $O(n \log^*n)$, where $n$ is the sum of the lengths of the generators for $H$.
  [Cite: Touikan: A fast algorithm for Stallings' folding process.]
  Once this has been constructed, we can solve membership in $H$ in linear time.}

As above, we get a basis for $H$ as a free group from a spanning tree in
$\mathcal{S}(H)$.

\begin{theorem}
  We can decide whether two f.g.\ subgroups of $F(S)$ are conjugate.
  Moreover, a f.g.\ subgroup $H$ is normal if and only if $\mathcal{S}(H)$
  is vertex-transitive.
\end{theorem}
\begin{proof}
  $G,H$ are conjugate of and only if their cores are equal.
\end{proof}

There are other connections between group theory and language theory:
\begin{theorem}[Anisimov and Seifert]
  A subgroup $H$ of $G$ is rational if and only if $H$ is finitely generated.
\end{theorem}
\begin{theorem}
  A subgroup $H$ of $G$ is recognizable if and only if it has finite index.
\end{theorem}
%%% Local Variables:
%%% mode: latex
%%% fill-column: 144
%%% TeX-master: "book"
%%% End:

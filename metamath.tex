\chapter{Metamathematical remarks}
\label{app:metamath}

Metamathematics is the study of mathematical theories as
mathematical objects in themselves.
This book is primarily a mathematical theory of symmetries.
Occasionally, however, we have made statements like
``the law of the excluded middle is not provable in our theory''.
This is a statement \emph{about}, and not \emph{in}, the type theory of this book.
As such it is a metamathematical statement.

Sometimes it is possible to encode statements
about a theory in the language of the theory itself.%
\marginnote{We leave aside that this sometimes can be done in different ways.
  Historically, the first way was by ``Gödel-numbering'':
  encoding all bits of syntax, including statements, as natural numbers,
  so that the constructions and deductions of the theory correspond to definable operations on the encoding numbers.
  In type theory, there are usually much more perspicacious ways of encoding mathematical theories
  using types and type families.}
Even if true, the encoded metamathematical statement can
be unprovable in the theory itself.
The most famous example is G\"{o}del's second incompleteness theorem.\footnote{%
  The original reference is~\citeauthor{Goedel2nd}\footnotemark{},
  translated into English in~\citeauthor{Heijenoort-source}\footnotemark{}.
  For an accessible introduction, see for instance~\citeauthor{Franzen-Goedel}\footnotemark{}
  or~\citeauthor{Smullyan-Goedel}\footnotemark{}.}%
\addtocounter{footnote}{-4}%
\stepcounter{footnote}\footcitetext{Goedel2nd}%
\stepcounter{footnote}\footcitetext{Heijenoort-source}%
\stepcounter{footnote}\footcitetext{Franzen-Goedel}%
\stepcounter{footnote}\footcitetext{Smullyan-Goedel}.
G\"{o}del encoded, for any theory $T$ extending Peano Arithmetic and satisfying
some general assumptions, the statement that $T$ is consistent as
a statement $\Con(T)$ in Peano Arithmetic.
Then he showed that $\Con(T)$ is not provable in $T$.

We say that a metamathematical statement about a theory $T$
is \emph{internally} provable if its encoding is provable in $T$.
For example, the metamathematical statement ``if $P$ is unprovable in $T$,
then $T$ is consistent'' is internally provable in $T$, for any $T$ that
satisfies the assumptions of G\"{o}del's second incompleteness theorem.

The type theory in this book satisfies the assumptions of
G\"{o}del's second incompleteness theorem, which include, of course,
the assumption that $T$ is consistent. Thus there is no hope
that we can prove the consistency of our type theory internally.
Moreover, by the previous paragraph, we must be prepared that
no unprovability statement can be proved internally.

[TODO For consistency of UA, LEM, etc, refer to simplicial set model \footcite{KapLum}.
For unprovability of LEM, refer to cubical set model \footcite{BezCoqHub}.]

One property of type theory that we will use is \emph{canonicity}.
We call an expression \emph{closed} if it does not contain free variables.
One example of canonicity is that every closed expression of type $\NN$
is a \emph{numeral}, that is, either $0$ or $S(n)$ for some numeral $n$.
Another example of canonicity is that every closed expression of 
type $L\coprod R$ is either of the form $\inl{l}$ for some $l:L$ or
of the form $\inr{r}$ for some $r:R$.

Both examples of canonicity above are clearly related to the
inductive definitions of the types involved: they are
expressed in terms of the constructors of the respective types.
One may ask what canonicity then means for the empty type $\false$,
defined in \cref{sec:finite-types} as the inductive type
with no constructors at all. The answer is that canonicity for $\false$
means that there cannot be a closed expression of type $\false$.
But this actually means that our type theory is consistent!
Therefore we cannot prove general canonicity internally.

[TODO no canonical forms: $x:\NN$, $\trp[P]{\ua(\id)}(0): \NN$,
with $P\defeq (p:\true \mapsto \NN)$
and (problematic) $\trp[Q]{\Sloop}(0): \NN$
with $Q\defeq (z:\Sc \mapsto \NN)$.]

[TODO A second important property of our theory is 
that one can compute canonical forms.]

\section{Definitional equality}
\label{sec:defeq}

\subsection{Basics}
\label{sec:defeq-basics}

The concept of definition was introduced in \cref{univalent-mathematics},
together with what it means to be \emph{the same by definition}. 
Being the same by definition, or being definitionally equal, 
(NB appears for the first time on p. 26!) 
is a relationship between syntactic expressions.
In this section we provide more details about this relationship.

There are four basic forms of definitional equality:
\begin{enumerate}
\item\label{it:exp-defeq} Resulting from making an explicit definition, 
e.g., $1 \defeq \Succ(0)$, after which we have $1 \jdeq \Succ(0)$;%
\footnote{The notation $\defeq$ tells the reader that we make a definition
(or reminds the reader that this definition has been made).}

\item\label{it:imp-defeq} Resulting from making an implicit definition, 
like we do in inductive definitions,
e.g., $n+0 \defeq n$ and $n+\Succ(m) \defeq \Succ(n+m)$,
after which we have $n+0 \jdeq n$ and $n+\Succ(m) \jdeq \Succ(n+m)$;

\item\label{it:beta} Simplifying the application of an explicitly defined 
function to an argument, e.g., $(x \mapsto e_x)(a) \jdeq e_a$;

\item\label{it:eta} Simplifying $(x \mapsto e_x)$ to $f$ when $e_x$
is the application of the function $f$ to the variable $x$,
e.g., $(x \mapsto S(x)) \jdeq S$.
\end{enumerate}

Definitional equality is the \emph{congruence closure} of these 
four basic forms, that is, the smallest reflexive, symmetric, transitive
and congruent relation that contains all instances of the four basic forms.
Here a congruent relation is a relation that is closed under all syntactic
operations of type theory. One such operation is substitution, so that we
get from the examples above that, e.g., $1+0 \jdeq 1$ 
and $n+\Succ(\Succ(m)) \jdeq \Succ(n+\Succ(m))$. Another important
operation is application. For example, we can apply $\Succ$ to each of
the sides of $n+\Succ(m) \jdeq \Succ(n+m)$ and get 
$\Succ(n+\Succ(m)) \jdeq \Succ(\Succ(n+m))$, and also
$n+\Succ(\Succ(m)) \jdeq \Succ(\Succ(n+m))$ by transitivity.

Let's elaborate $\id \circ f \jdeq f$ claimed on page~\pageref{page:idofetaf}.
The definitions used on the left hand side are 
$\id \defeq (y \mapsto y)$ and $g \circ f \defeq (x \mapsto g(f(x)))$.
In the latter definition we substitute $\id$ for $g$ and get
$\id \circ f \jdeq (x \mapsto \id(f(x)))$. Unfolding $\id$ we get
$(x \mapsto \id(f(x))) \jdeq (x \mapsto (y\mapsto y)(f(x)))$.
Applying \ref{it:beta} we can substitute $f(x)$ for $(y\mapsto y)(f(x)))$
and get $(x \mapsto (y\mapsto y)(f(x))) \jdeq (x \mapsto f(x))$.
By \ref{it:eta} the right hand side is definitionally equal to $f$.
Indeed $\id \circ f \jdeq f$ by transitivity.

Definitional equality is also relevant for typing.
For example, let $A: \UU$ and $P: A\to\UU$. If $B\jdeq A$,
then  $ (B\to\UU)\jdeq(A\to\UU)$ by congruence, and also $P: B\to\UU$,
and even $\prod_{x:B} P(x) \jdeq \prod_{x:A} P(x)$.


%Discuss later: If syntactic expressions $e$ and $e'$ are definitionally equal, denoted by $e\jdeq e'$, then $e$ and $e'$ have the same type $T$, then $\refl e : e=_T e'$; the converse is not true.

\subsection{Deciding definitional equality (not updated yet)}
\label{sec:defeq-computation}

By a \emph{decision procedure} we mean a terminating algorithmic procedure that 
answers a yes/no question.
Although it is possible to enumerate all true definitional equalities,
this does not give a test that answers whether or not a given instance $e\jdeq e'$ holds.
In particular when $e\jdeq e'$ does not hold, such an enumeration will not terminate.
A test of definitional equality is important for type checking,
as the examples in the last paragraph of the previous section show.

A better approach to a test of definitional equality is the following.
First direct the four basic forms of definitional equality from left to right
as they are given.\footnote{%
TODO: think about the last, $\eta$.}
For the first two forms this can be viewed as unfolding definitions,
and for the last two forms as simplifying function application and (unnecessary) 
abstraction, respectively.
This defines a basic reduction relation, and we write $e\to e'$ if $e'$ can
be obtained by a basic reduction of a subexpression in $e$. 
The reflexive transitive closure of $\to$ is denoted by $\to^*$.
The symmetric closure of $\to^*$ coincides with $\jdeq$.

We mention a few important properties of the relations $\to,\to^*$ and $\jdeq$.
The first is called the Church--Rosser property, and states that,
if $e\jdeq e'$, then there is an expression $c$ such that $e\to^* c$
and $e'\to^* c$. The second is called type safety and states that,
if $e:T$ and $e\to e'$, then also $e':T$.
The third is called termination and states that for well-typed expressions $e$
there is no infinite reduction sequence starting with $e$.
The proofs of Church--Rosser and type safety are long and tedious, but pose no essential
difficulties. For a non-trivial type theory such as in this book the last property,
termination, is extremely difficult and has not been carried out in full detail.
The closest come results on the Coq \footcite{Coq} (TODO: find good reference).

Testing definitional equality of given well-typed terms $e$ and $e'$ can now be done
by reducing them with $\to$ until one reaches irreducible expressions $n$ and $n'$
such that $e\to^* n$ and $e'\to^* n'$, and then comparing $n$ and $n'$. 
Now we have: $e\jdeq e'$ iff $n\jdeq n'$ iff (by Church--Rosser)
there exists a $c$ such that $n\to^* c$ and $n'\to^* c$.
Since $n$ and $n'$ are irreducible the latter is equivalent to
$n$ and $n'$ being identical syntactic expressions.

\section{The Limited Principle of Omniscience}
\label{sec:LPO}

\begin{remark}\label{rem:LPO-solves-halting problem}
Recall the Limited Principle of Omniscience (LPO), \cref{LPO}:
  for any function $P:\NN\to\bn 2$,
  either there is a smallest number $n_0:\NN$ such that $P(n_0)=1$,
  or $P$ is a constant function with value $0$.
We will show that LPO is not provable in our theory.

The argument is based on the halting problem: given a Turing machine
$M$ and an input $n$, determine whether $M$ halts on $n$.
It is known that the halting problem cannot be solved by an algorithm
that can be implemented on a Turing machine.\footnote{It's commonly accepted that
  every algorithm \emph{can} be thus implemented.}

We use a few more facts from computability theory.
First, Turing machines can be enumerated. We denote the $n$\th Turing machine $M_n$,
so we can list the Turing machines in order: $M_0,M_1,\ldots$.
Secondly, there exists a function $T(e,n,k)$ such that $T(e,n,k) = 1$
if $M_e$ halts on input $n$ in at most $k$ steps, and $T(e,n,k) = 0$
otherwise. This function $T$ can be implemented in our theory.

Towards a contradiction, assume we have a closed proof $t$ of LPO in our theory.
We assume as well that $t$ does not depend on any axiom.\footnote{It is possible to weaken the notion
  of canonicity so that the argument still works even if the proof $t$ uses the Univalence Axiom.
Of course, the argument must fail if we allow $t$ to use LEM!}
It is clear that $k\mapsto T(e,n,k)$ is a constant function with value $0$
if and only if $M_e$ does not halt on input $n$. Now consider $t(k\mapsto T(e,n,k))$,
which is an element of a type of the form $L\coprod R$.

We now explain how to solve the halting problem.
Let $e$ and $n$ be arbitrary numerals.
Then $t(k\mapsto T(e,n,k))$ is a closed element of $L\coprod R$.
Hence we can compute its canonical form. If $t(k\mapsto T(e,n,k))\jdeq\inr{r}$ for some
$r:R$, then $k\mapsto T(e,n,k)$ is a constant function with value $0$,
and $M_e$ does not halt on input $n$. If $t(k\mapsto T(e,n,k))\jdeq\inl{l}$ for some
$l:L$, then $M_e$ does halt on input $n$.
Thus we have an algorithm to solve the halting problem
for all $e$ and $n$. Since this is impossible, we have refuted the assumption
that there is a closed proof $t$ of LPO in our theory.
\end{remark}

\section{Topology}
\label{sec:topology}
In this section we will explain how our intuition about types relates to our intuition about topological spaces.

INSERT AN INTRODUCTORY PARAGRAPH HERE.

\begin{remark}
  \label{rem:injectionsurjectionisnotwhatyouthink}
  Our definitions of injections and surjections are lifted directly from the intuition about sets.  However, types need not be sets, and
  thinking of types as spaces may at this point lead to a slight confusion.

  The real line is contractible and the inclusion of the discrete subspace $\{0,1\}$ is, well, an inclusion (of sets, which is the same thing as
  an inclusion of spaces).  However, $\{0,1\}$ is not connected, seemingly contradicting the next result.

  This apparent contradiction is resolved once one recalls the myopic nature of our setup: the contractibility of the real line means that ``all
  real numbers are identical'', and \emph{our} ``preimage of $3{.}25$'' is not a proposition: it contains \emph{both} $0$ and $1$.  Hence
  ``$\{0,1\}\subseteq\mathbb R$'' would not count as an injection in our sense.

  We should actually have been more precise above: we were referring to the \emph{homotopy type} of the real line, rather than the real line itself.\footnote{\label{ft:cohesive}%
    We don't define this formally here,
    see \citeauthor{Shulman-Real-Cohesive}\footnotemark{} for a synthetic account.
    The idea is that the homotopy type $\constant{h}(X)$ of a type $X$
    has a map from $X$, $\iota : X \to \constant{h}(X)$,
    and any continuous function $f : [0,1] \to X$
    gives rise to a path
    $\iota(f(0)) = \iota(f(1))$ in
    $\constant{h}(X)$.}\footcitetext{Shulman-Real-Cohesive}
  We shall later (in the chapters on geometry) make plenty of use of the latter,
  which is as usual a set with uncountably many elements.
\end{remark}

\chapter{Choice for finite sets\titledagger}
\label{ch:choicefin}

This chapter is a short overview of how group theory is involved in
relating different choice principles for families of finite sets.  A
paradigm case is that if we have choice for all families of
$2$-element sets, then we have choice for all families of $4$-element
sets.%
\footnote{This is due to Tarski,
  see~\citeauthor{Jech-AC}\footnotemark{}, p.~107.}\footcitetext{Jech-AC}

The axiom of choice is a principle that we may add to our type theory
(it holds in the standard model), but there are many models where it doesn't hold.

\begin{principle}[The Axiom of Choice]\label{pri:ac}
  For every set $X$ and every family of \emph{non-empty} sets
  $P : X \to \Set_{\ne\emptyset}$,
  there merely exists an dependent function of type $\prod_{x:X}P(x)$.
  In other terms, for any set $X$ and any family of sets $P:X\to\Set$,
  we have
  \begin{equation}\label{eq:ac-impl}
    \prod_{x:X}\Trunc{P(x)} \to \Trunc[\bigg]{\prod_{x:X}P(x)}.\qedhere
  \end{equation}
\end{principle}

\begin{remark}
  We have an equivalence between the Pi-type $\prod_{x:X}P(x)$ and the
  type of sections of the projection map $\prj_1 : \sum_{x:X}P(x) \to X$,
  under which families of non-empty sets correspond to surjections between sets
  (using that $X$ is a set).
  Thus, the axiom of choice equivalently says that any surjection
  between sets admits a section.

  Because of this equivalence, we'll sometimes also call elements of the
  Pi-type \emph{sections}.
\end{remark}

The following is usually called Diaconescu's theorem\footcite{Diaconescu} or the Goodman--Myhill theorem\footcite{Goodman-Myhill}, but it was first observed in a problem in Bishop's book on constructive analysis~\footcite{Bishop}.

\begin{theorem}
  The axiom of choice implies the law of the excluded middle, \cref{pri:lem}.
\end{theorem}

\begin{proof}
  Let $P$ be a proposition, and consider the quotient map $q : \bn 2 \to \bn 2/\sim$,
  where $\sim$ is the equivalence relation on $\bn 2$ satisfying $(0 \sim 1) = P$.
  Like any quotient map, $q$ is surjective, so by the axiom of choice,
  and because our goal is a proposition,
  it has a section $s : \bn 2/\sim \to \bn 2$.
  That is, we also have $q\circ s = \id$.

  Using decidable equality in $\bn 2$, check whether $s([0])$ and $s([1])$ are equal
  or not.

  If they are, then we get the chain of identifications
  $[0] = q(s([0])) = q(s([1])) = [1]$, so $P$ holds.

  If they aren't, then assuming $P$ leads to a contradiction, meaning $\lnot P$ holds.
\end{proof}

We'll now define some restricted variants of the axiom of choice,
because our goal is to see how they relate to each other and to other principles.

\begin{definition}
  Let $\AC$ denote the full axiom of choice, as in~\cref{pri:ac}.
  If we fix the set $X$, and consider \eqref{eq:ac-impl} for arbitrary families $P:X\to\Set$, we call this the \emph{$X$-local axiom of choice}, denoted $\lAC{X}$.

  If we restrict $P$ to take values in $n$-element sets, for some $n:\NN$,
  we denote the resulting principle $\AC(n)$.
  (That is, here we consider families $P : X \to \BSG_n$.)

  If we both fix $X$ and restrict to families of $n$-element sets,
  we denote the resulting principle $\lAC{X}(n)$.
\end{definition}

\begin{xca}
  Show that $\lAC{X}$ is always true whenever $X$ is a finite set.
\end{xca}

[TODO, Elaborate: For a family of $n$-element sets over a base type $X$, $P : X
  \to \BSG_n$, there is a section if and only if there is a
  ``reduction of the structure group'' to a subgroup of $\SG_n$,
  whose action on the standard $n$-element set, $\bn n$, has a fixed point.]

\begin{lemma}\label{lem:ac-impl-triv-coh-sets}
  If $\lAC{X}$ holds for a set $X$,\marginnote{%
    In fancier language, this says that the axiom of choice
    implies that all cohomology sets $\constant{H}^1(X,G)$ are trivial.}
  then $\Trunc{X \to \BG}_0$ is contractible for any group $G$.
\end{lemma}

\begin{proof}
  Suppose we have a map $f : X \to \BG$.
  We need to show that $f$ is merely equal to the constant map.
  Consider the corresponding family of sets
  consisting of the underlying sets of the $G$-torsors represented by
  $f(x) : BG$, for $x:X$.
  That is, define $P : X \to \Set$ by setting $P(x) \defeq (\shape_G = f(x))$.
  Since $\BG$ is connected, this is a family of non-empty sets,
  so by the axiom of choice for families over $X$,
  there exists a section.
  Since we're proving a proposition, let $s : \prod_{x:X}(\shape_G = f(x))$
  be a section.
  Then $s$ identifies $f$ with the constant map, as desired.
\end{proof}

We might wonder what happens if we consider general \inftygps $G$
in~\cref{lem:ac-impl-triv-coh-sets}.
Then the underlying type of a $G$-torsor is no longer a set, but can be any type.
Correspondingly, we need an even stronger version of the axiom of choice,
where the family $P$ is allowed to be arbitrary.
Let $\AC_\infty$ denote this untruncated axiom of choice,
and let $\lAC{X}_\infty$ denote let local version, fixing a set $X$.
This is connected to another principle, which is much more constructive,
yet still not true in all models.

\begin{principle}[Sets Cover]\label{pri:sc}
  For any type $A$, there exists a set $X$ together with a surjection $X \to A$.
\end{principle}

We abbreviate this as $\constant{SC}$.

\begin{xca}
  Prove that the untruncated axiom of choice, $\AC_\infty$,
  is equivalent to the conjunction of the standard axiom of choice, $\AC$,
  and the principle that sets cover, $\constant{SC}$.
\end{xca}

\begin{xca}
  Prove that we cannot relax the requirement that $X$ is a set
  in the axiom of choice.
  Specifically, prove that $\lAC{\Sc}(2)$ is false
\end{xca}

We now come to the analogue of~\cref{lem:ac-impl-triv-coh-sets}
for arbitrary \inftygps.

\begin{xca}
  Prove that if the untruncated $X$-local axiom of choice, $\lAC{X}_\infty$,
  holds for a set $X$,
  then $\Trunc{X \to \BG}_0$ is contractible for all \inftygps $G$.
\end{xca}

\begin{theorem}[Blass]\label{thm:Blass}
  Let $X$ be a set with decidable equality such that $\Trunc{X \to \BG}_0$ is contractible
  for all groups $G$.
  Then every family of non-empty sets with decidable equality over $X$
  merely admits a section,
  \ie $\lAC{X}^{\mathrm{d}}$ holds.
\end{theorem}

\begin{proof}
  First suppose $P : X \to \Set$ is such that all the sets $P(x)$
  have the same size, \ie the function $P$ factors through
  $\BAut(S)$ for some non-empty set $S$.
  This in turn means that we have a function $h : X \to \BG$,
  where $G \defeq \Aut(S)$, with $P = \prj_1 \circ h$,
  where $\prj_1 : \BAut(S) = \sum_{A : \Set}\Trunc{S \simeq A} \to \Set$
  is the projection.

  By assumption, $h$ is merely equal to the constant family.
  But since we are proving a proposition, we may assume that $h$
  \emph{is} constant, so $P$ is the constant family at $S$.
  And this has a section since $S$ is non-empty.

  [TODO: complete the proof and think about whether we can relax the
  decidability requirements]
\end{proof}

\begin{theorem}
  Let $X$ be any set. Then $\lAC{X}(4)$ follows from $\lAC{X}(2)$ and $\lAC{X}(3)$.
\end{theorem}

\begin{proof}
  Let $P : X \to \BSG_4$ be a family of $4$-element sets over $X$.
  Consider the map $\Bf : \BSG_4 \to \BSG_3$ that maps a $4$-element set
  to the $3$-element set of its $2+2$ partitions.
  Choose a section of $\Bf \circ P$ by $\lAC{X}(3)$.
  Now use $\lAC{X}(2)$ twice to choose for each chosen partition
  first one of the $2$-element parts, and secondly one of the $2$
  elements in each chosen part.
\end{proof}

In the global case where we can change the base set,
we don't need choice for $3$-element sets.
This is Tarski's result alluded to above.

\begin{theorem}
  $\AC(2)$ implies $\AC(4)$.
\end{theorem}

\begin{proof}
  Let $P: X \to \BSG_4$ be a family of $4$-element sets indexed by a set $X$.
  Consider the new set $Y$ consisting of all $2$-element subsets
  of $P(x)$, as $x$ runs over $X$,
  \[
    Y \defeq \sum_{x:X}[P(x)]^2.
  \]
  The set $Y$ carries a canonical family of $2$-element sets,
  so we may choose an element of each.
  In other words, we have chosen an element of each of the $6$
  different $2$-element subsets of each of the $4$-element sets
  $P(x)$.

  For every $a : P(x)$, let $q_x(a)$ be the number of $2$-element
  subsets $\set{a,b}$ of $P(x)$ with $b\ne a$ for which $a$ is the
  chosen element.

  Define the sets $B(x) \defeq \setof{a:P(x)}{\text{$q_x(a)$ is a
      minimum of $q_x$}}$, and remember that they are subsets of $P(x)$.
  This determines a decomposition of $X$ into three parts $X = X_1 +
  X_2 + X_3$, where
  \[
    X_i \defeq \sum_{x:X}(\text{$B(x)$ has cardinality $i$}),
    \quad i = 1,2,3.
  \]
  Note that $B(x)$ can't be all of $P(x)$,
  since that would mean that $q_x$ is constant,
  and that is impossible, since the sum of $q_x$ over the $4$-element $P(x)$ is $6$.

  Over $X_1$, we get a section of $P$ by picking the unique element
  in $B(x)$.

  Over $X_3$, we get a section of $P$ by picking the unique element
  \emph{not} in $B(x)$.

  Over $X_2$, we get a section of $P$ by picking the already chosen
  element of the $2$-element set $B(x)$.
\end{proof}

The following appears as Theorem~6 in Blass\footcite{Blass-Finite-Choice}.
\begin{theorem}
  Assume $\Trunc{X \to \BCG_n}_0$ is contractible for all sets $X$ and
  positive integers $n$. Then $\AC(n)$ holds for all $n$.
\end{theorem}

\begin{proof}
  We use well-founded induction on $n$, the case $n\jdeq 1$ being trivial.

  Let $P : X \to \BSG_n$ be a family of $n$-element sets,
  and let $Y \defeq \sum_{x:X}P(x)$ be the domain set of this \covering.
  Consider the family $Q : Y \to \BSG_{n-1}$ defined by
  \[
    Q((x,y)) \defeq \setof{y' : P(x)}{y \ne y'} = P(x) \setminus \set{y},
  \]
  where we use the fact that $P(x)$ is an $n$-element set
  and thus has decidable equality,
  so we can form the $(n-1)$-element complement $P(x) \setminus
  \set{y}$.

  By induction hypothesis, we get a section of $Q$, which we can
  express as a family of functions
  \[
    f : \prod_{x:X}\bigl(P(x) \to P(x)\bigr)
  \]
  where $f_x(y) \ne y$ for all $x,y$.
  Since $P(x)$ is an $n$-element set, we can decide whether $f_x$
  is a permutation or not, and if so, whether it is a cyclic
  permutation.
  We have thus obtained a partition $X = X_1 + X_2 + X_3$,
  where
  \begin{align*}
    X_1 &\defeq \setof{x:X}{\text{$f_x$ is not a permutation}}, \\
    X_2 &\defeq \setof{x:X}{\text{$f_x$ is a non-cyclic permutation}}, \\
    X_3 &\defeq \setof{x:X}{\text{$f_x$ is a cyclic permutation}}.
  \end{align*}
  We get a section of $P$ over $X_1$ by induction hypothesis
  by considering the family of the images of $f_x$.

  We get a section of $P$ over $X_2$ by first choosing a cycle of $f_x$
  (there are fewer then $n$ cycles because there are no $1$-cycles),
  and then choosing an element of the chosen cycle.

  We get a section of $P$ over $X_3$ by the assumption applied
  to the map $X_3 \to \BCG_n$ induced by equipping each $P(x)$ with
  the cyclic order determined by the cyclic permutation $f_x$.
\end{proof}

[TODO: State the general positive result due to
Mostowski\footcite{Mostowski-Finite-Choice}, maybe as an exercise
and give references to the negative results, due to Gauntt (unpublished).]

%%% Local Variables:
%%% mode: latex
%%% TeX-master: "book"
%%% End:


%%% Local Variables:
%%% mode: latex
%%% fill-column: 144
%%% latex-block-names: ("lemma" "theorem" "remark" "definition" "corollary" "fact" "properties" "conjecture" "proof" "question" "proposition")
%%% TeX-master: "book"
%%% End:

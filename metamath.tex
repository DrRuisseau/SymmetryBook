\chapter{Metamathematical remarks}
\label{app:metamath}

[TODO: Give precise rules for our type theory and discuss some syntactic matters:
definitional equality, more precision around normalization and canonicity, etc.]

The statement that something is not provable is a statement \emph{about},
and not \emph{in}, a theory. Proving such a statement often requires
properties of the theory that cannot be formulated nor proved in the theory itself.

One such `metaproperty' that we use here is \emph{canonicity}.
We call an element \emph{closed} if it does not contain free variables.
An example of canonicity is that every closed natural number is a \emph{numeral},
that is, either $0$ or $S(n)$ for some numeral $n$.
Another example of canonicity is that every closed element of $L\coprod R$
is either of the form $\inl{l}$ for some $l:L$ or
of the form $\inr{r}$ for some $r:R$.
A second important metaproperty of our theory is that one can compute canonical forms.

[TODO: Write more about what metamathematics is.]

\section{The Limited Principle of Omniscience}
\label{sec:LPO}

\begin{remark}\label{rem:LPO-solves-halting problem}
Recall the Limited Principle of Omniscience (LPO), \cref{LPO}:
  for any function $P:\NN\to\bn 2$,
  either there is a smallest number $n_0:\NN$ such that $P(n_0)=1$,
  or $P$ is a constant function with value $0$.
We will show that LPO is not provable in our theory.

The argument is based on the halting problem: given a Turing machine
$M$ and an input $n$, determine whether $M$ halts on $n$.
It is known that the halting problem cannot be solved by an algorithm
that can be implemented on a Turing machine.\footnote{It's commonly accepted that
  every algorithm \emph{can} be thus implemented.}

We use a few more facts from computability theory.
First, Turing machines can be enumerated. We denote the $n$\th Turing machine $M_n$,
so we can list the Turing machines in order: $M_0,M_1,\ldots$.
Secondly, there exists a function $T(e,n,k)$ such that $T(e,n,k) = 1$
if $M_e$ halts on input $n$ in at most $k$ steps, and $T(e,n,k) = 0$
otherwise. This function $T$ can be implemented in our theory.

Towards a contradiction, assume we have a closed proof $t$ of LPO in our theory.
We assume as well that $t$ does not depend on any axiom.\footnote{It is possible to weaken the notion
  of canonicity so that the argument still works even if the proof $t$ uses the Univalence Axiom.
Of course, the argument must fail if we allow $t$ to use LEM!}
It is clear that $k\mapsto T(e,n,k)$ is a constant function with value $0$
if and only if $M_e$ does not halt on input $n$. Now consider $t(k\mapsto T(e,n,k))$,
which is an element of a type of the form $L\coprod R$.

We now explain how to solve the halting problem.
Let $e$ and $n$ be arbitrary numerals.
Then $t(k\mapsto T(e,n,k))$ is a closed element of $L\coprod R$.
Hence we can compute its canonical form. If $t(k\mapsto T(e,n,k))\jdeq\inr{r}$ for some
$r:R$, then $k\mapsto T(e,n,k)$ is a constant function with value $0$,
and $M_e$ does not halt on input $n$. If $t(k\mapsto T(e,n,k))\jdeq\inl{l}$ for some
$l:L$, then $M_e$ does halt on input $n$.
Thus we have an algorithm to solve the halting problem
for all $e$ and $n$. Since this is impossible, we have refuted the assumption
that there is a closed proof $t$ of LPO in our theory.
\end{remark}

%%% Local Variables:
%%% mode: latex
%%% fill-column: 144
%%% latex-block-names: ("lemma" "theorem" "remark" "definition" "corollary" "fact" "properties" "conjecture" "proof" "question" "proposition")
%%% TeX-master: "book"
%%% End:

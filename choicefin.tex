\section{Choice for finite sets\titledagger}
\label{sec:choicefin}

This chapter is a short overview of how group theory is involved in
relating different choice principles for families of finite sets.  A
paradigm case is that if we have choice for all families of
$2$-element sets, then we have choice for all families of $4$-element
sets.%
\footnote{This is due to Tarski,
  see~\citeauthor{Jech-AC}\footnotemark{}, p.~107.}\footcitetext{Jech-AC}

The axiom of choice is a principle that we may add to our type theory
(it holds in the standard model), but there are many models where it doesn't hold.

\begin{principle}[The Axiom of Choice]\label{pri:ac}
  For every set $X$ and every family of \emph{non-empty} sets
  $P : X \to \Set_{\ne\emptyset}$,
  there merely exists an dependent function of type $\prod_{x:X}P(x)$.
  In other terms, for any set $X$ and any family of sets $P:X\to\Set$,
  we have
  \begin{equation}\label{eq:ac-impl}
    \prod_{x:X}\Trunc{P(x)} \to \Trunc[\bigg]{\prod_{x:X}P(x)}.\qedhere
  \end{equation}
\end{principle}

\begin{remark}
  We have an equivalence between the Pi-type $\prod_{x:X}P(x)$ and the
  type of sections of the projection map $\prj_1 : \sum_{x:X}P(x) \to X$,
  under which families of non-empty sets correspond to surjections between sets
  (using that $X$ is a set).
  Thus, the axiom of choice equivalently says that any surjection
  between sets admits a section.

  Because of this equivalence, we'll sometimes also call elements of the
  Pi-type \emph{sections}.
\end{remark}

The following is usually called Diaconescu's theorem\footcite{Diaconescu} or the Goodman--Myhill theorem\footcite{Goodman-Myhill}, but it was first observed in a problem in Bishop's book on constructive analysis~\footcite{Bishop}.

\begin{theorem}
  The axiom of choice implies the law of the excluded middle, \cref{pri:lem}.
\end{theorem}

\begin{proof}
  Let $P$ be a proposition, and consider the quotient map $q : \bn 2 \to \bn 2/\sim$,
  where $\sim$ is the equivalence relation on $\bn 2$ satisfying $(0 \sim 1) = P$.
  Like any quotient map, $q$ is surjective, so by the axiom of choice,
  and because our goal is a proposition,
  it has a section $s : \bn 2/\sim \to \bn 2$.
  That is, we also have $q\circ s = \id$.

  Using decidable equality in $\bn 2$, check whether $s([0])$ and $s([1])$ are equal
  or not.

  If they are, then we get the chain of identifications
  $[0] = q(s([0])) = q(s([1])) = [1]$, so $P$ holds.

  If they aren't, then assuming $P$ leads to a contradiction, meaning $\lnot P$ holds.
\end{proof}

We'll now define some restricted variants of the axiom of choice,
because our goal is to see how they relate to each other and to other principles.

\begin{definition}
  Let $\AC$ denote the full axiom of choice, as in~\cref{pri:ac}.
  If we fix the set $X$, and consider \eqref{eq:ac-impl} for arbitrary families $P:X\to\Set$, we call this the \emph{$X$-local axiom of choice}, denoted $\lAC{X}$.

  If we restrict $P$ to take values in $n$-element sets, for some $n:\NN$,
  we denote the resulting principle $\AC(n)$.
  (That is, here we consider families $P : X \to \BSG_n$.)

  If we both fix $X$ and restrict to families of $n$-element sets,
  we denote the resulting principle $\lAC{X}(n)$.
\end{definition}

\begin{xca}
  Show that $\lAC{X}$ is always true whenever $X$ is a finite set.
\end{xca}

[TODO, Elaborate: For a family of $n$-element sets over a base type $X$, $P : X
  \to \BSG_n$, there is a section if and only if there is a
  ``reduction of the structure group'' to a subgroup of $\SG_n$,
  whose action on the standard $n$-element set, $\bn n$, has a fixed point.]

\begin{lemma}\label{lem:ac-impl-triv-coh-sets}
  If $\lAC{X}$ holds for a set $X$,\marginnote{%
    In fancier language, this says that the axiom of choice
    implies that all cohomology sets $\constant{H}^1(X,G)$ are trivial.}
  then $\Trunc{X \to \BG}_0$ is contractible for any group $G$.
\end{lemma}

\begin{proof}
  Suppose we have a map $f : X \to \BG$.
  We need to show that $f$ is merely equal to the constant map.
  Consider the corresponding family of sets
  consisting of the underlying sets of the $G$-torsors represented by
  $f(x) : BG$, for $x:X$.
  That is, define $P : X \to \Set$ by setting $P(x) \defeq (\shape_G = f(x))$.
  Since $\BG$ is connected, this is a family of non-empty sets,
  so by the axiom of choice for families over $X$,
  there exists a section.
  Since we're proving a proposition, let $s : \prod_{x:X}(\shape_G = f(x))$
  be a section.
  Then $s$ identifies $f$ with the constant map, as desired.
\end{proof}

We might wonder what happens if we consider general \inftygps $G$
in~\cref{lem:ac-impl-triv-coh-sets}.
Then the underlying type of a $G$-torsor is no longer a set, but can be any type.
Correspondingly, we need an even stronger version of the axiom of choice,
where the family $P$ is allowed to be arbitrary.
Let $\AC_\infty$ denote this untruncated axiom of choice,
and let $\lAC{X}_\infty$ denote let local version, fixing a set $X$.
This is connected to another principle, which is much more constructive,
yet still not true in all models.

\begin{principle}[Sets Cover]\label{pri:sc}
  For any type $A$, there exists a set $X$ together with a surjection $X \to A$.
\end{principle}

We abbreviate this as $\constant{SC}$.

\begin{xca}
  Prove that the untruncated axiom of choice, $\AC_\infty$,
  is equivalent to the conjunction of the standard axiom of choice, $\AC$,
  and the principle that sets cover, $\constant{SC}$.
\end{xca}

\begin{xca}
  Prove that we cannot relax the requirement that $X$ is a set
  in the axiom of choice.
  Specifically, prove that $\lAC{\Sc}(2)$ is false
\end{xca}

We now come to the analogue of~\cref{lem:ac-impl-triv-coh-sets}
for arbitrary \inftygps.

\begin{xca}
  Prove that if the untruncated $X$-local axiom of choice, $\lAC{X}_\infty$,
  holds for a set $X$,
  then $\Trunc{X \to \BG}_0$ is contractible for all \inftygps $G$.
\end{xca}

\begin{theorem}[Blass]\label{thm:Blass}
  Let $X$ be a set with decidable equality such that $\Trunc{X \to \BG}_0$ is contractible
  for all groups $G$.
  Then every family of non-empty sets with decidable equality over $X$
  merely admits a section,
  \ie $\lAC{X}^{\mathrm{d}}$ holds.
\end{theorem}

\begin{proof}
  First suppose $P : X \to \Set$ is such that all the sets $P(x)$
  have the same size, \ie the function $P$ factors through
  $\BAut(S)$ for some non-empty set $S$.
  This in turn means that we have a function $h : X \to \BG$,
  where $G \defeq \Aut(S)$, with $P = \prj_1 \circ h$,
  where $\prj_1 : \BAut(S) = \sum_{A : \Set}\Trunc{S \simeq A} \to \Set$
  is the projection.

  By assumption, $h$ is merely equal to the constant family.
  But since we are proving a proposition, we may assume that $h$
  \emph{is} constant, so $P$ is the constant family at $S$.
  And this has a section since $S$ is non-empty.

  [TODO: complete the proof and think about whether we can relax the
  decidability requirements]
\end{proof}

\begin{theorem}
  Let $X$ be any set. Then $\lAC{X}(4)$ follows from $\lAC{X}(2)$ and $\lAC{X}(3)$.
\end{theorem}

\begin{proof}
  Let $P : X \to \BSG_4$ be a family of $4$-element sets over $X$.
  Consider the map $\Bf : \BSG_4 \to \BSG_3$ that maps a $4$-element set
  to the $3$-element set of its $2+2$ partitions.
  Choose a section of $\Bf \circ P$ by $\lAC{X}(3)$.
  Now use $\lAC{X}(2)$ twice to choose for each chosen partition
  first one of the $2$-element parts, and secondly one of the $2$
  elements in each chosen part.
\end{proof}

In the global case where we can change the base set,
we don't need choice for $3$-element sets.
This is Tarski's result alluded to above.

\begin{theorem}
  $\AC(2)$ implies $\AC(4)$.
\end{theorem}

\begin{proof}
  Let $P: X \to \BSG_4$ be a family of $4$-element sets indexed by a set $X$.
  Consider the new set $Y$ consisting of all $2$-element subsets
  of $P(x)$, as $x$ runs over $X$,
  \[
    Y \defeq \sum_{x:X}[P(x)]^2.
  \]
  The set $Y$ carries a canonical family of $2$-element sets,
  so we may choose an element of each.
  In other words, we have chosen an element of each of the $6$
  different $2$-element subsets of each of the $4$-element sets
  $P(x)$.

  For every $a : P(x)$, let $q_x(a)$ be the number of $2$-element
  subsets $\set{a,b}$ of $P(x)$ with $b\ne a$ for which $a$ is the
  chosen element.

  Define the sets $B(x) \defeq \setof{a:P(x)}{\text{$q_x(a)$ is a
      minimum of $q_x$}}$, and remember that they are subsets of $P(x)$.
  This determines a decomposition of $X$ into three parts $X = X_1 +
  X_2 + X_3$, where
  \[
    X_i \defeq \sum_{x:X}(\text{$B(x)$ has cardinality $i$}),
    \quad i = 1,2,3.
  \]
  Note that $B(x)$ can't be all of $P(x)$,
  since that would mean that $q_x$ is constant,
  and that is impossible, since the sum of $q_x$ over the $4$-element $P(x)$ is $6$.

  Over $X_1$, we get a section of $P$ by picking the unique element
  in $B(x)$.

  Over $X_3$, we get a section of $P$ by picking the unique element
  \emph{not} in $B(x)$.

  Over $X_2$, we get a section of $P$ by picking the already chosen
  element of the $2$-element set $B(x)$.
\end{proof}

The following appears as Theorem~6 in Blass\footcite{Blass-Finite-Choice}.
\begin{theorem}
  Assume $\Trunc{X \to \BCG_n}_0$ is contractible for all sets $X$ and
  positive integers $n$. Then $\AC(n)$ holds for all $n$.
\end{theorem}

\begin{proof}
  We use well-founded induction on $n$, the case $n\jdeq 1$ being trivial.

  Let $P : X \to \BSG_n$ be a family of $n$-element sets,
  and let $Y \defeq \sum_{x:X}P(x)$ be the domain set of this \covering.
  Consider the family $Q : Y \to \BSG_{n-1}$ defined by
  \[
    Q((x,y)) \defeq \setof{y' : P(x)}{y \ne y'} = P(x) \setminus \set{y},
  \]
  where we use the fact that $P(x)$ is an $n$-element set
  and thus has decidable equality,
  so we can form the $(n-1)$-element complement $P(x) \setminus
  \set{y}$.

  By induction hypothesis, we get a section of $Q$, which we can
  express as a family of functions
  \[
    f : \prod_{x:X}\bigl(P(x) \to P(x)\bigr)
  \]
  where $f_x(y) \ne y$ for all $x,y$.
  Since $P(x)$ is an $n$-element set, we can decide whether $f_x$
  is a permutation or not, and if so, whether it is a cyclic
  permutation.
  We have thus obtained a partition $X = X_1 + X_2 + X_3$,
  where
  \begin{align*}
    X_1 &\defeq \setof{x:X}{\text{$f_x$ is not a permutation}}, \\
    X_2 &\defeq \setof{x:X}{\text{$f_x$ is a non-cyclic permutation}}, \\
    X_3 &\defeq \setof{x:X}{\text{$f_x$ is a cyclic permutation}}.
  \end{align*}
  We get a section of $P$ over $X_1$ by induction hypothesis
  by considering the family of the images of $f_x$.

  We get a section of $P$ over $X_2$ by first choosing a cycle of $f_x$
  (there are fewer then $n$ cycles because there are no $1$-cycles),
  and then choosing an element of the chosen cycle.

  We get a section of $P$ over $X_3$ by the assumption applied
  to the map $X_3 \to \BCG_n$ induced by equipping each $P(x)$ with
  the cyclic order determined by the cyclic permutation $f_x$.
\end{proof}

[TODO: State the general positive result due to
Mostowski\footcite{Mostowski-Finite-Choice}, maybe as an exercise
and give references to the negative results, due to Gauntt (unpublished).]

%%% Local Variables:
%%% mode: latex
%%% TeX-master: "book"
%%% End:

\chapter{Subgroups}
\label{ch:subgroups}
\section{Subgroups}
\label{sec:subgroups}
In our discussion of the group $\ZZ=\aut_{S^1}(\base)$ of integers in we discovered that the ``subsymmetries'' formed a very organized structure.
For each natural number $n$ we obtained a set of subsymmetries in the identity type $\base=\base$, namely the set of all the iterates $(\Sloop^{n})^m$ where $m$ varies over the integers.
When $n$ was positive this was realized as the $n$-fold \covering of $S^1$, when $n=0$ this was given by the universal \covering.


The other extreme of the idea of ``subsymmetries'' was exposed in \cref{sec:groupssubperm} in the form of the slogan ``any symmetry is a symmetry of $\Set$''.
By this we meant that, if $G=\aut_A(a)$ is a group, we produced a monomorphism $\rho_G:\Hom(G,\aut_{\USym G}(\Set))$, \ie any symmetry of $a$ is uniquely given by a symmetry (``permutation'') of the set $\USym G\defequi (a=a)$.

For other groups the ``subsymmetries'' form more involved structures,
and there are several ways to pin them down.  If $A$ is a groupoid, singling out a group of subsymmetries of $a:A$ should be a way of picking out just some of the symmetries of $a$ in $A$ in a way so that we can compose subsymmetries compatibly.  To make a long story short; we want a group $H$ and a homomorphism $i:\Hom(H,G)$ so that $\US i:\USym H\to\USym G$ is injective.\footnote{in classical set theory it is an important difference between saying that a function is the inclusion of a subset (which is what one classically wants) and saying that it is an injection.  We'll address this in a moment, so rest assured that all is well as you read on.}  We have a name for such a setup: $i$ is a \emph{monomorphism} as laid out in different intermpretations in \cref{lem:eq-mono-cover}.

\subsection{Subgroups as monomorphisms}

The proposition $\ismono(i)$ is equivalent to saying that $\US i:\USym H\to \USym G$ is an injection (all preimages of $\USym i$ are propositions), and also to saying that $B i:\BH\to\BG$ is a \covering, and (since $\BG$ is connected), to $\isset((Bi)^{-1}(\sh_G))$.
%\newcommand{\typemono}{Mono}
\begin{definition}
  \label{def:typeofmono}
  If $G$ is a group, the \emph{type of monomorphisms into $G$} is
  $$\typemono_G\defequi\sum_{H:\typegroup}\sum_{i:\Hom(H,G)}\ismono(i).$$
  A monomorphism $(H,i,!)$ is
      \begin{enumerate}
      \item \emph{trivial}\index{trivial monomorphism} if $\BH$ is contractible (or, equivalently, if $\USym H$ is contractible),
      \item \emph{proper}\index{proper monomorphism} if $Bi$ is not an equivalence (or, equivalently, if $\USym i$ is not an equivalence).
      \end{enumerate}
\end{definition}

\begin{example}
  \label{ex:sigma2inSigma3}
   \marginnote{
     That $i:\Sigma_2\to\Sigma_3$ is a monomorphism can visualized as follows: if $\Sigma_3$ represent all symmetries of an equilateral triangle in the plane (with vertices $1$, $2$, $3$), then $i$ is represented by the inclusion of the symmetries leaving $3$ fixed; \ie reflection through the line marked with dots in the picture.
   $$\xymatrix{&3\ar@{.}[dd]&\\&&\\
    1\ar@{-}[uur]\ar@{-}[rr]&&2\ar@{-}[uul]}$$}
  Consider the  homomorphism $i:\Sigma_2\to\Sigma_3$ of permutation groups corresponding to $(?+\bn 1,\refl{\bn 3}):\fin_2\to_*\fin_3$ (sending $A$ to $A+\bn 1$).  This is a monomorphism since $\US i:\USym\Sigma_2\to\USym\Sigma_3$ is an injection.
\end{example}

\begin{example}
  \label{ex:prodinclismono}
  If $G$ and $G'$ are groups, then the first inclusion $i_1:\Hom(G,G\times G')$ is a monomorphism because $\US{i_1}:\USym G\to \USym(G\times G')$ is an injection.

  More generally, if $i:\Hom(H,G)$ is a homomorphism for which there (merely) exists a homomorphism $f:\Hom(G,H)$ such that $\id_H=fi$, then $i$ is a monomorphism.
  \end{example}

We will be interested in knowing when two monomorphisms into $G$ are identical.

\begin{lemma}
  \label{lem:setofsubgroups}
  Let $G$ be a group and $(H,i_H,!),(H',i_{H'},!):\typemono_G$ be two monomorphisms into $G$.  The identity type $(H,i_H,!)=(H',i_{H'},!)$ is equivalent to
  \marginnote{$$\xymatrix{H\ar[rr]^f_\simeq\ar[dr]_{i_H}&&H'\ar[dl]^{i_{H'}}\\
    &G&}$$}
  $$\sum_{f:\Hom(H,H')}\isEq(\US f)\times (i_{H'}=i_H f)$$ and is a proposition.
  In particular, the type $\typemono_G$ of monomorphisms into $G$ is a set.
\end{lemma}
\marginnote{If you're familiar with the set-theoretic flavor of things, you know that it is important to distinguish between subgroups and injective group homomorphisms.   
Our use of monomorphisms can be defended because two monomorphisms into $G$ are identical exactly if they differ by precomposition by an identitification.  
In set-theoretic language this corresponds to saying that a subgroup is an injective abstract homomorphism \emph{modulo} the relation forcing that precomposing with an isomorphism yields identical subgroups.  
Classical set-theory offers the luxury of having a preferred representative in every equivalence class: namely the image of the injection, type theory does not.  We only know that the type $\typemono$ is a set.}
\begin{proof}
By \cref{lem:isEq-pair=} an identity between $(H,i_H,!)$ and $(H',i_{H'},!)$ is uniquely given by an identity $p:H'=_{\typegroup}H$ such that $i_{H'}=i_H\,p$ (a proposition since $\Hom(H',G)$ is a set).
  The description of the identity type follows since by univalence and \cref{lem:eqofconntypes}, the identity type $H=H'$ is equivalent to the set 
$$\sum_{f:\Hom(H,H')}\isEq(\US f).$$  
% If $(H,i_H,!)$ is a subgroup of $G$, then

Now, $i_{H'}=i_Hf$ is equivalent to $\US i_{H'}=\US i_H \US f$, and since $\US i_H$ is an injection of sets there is at most one such function $\US f$; translating back we see that there is at most one $f$, making $\sum_{f%:\Hom(H,H')
}\isEq(\US f)\times (i_{H'}=i_H f)$ a proposition.  
% Consequently, the identity type
% $(H,i_H,!)=_{\typesubgroup_G}(H,i_H,!)$ is equivalent to the type of homomorphisms $f:\Hom(H,H)$ which are such that $!:i_H^==i_H^=f^=$ and such that $f^=$ is an equivalence (as we see in a moment this last requirement is redundant).  
% Now, since $(H,i_H,!)$ is a subgroup, $i_H^=$ is an injection of sets, which forces $!:f^==\refl{\USym H}$, which ultimately forces $f$ to be (identical to) the identity homomorphism. 
\end{proof}

\subsection{Subgroups through $G$-sets}

For many purposes it is useful to define ``subgroups'' slightly differently.
A monomorphism into $G$ is given by a pointed connected groupoid  $\BH=(\BH_\div,\pt_H)$, a function $F:\BH_\div\to\BG_\div$ whose fibers are sets (a \covering) and an identification $p_f:\sh_G=F(\sh_H)$.  There is really no need to specify that $\BH_\div$ is a groupoid: if $F:T\to \BG$ is a \covering, then $T$ is automatically a groupoid.  

On the other hand,  the type of \coverings over $\BG$ is equivalent to the type of $G$-sets: if $X:\BG\to\Set$ is a $G$-set, then the \covering is given by the first projection $\tilde X\to \BG$ where $\tilde X\defequi\sum_{y:\BG}X(y)$ and the inverse is obtained by considering the fibers of a \covering.  Furthermore, we saw in \cref{lem:conistrans} that $\tilde X$ being connected is equivalent to the condition $\istrans(X)$ of \cref{def:transitiveGset} claiming that the $G$-set $X$ is transitive. 

Hence, the type (set, really) $\typemono_G$ of monomorphisms into $G$ is equivalent to the type of pointed connected \coverings over $\BG$, which again is equivalent to the type $\typesubgroup_G$ of transitive $G$-sets $X:\BG\to\Set$ together with a point in $X(\sh_G)$.

\begin{definition}
  Let $G$ be a group then the set of \emph{subgroups of $G$} is
  $$\typesubgroup_G\defequi\sum_{X:\BG\to\Set}{\,}X(\sh_G)\times\mathrm{isTrans}(X).$$
  The preferred equivalence 
  with the set of monomorphisms into $G$ is given by the function
  \marginnote{%
   The inverse equivalence to $E$ is given by sending $(X,\pt,!)$ to the monomorphism associated with the first projection $\sum_{z:\BG}X(z)\to\BG$.
 }%
  $$E:\typemono_G\to\typesubgroup_G\qquad (H,i,!)\mapsto E(H,i,!)\defequi ((Bi)^{-1},(\sh_H,p_i),!),$$
  where the monomorphism $i:\Hom(H,G)$ is -- as always -- given by the pointed map $(Bi_\div,p_i):(\BH_\div,\sh_H)\to_*(\BG_\div,\sh_G)$; and where $(Bi)^{-1}:\BG\to\Set$ is the preimage $G$-set (since $i$ is a monomorphism)  and finally $(\sh_H,p_i):(Bi)^{-1}(y)\defequi \sum_{x:\BH}(y=Bi(x))$.
\end{definition}

\marginnote{%
  Which of the equivalent sets $\typemono_G$ and $\typesubgroup_G$ is allowed to be called ``the set of subgroups of $G$'' is, of course, a choice.  It could easily have been the other way around and we informally refer to elements in either sets as ``subgroups'' and use the given equivalence $E$ as needed.
}%
\marginnote{%
An argument for our choice can be that the identity type in $\typesubgroup_G$ seems more transparent than the one in $\typemono_G$  (``more things are equal'' in $\typemono_G$?), just as  $A\to\Prop$ gives more the intuition of picking out a subset by means of a characteristic function than what you get when considering the equivalent type of injections into $A$.}
  \begin{example}
    The monomorphism of $\Sigma_2$ into $\Sigma_3$ of \cref{ex:sigma2inSigma3} can be displayed as a subgroup of $\Sigma_3$ through 
    $$X:\fin_3\to\Set
    $$
    given by $A\mapsto\sum_{B:\fin 2}(A=B+\bn 1)$ together with a proof that $\sum_{B:\fin 2}(A=B+\bn 1)$ is a set [the identity type $(B,p)=(B',p')$ is equivalent to $\sum_{q:B=B'}(q+\bn 1)p=p'$, so $q+\bn 1=p'p^{-1}$ which pins down $q$ uniquely  -- check!]
  \end{example}
  \begin{xca}
    Given a group $G$ we defined in \cref{sec:groupssubperm} a monomorphism from $G$ to the permutation group $\aut_{\USym G}(\Set)$. Write out the corresponding subgroup of $\aut_{\USym G}(\Set)$.
  \end{xca}
\begin{example}  
  \label{ex:prodinclisGset}
  We saw in \cref{ex:prodinclismono} that the first inclusion $i_1:G\to G\times G'$ is a monomorphism.
  The corresponding $G\times G'$-set is the composite of the first projection $\mathrm{proj}_1:\BG_\div\times\BG'_\div\to \BG_\div$ followed by the principal $G$-torsor $\princ G:\BG\to\Set$.
  
  More generally, if $i:\Hom(H,G)$ and $f:\Hom(G,H)$, and $fi=\id_H$, then $(H,i,!):\typemono_G$, corresponding to the subgroup with $G$-set given by the composite of $\Bf$ with the princial $H$-torsor $\princ H$.
\end{example}



  Translating the concepts in \cref{def:typeofmono} through the equivalence $E$ we say that a subgroup $(X,\pt,!):\typesubgroup_G$ is
      \begin{enumerate}
      \item \emph{trivial}\index{trivial subgroup} if $X$ is identical to the principal $G$-torsor.
      \item \emph{proper}\index{proper subgroup} if $X(\sh_G)$ is not contractible.
      \end{enumerate}

      \begin{remark}
      \label{rem:notationsubgroup}
      A note on classical notation is in order.  
If $(X,\pt,!)$ is a subgroup corresponding to a monomorphism $(H,i,!)$ into a group $G$, tradition would permit us to relax the burden of notation and we could write ``a subgroup $i:H\subseteq G$'', or, if we didn't need the name of $i:\Hom(H,G)$, simply ``a subgroup $H\subseteq G$'' or ``a subgroup $H$ of $G$''. 
    \end{remark}

\subsection{The geometry of subgroups: some small examples}\footnote{this subsection is not touched: it needs attention}
\label{smallsubgpex}

As a teaser, and in order to get a geometric feel for the subgroups and their intricate interplay, it can be useful to have some fairly manageable examples to stare at.  
Some of the main tools for analyzing the geometry of subgroups are collected in \cref{sec:fingp} on finite groups, and we hope the reader will be intrigued by our mysterious claims and go on to study \cref{sec:fingp}.
That said, the examples we'll present are possible to muddle through by hand without any fancy machinery, but brute force is generally not an option and even for the present examples it is not something you want to show publicly.

When presenting the subgroups of a group $G$, three types are especially revealing: the set of subgroups $\typesubgroup_G(\sh_G)$, the \emph{groupoid of subgroups} $\typesubgroup(G)\defequi\sum_{y:\BG}\typesubgroup_G(y)$ and what we for now call the ``set of normal subgroups'' $\prod_{y:\BG}\typesubgroup_G(y)$.   Our local use of ``normal subgroup'' is equivalent to the official definition to come.  

The first projection $\typesubgroup(G)\to \BG$ is referred to as the \emph{\covering of subgroups}.

\footnote{Write out and fix the concrete examples (cyclic groups and $\Sigma_3$) commented out}
% \begin{remark}
% In  \cref{cha:circle} we studied the subgroups of the group of integers $G=\ZZ$ through \coverings over the circle $S^1$ (which we showed was equivalent to $B\ZZ$).
% We discovered a subgroup $n\ZZ$ for each natural number $n:\NN$ and in the groupoid $\typesubgroup({\ZZ})$ these sit as elements in separate components.  Each of these components are contractible (because addition is commutative: $\ZZ$ is an abelian group).

% In general, a component $K$ of the groupoid $\sum_{y:\BG}\typesubgroup_G(y)$ of subgroups of a group $G$ may be much more interesting. For one thing the, $K$ can contain many subgroups in the sense that the preimage of the first projection $K\to \BG$ is a set that may have many different elements; each representing a subgroup.  However, this set of subgroup will be a \emph{conjugacy class} of subgroups: the different subgroups are related by the conjugation action of $G$.  

% If $G$ is abelian this action is trivial, and $\sum_{y:\BG}\typesubgroup_G(y)$ consists of contractible components indexed over the subgroups of $G$.  Otherwise different subgroups may live in the same component of the groupoid of subgroups -- we'll see examples in a moment.

% In addition, the components will not in general be contractible, revealing the symmetries of the subgroups under the conjugation action.
% \end{remark}


% \begin{example}
%   The trivial group only has itself as a subgroup; the groupoid of subgroups and the set of normal subgroups are singletons.
% \end{example}
% \begin{example}
%   The cyclic group $C_p$ of prime order $p$ has only two subgroups, the trivial and the full subgroup itself and both are normal.  In fact, all subgroups of abelian groups are normal.  

% In general, the cyclic group $C_n$ of order $n$ has exactly one subgroup for each divisor $i$ of $n$.
% \end{example}


% \begin{example}
%   The group $C_2\times C_2$ has has no less than five subgroups: the trivial one, three subgroups that as groups (as opposed as \emph{sub}groups) are equivalent to $C_2$ and the full group $C_4$ itself.
% \end{example}
% \begin{remark}
%   The permutation group $\Sigma_3$ has four nontrivial proper subgroups.  Three conjugate subgroups isomorphic as groups to $C_2$ and one normal one which is as a group is isomorphic to $C_3$.  The component containing the copies of $C_2$ is equivalent to a circle.
% \end{remark}
\section{Images, kernels and cokernels}
\label{subsec:ker}

The set of subgroups of a group $G$ encodes much information about $G$, partially because homomorphisms between $G$ and other groups give rise to subgroups.

In \cref{ex:Cm} we studied a homomorphism from $\ZZ$ to $\Sigma_m$ defined via the pointed map $cy_m:S^1\to_*\fin_m$ given by sending $\base$ to $\bn m$ and 
$\Sloop$ to the cyclic permutation $s_m\colon\USym\Sigma_m\oldequiv(\bn m=\bn m)$, singling out the iterates of $s_m$ among all permutations.  From this we defined the group $C_m$ through a quite general process which we define in this section, namely by taking the \emph{image} of $cy_m$.

We also noted that the resulting pointed map from $S^1$ to $\B C_m$ was intimately tied up with the $m$-fold \covering $-^m:S^1\to_*S^1$ -- picking out exactly the iterates of $\Sloop^m$ -- which in our current language corresponds to a monomorphism $i_m:\Hom(\ZZ,\ZZ)$. This process is also a special case of something, namely the \emph{kernel}.

The construction of $\ZZ/m$ and the isomorphism between $\ZZ/m$ and $C_m$ is also a consequence of what we do in this section, but we'll have to wait till \footnote{findref} to say this elegantly.

\marginnote{For those familiar with the classical notion, the following summary may guide the intuition.}

 \marginnote{ If $\phi:\Hom^\abstr(\mathcal G,\mathcal G')$ is an abstract group homomorphism, the preimage $\phi^{-1}(e_G)$ is an abstract subgroup which is classically called the kernel of $\phi$.}

\marginnote{On the other hand, the cokernel is the quotient set of $\mathcal G'$ by the relation that if $g:\mathcal G$ and $g':\mathcal G'$, then $g'\sim g'\cdot\phi(g)$.}

In our setup with a group homomorphism 
$f:\Hom(G,G')$ being given by a pointed function $\Bf:\BG\to_*\BG'$, the above mentioned kernel, cokernel and image are just different aspects of the preimages  
$$(\Bf)^{-1}(z)\defequi\sum_{x:\BG}(z=\Bf(x))$$
for $z:\BG'$.  Note that all these preimages are groupoids.

The kernel will correspond to a preferred component of the preimage of $\sh_{G'}$, the cokernel will be the ($G'$-)set of components and for the image we will choose the monomorphism into $G'$ corresponding to the cokernel.  This point of view makes it clear that the image will be a subgroup of $G'$, the kernel will be a subgroup of $G$, whereas there is no particular reason for the cokernel to be more than a ($G'$-) set.
\marginnote{Even though the cokernel is in general just a $G'$-set, we will see in \cref{def:normalquotient} that in certain situations it gives rise to a group called the quotient group. }

\subsection{Kernels and cokernels}
\label{sec:kerandcoker}
\begin{definition}
  \label{def:kernel}
 We define a function
  $$\ker:\Hom(G,G')\to\typemono_G$$
  which we call the \emph{kernel}\index{kernel}.  
 If $f:\Hom(G,G')$  is a homomorphism we must specify the ingredients in  $\ker f\defequi(\Ker f,\incl_{\ker f},!):\typemono_G$.    Consider the element 
$$\sh_{\Ker f}\defequi(\sh_G,p_f):(\Bf)^{-1}(\sh_{G'})$$ (where $p_f:\sh_{G'}=\Bf(\sh_G)$ is the part of $\Bf$ claiming it is a pointed map). 
Define the \emph{kernel group}\index{kernel!group} (or most often just the kernel) 
\marginnote{There is an inherent abiguity in our notation:
  is the kernel of $f$ a group or a monomorphism into $G$?
  This is common usage and is only resolved by a typecheck.}
of $f$ to be the group $\Ker f$ defined by the pointed component  of $\sh_{\Ker f}$ in $(\Bf)^{-1}(\sh_{G'})$:
$$\Ker f\defequi \aut_{(\Bf)^{-1}(\sh_{G'})}(\sh_{\Ker f}).
$$ 
The first projection $B\ker f\to \BG$ is a \covering (the preimages are equivalent to the sets $\sum_{p:\sh_{G'}=\Bf(z)}\Trunc{\sh_{\Ker f}=(z,p)}$) giving a monomorphism
% $\kermap f$
$\incl_{\ker f}$ of $\ker f$ into $G$; together defining $\ker f=(\Ker f,\incl_{\ker f},!):\typemono_G$.
\end{definition}

Written out, the classifying type of the kernel,
$B\ker f_\div$, is $$\sum_{z:\BG}\sum_{p:\sh_{G'}=f(z)}\Trunc{\sh_{\Ker f}=(z,p)}.$$

\marginnote{\tiny{\color{blue}THIS WILL BE CHANGED} A subgroup is said to be \emph{normal}\index{normal} if it is the kernel of a surjective homomorphism. {clarify the relation between the surjective homomorphism and the subgroup}}
\begin{definition}
  \label{def:cokernel}
  Let $f:\Hom(G,G')$  be a homomorphism.
The \emph{cokernel}\index{cokernel} of $f$ is the $G'$-set
\[
  \coker f:\BG'\to\Set,\qquad \coker f(z)\defequi  \Trunc{(\Bf)^{-1}(z)}_0;
\]
defining a function of sets
$$\coker:\Hom(G,G')\to G'\text{-}\Set.$$
\marginnote{The associated $\abstr(G')$-set $\coker f(\sh_{G'})$ is also referred to as the cokernel of $f$.  }
If $f:\Hom(G,G')$ is clear from the context and displays $G$ as a subgroup of $G'$, we often write $G'/G$ for the cokernel of $f$. 
\end{definition}
\marginnote{The monomorphisms into $G'$ associated with the cokernel is the ``image'' of the next section.}
% \newcommand*{\kermap}[1]{\mathrm{in}_{\ker #1}} something wrong fix
\begin{xca}
  Show that the cokernel is transitive.
  %solution: enough to show that for all $|x,p|\in\coker(\sh_{G'})$ there is a $g:\USym G$ s.t. $g\cdot |\sh_G,p_f|= |x,p|$.  Enough to do this for $x$ being $\sh_G$, and then $g\defequi p_f^{-1}p$ will do
\end{xca}

\begin{xca}
  Given a homomorphism $f:\Hom(G,G')$, prove that
    \marginnote{Hint: consider the corresponding property of the preimage of $\Bf$.
      $$\xymatrix{L\ar[drr]^h\ar@{.>}[dr]^{k}\ar[ddr]&&\\
        &\Ker f\ar[r]_{\incl_{\ker f}}\ar[d]&G\ar[d]^f\\
        &{1}\ar[r]&\,G'.}$$}
  \begin{enumerate}
  \item $f$ is a monomorphism if and only if the kernel is trivial
  \item $f$ is an epimorphims if and only if the cokernel is contractible.
  \item if $h:\Hom(L,G)$ is a homomorphism such that $fh:\Hom(H,G')$ factors over the trivial group $1$, then there is a unique $k:\Hom(H,\Ker f)$ such that $h=\incl_{\ker f}k$. 
  \end{enumerate}
\end{xca}


The kernel, cokernel and image constructions satisfy a lot of important relations which we will review in a moment, but in our setup many of them are just complicated ways of interpreting the following fact about preimages
\begin{lemma}
  \label{lem:fibersofcomposites}
  Consider pointed functions $(f_1,p_1):(X_0,x_0)\to_*(X_1,x_1)$ and $(f_2,p_2):(X_1,x_1)\to_*(X_2,x_2)$ and the resulting functions
  $$F_1:f_1^{-1}(x_1)\to(f_2f_1)^{-1}(x_2),\qquad F_1(x,p)\defequi(x,p_2f_2p),$$
  $$F_2:(f_2f_1)^{-1}(x_2)\to f_2^{-1}(x_2),\qquad F_2(x,q)\defequi(f_1x,q)$$
  \marginnote{(here the function $\xymatrix{((x_1,p_2)=(f_1x,q))\ar[r]^-{(\overline{p,r})\mapsto p}&(x_1=f_1(x))}$ is the ``first projection'' explained in the discussion of the interpretation of pairs following \cref{def:pairtopath})}
  $$H:F_2^{-1}(x_1,p_2)%\oldequiv\sum_{(x,q)\in (f_2f_1)^{-1}(x_2)}((x_1,p_2)=(f_1x,q))\to \\
  %\sum_{x:X_0}(x_1=f_1(x))\oldequiv
  \to f_1^{-1}(x_1),\qquad H(x,q,(\overline{p,r}))\defequi (x,p))$$
  % \begin{multline*}
  %   H:F_2^{-1}(x_1,p_2)\oldequiv\sum_{(x,q)\in (f_2f_1)^{-1}(x_2)}((x_1,p_2)=(f_1x,q))\to \\
  %   \sum_{x:X_0}(x_1=f_1(x))\oldequiv f_1^{-1}(x_1),\qquad H(x,q,(\overline{p,r}))\defequi (x,p))
  % \end{multline*}
  % $$H:F_1^{-1}(x_1,p_1)%\oldequiv\sum_{(x,q)\in (f_2f_1)^{-1}(x_2)}((x_1,p_1)=(x,p_1f_2q))\to \sum_{x:X_0}(x_1=f_1(x))\oldequiv
  % \to f_1^{-1}(x_1),\qquad H(x,q,t)\defequi (x,\fst(t))$$
  \marginnote{$$\xymatrix{
      F_2^{-1}(x_1,p_2)\ar[r]^H_\simeq\ar[d]_{\fst}&f_1^{-1}(x_1)\ar[d]^{\fst}\ar[dl]_{F_1}&\\
      (f_2f_1)^{-1}(x_2)\ar[r]^{\fst}\ar[d]^{F_2}&X_0\ar[r]^{f_2f_1}\ar[d]^{f_1}&X_2\ar@{=}[d]\\
    f_2^{-1}(x_2)\ar[r]^{\fst}&X_1\ar[r]^{f_2}&X_2.}
  $$}

Then
\begin{enumerate}
\item $H$ is an equivalence with inverse
$$H^{-1}(x,q)\defequi((x,p_2f_2q),(\overline{q,\refl{p_2f_2q}})),$$ 
\item the composite $F_1H$ is definitionally equal to the first projection $\fst:{F_2^{-1}(x_1,p_2)\to(f_2f_1)^{-1}(x_2)}$. 
\end{enumerate}
Hence, through univalence, $H$ provides an identification
  $$\bar H:(F_2^{-1}(x_1,p_2),\fst)=(f_1^{-1}(x_1),F_1)$$ in the type $\sum_{X:\UU}(X\to(f_2f_1)^{-1}(x_2))$ of function with codomain $(f_2f_1)^{-1}(x_2)$.
  %($\fst(t)$ refers to the discussion following \cref{def:pairtopath} so that if $t\oldequiv(\overline{a,b}):(x_1,p_1)=(x,p_1f_2q)$, then $\fst(t)\defequi a:x_1=f_1x$).
\end{lemma}
\begin{proof}
  That $H$ is an equivalence is seen by noting that $F_2^{-1}(x_1,p_2)$ is equivalent to
  $\sum_{x:X_0}\sum_{q:x_2=f_2f_1x}\sum_{p:x_1=f_1x}q=p_2f_2p$ and that $\sum_{q:x_2=f_2f_1x}q=p_2f_2p$ is contractible.
\end{proof}
% When referring to \cref{lem:fibersofcomposites}
% it is often useful to display an
From the universal property of the preimage it furthermore follows that $F$ is the unique map such that $\fst F=_{f_1^{\-1}(x_1)\to X_0}\fst$ and $H^{-1}$ is similarly unique wrt. $\fst H^{-1}=F$.

\begin{corollary}
  \label{cor:cokermaps}
  \marginnote{$$\xymatrix{\Ker f_1\ar@{=}[r]\ar[d]^{F_1}&\Ker f_1\ar[d]^{\incl_{\ker f_1}}\\
      \Ker f_2f_1\ar[d]^{F_2}\ar[r]^{\incl_{\ker f_2f_1}}&G_0\ar[d]^{f_1}\ar[r]^{f_2f_1}&G_2\ar@{=}[d]\\
      \Ker f_2\ar[r]^{\incl_{\ker f_2}}&G_1\ar[r]^{f_2}&G_2% \\
   % \coker F_2&\coker f_1
 }
  $$}
  Consider homomorphisms $f_1:\Hom(G_0,G_1)$ and $f_2:\Hom(G_1,G_2)$.
  There is a unique monomorphisms $F_1$ from $\ker f_1$ to $\ker(f_2f_1)$ and a unique homomorphism $F_2$ from $\ker(f_2f_1)$ to $\ker f_2$ such that $\incl_{\ker f_1}=\incl_{\ker f_2f_1}F_1$ and $f_1\incl_{\ker f_2f_1}=\incl_{\ker f_2}F_2$.
  Furthermore, $$F_1=_{\typemono_{G_1}}\incl_{\ker F_2}$$ and $$(\coker f_1)\,\B\incl_{\ker f_2}=_{\B\Ker f_2\to\Set}\coker (F_2).$$

  Consequequently,
  \begin{enumerate}
  \item if $f_2$ is a monomorphism then $F_1:\Ker f_1\to\Ker f_2f_1$ is an isomorphism and 
  \item if $f_1$ is a monomorphism then $F_2:\Ker f_2f_1\to\Ker f_2$ is an isomorphism.
  \end{enumerate}
\footnote{If $f,g:A\to\Set$ are two $A$-sets, then $f\to g$ is defined to be the set $$\prod_{a:A}(f(a)\to g(a))$$ and we say that $\phi:f\to g$ is an equivalence if $\prod_{a:A}\isEq\phi(a)$.}
  Likewise, the set truncation of the maps $F_1$ and $F_2$ constructed in \cref{lem:fibersofcomposites} give maps  of families $F_1':\coker f_1\to_{\BG_1\to\Set}\coker f_2f_1\,Bf_2$ and $F_2':\coker f_2f_1\to_{\BG_2\to\Set}\coker f_2$
  such that
  \begin{enumerate}
  \item if $f_2$ is an epimorphism then $F_1':\coker f_1\to_{\BG_2\to\Set}\coker f_2f_1\,Bf_2$ is an equivalence and 
  \item if $f_1$ is an epimorphism then $F_2':\coker f_2f_1\to_{\BG_2\to\Set}\coker f_2$  is an equivalence.
  \end{enumerate}
\end{corollary}
\begin{xca}
  Let $f:\Hom(G,G')$.  Then the subgroup $E(\ker f):\typesubgroup_G$ associated with the kernel is given by a $G$-set equivalent to the one sending $x:\BG$ to
  $$\sum_{p:\sh_{G'}=\Bf(x)}\Trunc{\sum_{\beta:\sh_G=x}p=\USym f(\beta)p_f}.$$
  If $f$ is an epimorphism this is furthermore equivalent to
  $$x\mapsto(\sh_{G'}=\Bf(x)).$$
\end{xca}

\subsection{The image}
\label{sec:image}

For a function $f:A\to B$ of sets (or, more generally, of types) the notion of the ``image'' gives us a factorization through a surjection followed by an injection: noting that $a\mapsto (f(a),!)$ is a surjection from $A$ to the ``image'' $\sum_{b:B}\Trunc{f^{-1}(b)}$, from which we have an injection (first projection) to $B$.  
\marginnote{The formula for the image in group theory is the same as the one for sets, except that the propositional truncation we have for the set factorization is replaced by the set truncation present in our formulation of the cokernel $\coker(f)\defequi\Trunc{(\Bf)^{-1}(z)}_0$.}
For homomorphism $f:\Hom(G,G')$ of groups we similarly have a factorization through an epimorphism followed by a monomorphism.  The image is now represented by %type $\sum_{b:B}\Trunc{f^{-1}(b)}$ is simply replaces by
$\sum_{z\BG'}\Trunc{(\Bf)^{-1}(z)}_0$.

In other words, the image is nothing but the subgroup $E(\coker f):\typesubgroup_{G'}$ associated with the cokernel.  There is quite a lot to say about this construction and we summarize the most important features.
\begin{construction}\label{con:im}
  We define a function
  $$\img:\Hom(G,G')\to\typemono_G$$
  called the \emph{image}\index{image!function} %(or just image)
  and a function
  $$\prjim:\Hom(G,G')\to\typeepi_G
  $$
  called the \emph{projection to the image}\index{image!projection to},
  so that if  $f:\Hom(G,G')$  is a homomorphism, then 
  $\img f\defequi(\Img f,\incl_{\img f},!):\typemono_G$ and
  $\prjim f\defequi(\Img f,\prj_{\Img f},!):\typeepi_G$
  with common first projection the \emph{image group}\index{image!group} (or most often just the image) $\Img f$, and so that we have a definitional equality in $\Hom(G,G')$
  \marginnote{$$\xymatrix{G\ar[rr]^f\ar@{->>}_{\prj_{\Img f}}[dr]&&G'\\&\,\image f.\,\ar@{>->}[ur]_{\incl_{\Img f}}}
  $$}
$$f\oldequiv\incl_{\Img f}\prj_{\Img f},$$
referred to as the \emph{factorization of $f$ through its image.}

Given two homomorphisms $f_1:\Hom(G_0,G_1)$ and $f_2:\Hom(G_1,G_2)$, there is a unique homomorphism
$u:\Hom(\Img f_2f_1,\Img f_2)$ satisfying
\marginnote{$$\xymatrix{G_0\ar[rr]^{\prj_{\img f_2f_1}}\ar[d]_{f_1}&&
    \Img f_2f_1\ar[r]^-{\incl_{\img f_2f_1}}\ar[d]_{u}&G_2\ar@{=}[d]\\
G_1\ar[rr]_-{\prj_{\img f_2}}&&\Img f_2\ar[r]_{\incl_{\img f_2}}&G_2}$$}
\begin{enumerate}
\item $\incl_{\img f_2f_1}=\incl_{\img f_2}u$ and 
\item $\prj_{\img f_2}f_1=u\,prj_{\img f_2f_1}$.  
\end{enumerate}
The homomorphism $u$ is always a monomorphism and is an isomorphism if and only if  $f_1$ is an epimorphism.
% Given two homomorphisms $f_1:\Hom(G_0,G_1)$ and $f_2:\Hom(G_1,G_2)$, the homomorphism
% $\incl_{\img \prj_{\img f_2}f_1}:\Hom(\Img f_2f_1,\Img f_2)$ is a monomorphism satisfying
% \marginnote{$$\xymatrix{G_0\ar[rr]^{\prj_{\img f_2f_1}}\ar[d]_{f_1}&&
%     \Img f_2f_1\ar[dr]^-{\incl_{\img f_2f_1}}\ar[d]_{\incl_{\img{\prj_{\img f_2}f_1}}}&\\
% G_1\ar[rr]_-{\prj_{\img f_2}}&&\Img f_2\ar[r]_{\incl_{\img f_2}}&G_2}$$}
% \begin{enumerate}
% \item $\incl_{\img \prj_{\img f_2}f_1}$ is an isomorphism if and only if  $f_1$ is an epimorphism,
% \item $\incl_{\img f_2f_1}=\incl_{\img f_2}\incl_{\img \prj_{\img f_2} f_1}$ and 
% \item $\prj_{\img f_2}f_1=\incl_{\img \prj_{\img f_2} f_1}\prj_{\img f_2f_1}$.  
% \end{enumerate}

The factorization of $f:\Hom(G,G')$ through its image is unique
in the sense that if given two homomorphisms $f_1:\Hom(G_0,G_1)$ and $f_2:\Hom(G_1,G_2)$ such that $f_1$ is an epimorphism and $f_2$ in a monomorphism, 
\marginnote{%
$$\xymatrix{G_0\ar[d]_{f_1}\ar[r]^-{\prj_{\img f_2f_1}}&\Img f_2f_1\ar[d]^{\incl_{\img f_2f_1}}\\
  G_1\ar@{.>}[ur]^t\ar[r]_{f_2}&G_2}
$$
% $$\xymatrix{&\Img f_2f_1\ar[dr]^{\incl_{\img f_2f_1}}\ar@{.>}[dd]^t&\\
%     G_0\ar[dr]_{f_1}\ar[ur]^{\prj_{\img f_2f_1}}&&G_2\\
%   &G_1\ar[ur]_{f_2}&}
% $$
}
then there is a unique isomorphism $t:\Hom(\Img f_2f_1,G_1)$ such that $f_1=t\,\prj_{\img f_2f_1}$ and $f_2\,t=\incl_{\img f_2f_1}$.  Through univalence $t$ gives rise to identifications
$$f_1=_{\typeepi_G}\prj_{\img f_2f_1}\text{ and }f_2=_{\typemono_{G'}}\incl_{\img f_2f_1}.$$

\end{construction}
\begin{implementation}{con:im}
  Consider the element
  $$\sh_{\Img f}\defequi (\sh_{G'},|\sh_G,p_f|):\sum_{z:\BG'}\coker f(z)$$ and define the image group of $f$ to be
  $$\Img f\defequi \aut_{\sum_{z:\BG'}\coker f(z)}(\sh_{\Img f}).$$
  The first projection $\fst: B\image f\to \BG'$ is a \covering
  (since $\coker f(z)$ is a set) giving the desired monomorphism $\incl_{\img f}$ of $\Img f$ into $G$.

  The homomorphism 
  $\prj_{\img f}:\Hom(G,\Img f)$  is given on the level of classifying types by sending $x:\BG$ to
  $$B\prj_{\Img f} f(x)\defequi (\Bf(x),|x,\refl{\Bf(x)}|):\image f.$$
  From this it is clear that $\Bf$ is definitionally equal to the composite of $B\incl_{\Img f}$ and $B\prj_{\Img f}$.
  \marginnote{and if the wrapping destroys this fact in some ugly manner, it is an argument against wrapping}
  That $\prj_{\img f}$ is an epimorphims is best seen by using the equivalence between $BG$ and $\sum_{z:\BG'}(\Bf)^{-1}(z)$ which translates $\prj_{\img f}$ to the sum over $z:\BG'$ of the truncation $(\Bf)^{-1}(z)\to\Trunc{(\Bf)^{-1}(z)}_0\oldequiv\coker f(z)$ which has connected fiber (recall that a homomorphism is an epimorphism iff its classifying map has connected fibers).

  Assume given homomorphisms $f_1\Hom(G_0,G_1)$ and $f_2:\Hom(G_1,G_2)$.  The claimed homomorphism $u:\Hom(\Img f_2f_1,\Img f_2)$ has classifying map $Bu:\sum_{z:\BG_2}(\coker f_2f_1)(z)\to_*\sum_{z:\BG_2}(coker f_2)(z)$ given by $F_2'(z):(\coker f_2f_1)(z)\to(\coker f_2)(z)$ from \cref{cor:cokermaps} (which was the truncation of the map of preimages $F_2:(\Bf_2f_1)^{-1}(z)\to(\Bf_2)^{-1}(z)$ with $F_2(x,p)\oldequiv(f_1x,p)$).  That $\incl_{\img f_2f_1}=\incl_{\img f_2}u$ and $\prj_{\img f_2}f_1=u\,prj_{\img f_2f_1}$ follows by the definitions of the maps and the uniqueness of $u$ follows from the fact that $\incl_{\img f_2}$ is a monomorphism and the demand $\incl_{\img f_2f_1}=\incl_{\img f_2}u$ -- which also forces $u$ to be a monomorphism since $\incl_{\img f_2f_1}$ is a monomorphism.  Conversely, if $f_1$ is an epimorphism, then the composite $\prj_{\img f_2}f_1$ is an epimorphism and $\prj_{\img f_2}f_1=u\,prj_{\img f_2f_1}$ forces $u$ to be an epimorphism, and so an isomorphims.

  As for the uniqueness of the factorization, assume given $f_1\Hom(G_0,G_1)$ and $f_2:\Hom(G_1,G_2)$ such that $f_1$ is an epimorphism and $f_2$ in a monomorphism, we let $t:\Hom(G_1,\Img f_2f_1)$ be the composite $u^{-1}\prj_{\img f_2}$, where $u$ is invertible since $f_1$ is an epimorphism.  Since $f_1$ is an epimorphism, $f_2$ a monomorphism and the claimed diagram commutes this forces $t$ to be an isomorphism. % Since  $\incl_{\img f_2f_1}$ is a monomorphism, there is at most one homomorphism $t:\Hom(G_1,\Img f_2,f_1)$ such that $f_2=\incl_{\img f_2f_1}t$.  We must show that such a $t$ exist and that it furthermore satisfies  $t\,f_1=\prj_{\img f_2f_1}$.  This is achieved by setting
  % $$Bt_\div(y)\defequi (\Bf_2(y),|(\Bf_2(y),\refl{}|.
  % $$
  
\end{implementation}

% \begin{definition}
%   \label{def:image}
%  We define a function
%   $$\img:\Hom(G,G')\to\typemono_G$$
%   which we call the \emph{image}\index{image}.
%   If $f:\Hom(G,G')$  is a homomorphism we must specify the ingredients in $\img f\defequi(\Img f,\incl_{\img f},!):\typemono_G$.  
%   The \emph{image group}\index{image!group} (or most often just the image) of $f$ is the group
%   $$\Img f\defequi \aut_{\sum_{z:\BG'}\coker f(z)}(\sh_{G'},|\sh_G,p_f|).$$
%   The first projection $\fst: B\image f\to \BG'$ is a \covering
%   (since $\coker f(z)$ is a set) giving a monomorphism $\incl_{\img f}$ of $\Img f$ into $G$;
%   together defining $\img f=(\Img f,\incl_{\img f},!):\typemono_G$.  


% The \emph{induced homomorphism} $\tilde f:\Hom(G,\image f)$ is given by sending $x:\BG$ to
% $$B\tilde f(x)\defequi (\Bf(x),|x,\refl{\Bf(x)}|):\image f.$$ 
% \end{definition}

% % \begin{remark}   \label{rem:cokerasGset}   If $f:\Hom(G,G')$ we notice that the abstract group $\abstr(G')$ acts on $\coker(f)\defequi\Trunc{f^{-1}(\sh_{G'})}_0$, making the cokernel an $\abstr(G')$-set.  If we prefer to talk about a $G'$-set, we consider the cokernel as the set-family $$\BG'\to\Set,\qquad z\mapsto   \Trunc{f^{-1}(z)}_0.$$   We will see this used most frequently when considerint inclusions of subgroups: if $H$ is a subgroup of $G$, then $G/H$ is a $G$-set. \end{remark}
In view of \cref{ex:charsurinj} below, the families  
$$\mathrm{isepi},\mathrm{ismono}:\Hom(G,G')\to\Prop
$$
of propositions that a given homomorphism is an epimorphism or monomorphism have several useful interpretations (parts of the exercise have already been done).
\begin{xca}
  \label{ex:charsurinj}
  Let $f:\Hom(G,G')$ Prove that
  \begin{enumerate}
  \item the following are equivalent
    \begin{enumerate}
    \item $f$ is an epimorphism,
    \item $\USym f$ is a surjection
    \item the cokernel of $f$ is contractible,
    \item the ``inclusion of the image'' $\image_f:\Hom(\image f,G')$ is an isomorphism,
    \end{enumerate}
  \item the following are equivalent
    \begin{enumerate}
    \item $f$ is a monomorphism,
    \item $\USym f$ is an injection
\item the kernel of $f$ is trivial
\item $\Bf:\BG\to \BG'$ is a \covering.
\item the projection onto the image $\prj_{\img f}:\Hom(G,\Img f)$ is an isomorphism.
    \end{enumerate}
  \end{enumerate}
\end{xca}



% Note that if $f:\Hom(G,G')$, then the composite of the induced homomorphism $\tilde f:\Hom(G,\image f)$ with the subgroup inclusion $\image_f$ (first projection on the level of classifying types) of $\image f$ in $G'$ is $f$ by definition.  We will refer to this as the \emph{factorization of $f$ through its image}.
% \marginnote{$$\xymatrix{G\ar[rr]^f\ar@{->>}_{\tilde f}[dr]&&G'\\&\,\image f.\,\ar@{>->}[ur]_{\image_f}}
%   $$}



\begin{lemma}
  \label{lem:kerandcoker}
  \label{lem:countinggps}
  Let $f:\Hom(G,G')$ be a group homomorphism.
  The induced map $(B\prj_{\img f})^{-1}(\sh_{\Img f})\to (\Bf)^{-1}(\sh_{G'})$ gives an identification
  $$\ker\prj_{\img f}=_{\typemono_G} \ker f.$$
\end{lemma}
\begin{proof}
  Using univalence, this is a special case of \cref{cor:cokermaps} with $f_2\defequi\incl_{\img f}$ and $f_1\defequi\prj_{\img f}$. 
% This follows by univalence and connectivity since since for $x:\BG$ the first projection from the identity type $\sh_{\Img f}=(\Bf(x),|x,\refl{\Bf(x)}|)$ to $\sh_{G'}=\Bf(x)$ is an equivalence (the fibers are true propositions).
  \footnote{\color{blue}  
 Also show counting results for the finite group part somewhere.} 
\end{proof}
\begin{xca} 
  \begin{enumerate}
  \item If $f:\typemono_{G'}$, then $\ua(\prj_{\img f}):f=_{\typemono_{G'}}\incl_{\img f}$.
  \item If $f:\typeepi_G$, then $\ua(\incl_{\img f}):f=_{\typeepi_{G}}\prj_{\img f}$.
  \end{enumerate}
(True propositions suppressed).
\end{xca}


% Finally, the image factorization would have been useless were it not for the fact that it is unique:
% \begin{lemma}
%   \label{lem:uniquenessofimagefactorizationforgroups}
%   Let $G,H,G'$ be groups, let $h:\Hom(G,H)$ and $j:\Hom(H,G')$ be homomorphisms and let $!:f=j\,h$.  If $h$ is surjective there is a unique homomorphism $t:\Hom(H,\image f)$ so that $\tilde f=t\, h$ and $j$ is $t$ composed with the first projection from $\image f$ to $ G'$.
% \end{lemma}
% \begin{proof}
%   We've used that we're operating with groupoids to simplify the statement, but a similar statement follows generally by essentially the proof below if you keep track of the element in $f=j\,h$.  To simplify we drop the ``$B$''s from the notation, writing ``$f$'' instead of ``$\Bf$''.  

% That $h$ is a surjective homomorphism amounts to saying that for $y:\BH$, then the set truncation $\Trunc{h^{-1}(y)}_0$ of the preimage is contractible, and so the first projection $\mathrm{pr_1}:\sum_{y:\BH}\Trunc{h^{-1}(y)}_0\to \BH$ is an equivalence.

% For $y:\BH$, consider the map 
% $$T_y:h^{-1}(y)\to f^{-1}(j y),\qquad T_y(x,p)\defequi (x,!_xj(p))$$ where $x:\BG$, $p:y=h(x)$ and $!_xj(p):j(y)=f(x)$ is the composite of $j(p):j(y)=j\,h(x)$ and $!:j\,h=f$ (as applied to $x$).  Performing set-truncation on $T_y$ and precomposing with the inverse of the first projection, we get a map
% $$t:\BH%\sum_{y:\BH}\Trunc{h^{-1}(y)}_0
% \to\sum_{z:\BG'}\Trunc{f^{-1}(z)}_0\oldequiv B\image f,\qquad Bt(y)\defequi(jy,|T_y|(q_y))$$
% where $q_y:\Trunc{h^{-1}(y)}_0$ is the second projection of the inverse of the first projection.  The agreement of $t$ with $\tilde f$ and $j$ follows by construction.
% \end{proof}

\begin{example}
  An example from linear algebra: let $A$ be any $n\times n$-matrix with nonzero determinant and with integer entries, considered as a homomorphism $A:\Hom(\ZZ^n,\ZZ^n)$. ``Nonzero determinant'' corresponds to ``monomorphism''.  Then the cokernel of $A$ is a finite set with cardinality the absolute value of the determinant of $A$.  You should picture $A$ as a $|\det(A)|$-fold \covering of the $n$-fold torus $(S^1)^{\times n}$ by itself.

  In general, for an $m\times n$-matrix $A$, then the ``nullspace'' is given by the kernel and the ``rowspace'' is given by the image.  
\end{example}


      \section{The action on the set of subgroups}
\label{sec:actiononsub}

Not only is the type of subgroups  of $G$ a set, it is in a natural way (equivalent to the value at $\sh_G$ of) a $G$-set which we denote by the same name.  We first do the monomorphism interpretation
\begin{definition}
  If $G$ is the group, the \emph{$G$-set of monomorphisms into $G$}\index{Gset of monomorphisms into G@$G$-set of monomorphisms into $G$} $\typemono_G:\BG\to\Set$ is given by
  $$%\typesubgroup_G:\BG\to\Set,\qquad 
  \typemono_G(y)\defequi \sum_{H:\typegroup}\sum_{f:\Hom(H,G)(y)}\isset(\Bf^{-1}(\sh_G))$$
  for $y:\BG$, 
where  -- as in \cref{ex:HomHGasGset} -- 
$$\Hom(H,G)(y)\defequi\sum_{F:\BH_\div\to \BG_\div}(y=F(\sh_H))$$
is the $G$-set of homomorphisms from $H$ to $G$.
\end{definition}
\marginnote{The type of monomorphisms into $G$ is $\typemono(\sh_G)$, and as $y:\BG$ varies, the only thing that changes in $\typemono_G(y)$ is that $\BG=(\BG_\div,\sh_G)$ is replaced by $(\BG_\div,y)$.}

\begin{definition}
  \label{def:conjactonmonos}
  If $G$ is a group, then the action of $G$ on the set of monomorphisms into $G$ is called \emph{conjugation}\index{conjugation}. 

  \label{def:conjugate}
  If $(H,F,p,!):\typemono_G(\sh_G)$ is a monomorphism into $G$ and $g:\USym G$, then the monomorphisms  $(H,F,p,!),(H,F,p\,g^{-1},!):\typemono_G(\sh_G)$ are said to be \emph{conjugate}\index{conjugate}\index{conjugate}. 
\end{definition}
\begin{remark}
  \label{rem:whyconjugate}
  The term ``conjugation'' may seem confusing as the %(abstract) 
action of $g:\USym G$ on a monomorphism $(H,F,p,!):\typemono_G(\sh_G)$ (where $p:x=F(\sh_H)$) is simply $(H,F,p\,g^{-1},!)$, which does not seem much like conjugation.  
However, as we saw in \cref{ex:abstrandconj}, under the equivalence $\abstr:\Hom(H,G)\we\Hom^\abstr(\abstr(H),\abstr(G))$, the corresponding action on $\Hom^\abstr(\abstr(H),\abstr(G))$ is exactly (postcomposition with) conjugation $c^g:\abstr(G)=\abstr(G)$.  
\footnote{The same phenomenon appeared in \cref{xca:HomZGvsAdG} where we gave an equivalence between the $G$-sets $\Hom(\ZZ,G)$ and $\Ad_G$ (where the action is very visibly by conjugation).}
  \label{rem:conjactiononmonos}
\end{remark}
Summing up the remark:
\begin{lemma}
  \label{lem:conjugationabstractly}
  Under the equivalence of \cref{lem:actionsconcreteandabstract} between $G$-sets and $\abstr(G)$-sets, the $G$-set $\typemono_G$ corresponds to the $\abstr(G)$-set
$$\sum_{H:\typegroup}\sum_{\phi:\Hom^\abstr(\abstr(H),\abstr(G))}\isprop(\phi^{-1}(e_G))$$ of \emph{abstract monomorphisms}\index{abstract monomorphisms} of $\abstr(G)$, with action $g\cdot(H,\phi,!)\defequi(H,c^g\,\phi,!)$ for $g:\abstr(G)$, where  $c^g:\abstr(G)=\abstr(G)$ is conjugation as defined in \cref{ex:conjhomo}.
\end{lemma}
\begin{remark}
  \label{rem:typeofsubgpstrivifab}
  We know that a group $G$ is abelian if and only if conjugation is trivial: for all $g:\USym G$ we have $c^g=\id$, and so we get that $\typemono_G$ is a trivial $G$-set if and only if $G$ is abelian.
\end{remark}

The subgroup analog of $y\mapsto\typemono_G(y)$ is

\begin{definition}
  Let $G$ be a group and $y:\BG$, then the $G$-set of \emph{subgroups of $G$} is
  $$\typesubgroup_G:\BG\to\Set,\qquad\typesubgroup_G(y)\defequi\sum_{X:\BG\to\Set}
  % \sum_{\pt_y:X(y)}
  X(y)\times\mathrm{isTrans}(X).$$
\end{definition}
The only thing depending on $y$ in $\typesubgroup_G(y)$ is where the ``base''point is residing.

Extending the equivalence of sets we get an equivalence of $G$-sets $E:\typemono_G\to\typesubgroup_G$ via
$$E(y):\typemono_G(y)\to\typesubgroup_G(y),\qquad E(H,F,p_F,!)=(F^{-1}, (\sh_H,p_F),!)
$$
for $y:\BG$ (where $H$ is a group, $F:\BH_\div\to \BG_\div$ is a map and $p_F:y=F(\sh_H)$ an identity in $\BG$; and $F^{-1}:\BG\to\Set$ is $G$-set given by the preimages of $F$ and $(\sh_H,p_F):F^{-1}(y)\defequi \sum_{x:\BH}y=F(x)$ is the base point).  If $y$ is $\sh_G$ we follow our earlier convention of dropping it from the notation.


Since the families are equivalent (via $E$) we use $\typemono_G$ or $\typesubgroup_G$ interchangeably.

\section{Normal subgroups}
\label{sec:normal}
In the study of groups, the notion of a normal subgroup is of vital important, and as any important concept it comes in many guises, revealing the different aspects.
For now we just state the definition in the form that it is a subgroup fixed under the action of $G$ on the $G$-set of subgroups.
\begin{definition}
  \label{def:normalsubgroup}
  The set of \emph{normal subgroups}\index{normal subgroup}\index{type! of normal subgroups} is
  $$\typenormal_G\defequi\prod_{y:\BG}\typesubgroup_G(y)$$
  considered as a subset of $\typesubgroup_G$ via the injection
  $$i:\typenormal_G\to\typesubgroup_G,\qquad i(N)\defequi N(\sh_G).$$
\end{definition}

  The corresponding set of fixed point in the $G$-set of monomorphisms 
  \marginnote{Restricting the equivalence $E:\typemono_G\to\typesubgroup_G$ to the fixed sets, we get an equivalence from $\prod_{y:\BG}\typemono_G(y)$ to $\typenormal_G$}
  $$\prod_{y:\BG}\typemono_G(y)$$
will not figure as prominently, since the focus shifts naturally to an equivalent set which we have already defined, namely the kernels.
  \begin{definition}
    \label{def:setofkernels}
    If $G$ is a group, let
    $$\xymatrix{\typeepi_G\ar@{->>}[r]^{\ker}&\typekernel_G\ar@{>->}[r]^i&\typemono_G}$$\index{set! of kernels}
    be the surjection/injection factorization of the kernel function restricted to the epimorphisms from $G$.  We call the subset $\typekernel_G$ the \emph{set of kernels}.
  \end{definition}

  Our aim is twofold:
  \begin{enumerate}
  \item we want to show that $\ker:\typeepi_G\to\typekernel_G$ is an equivalence, so that knowing that a monomorphism is \emph{a} kernel is equivalent to knowing an epimorphism it is \emph{the} kernel of.
  \item we want to show that the kernels correspond via the equivalence $E$ to the fixed points under the $G$ action on the $G$-set of subgroups.
  \end{enumerate}
\marginnote{We will achieve these goals by defining a function $\nor:\typeepi_G\to\typenormal_G$ which we show is an equivalence and, furthermore, that the two functions  $i\nor,Ei\ker:\typeepi_G\to\typesubgroup_G$ are identical.  Since $i\nor$ is an injection, this forces the surjection $\ker$ to be injective too and we are done.}



% {\color{blue}To be deleted by BID june 20\tiny
%   are called \emph{kernels} with inclusion $i:\typekernel_G\to \typemono_G$ again given by $i(M)\defequi M(\sh_G)$.  We let $E^G:\typekernel_G\to\typenormal_G$ denote the equivalence given by the equivalence of $G$-sets $E:\typemono_G\to\typesubgroup_G$.
% \begin{remark}
%   From \cref{rem:typeofsubgpstrivifab} we know that the $G$-set $\typemono_G$ is a trivial $G$-set (the family is constant) if and only if $G$ is abelian.  Consequently, $G$ is abelian if and only if all ``subgroups are normal'' (\ie $i\typenormal_G\to\typesubgroup_G$ is an equivalence).
% \end{remark}
% \sususe{Subgroups through $G$-sets}
% Occasionally it is useful to define ``subgroups'' slightly differently.  As we've defined it a subgroup of a group $G$ of the form $(H,f,!)$ where $H$ is a group (pointed connected groupoid  $\BH$), $f:\BH\to_* \BG$ is a pointed map whose fibers are sets (a pointed \covering).  There is really no need to specify that $H$ is a group: if $F:T\to \BG$ is a \covering, then $T$ is automatically a groupoid.  

% On the other hand,  the type of \coverings over $\BG$ is equivalent to the type of $G$-sets: if $X:\BG\to\Set$ is a $G$-set, then the \covering is given by the first projection $\tilde X\to \BG$ where $\tilde X\defequi\sum_{y:\BG}X(y)$ and the inverse is obtained by considering the fibers of a \covering.  Furthermore, we saw in \cref{lem:conistrans} that $\tilde X$ being connected is equivalent to the condition $\istrans(X)$ of \cref{def:transitiveGset} claiming that the $G$-set $X$ is transitive. 

% Hence, the type (set, really) $\typesubgroup_G$ of subgroups of $G$ is equivalent to the type of pointed connected \coverings over $\BG$, which again is equivalent to the type $\typesubgroup_G'$ of transitive $G$-sets $X:\BG\to\Set$ together with a point in $X(\sh_G)$.  

% The family of sets $\typesubgroup_G(y)$ where we let the element $y:\BG$ vary is by the same reasoning equivalent to the family $\typesubgroup_G'(y)$ which we for reference spell out in symbols.

% \begin{definition}
%   Let $G$ be a group and $y:\BG$, then the $G$-set of \emph{subgroups' of $G$} is
%   $$\typesubgroup_G':\BG\to\Set,\qquad\typesubgroup_G'(y)\defequi\sum_{X:\BG\to\Set}\sum_{\pt_y:X(y)}\mathrm{isTrans}(X)$$ and the type of \emph{normal subgroups'} is the set of fixed points
% $$\typenormal_G'\defequi\prod_{y:\BG}\typesubgroup_G'(y).$$
% \end{definition}
% Likewise, in symbols, the above described equivalence between the families $\typesubgroup_G$ and $\typesubgroup_G'$ is provided by the map 
% $$E(y):\typesubgroup_G(y)\to\typesubgroup_G'(y),\qquad E(H,F,p_F,!)=(F^{-1}, (\sh_H,p_F),!)$$
% (where $H$ is a group, $F:\BH_\div\to \BG_\div$ is a map and $p_F:y=F(\sh_H)$ an identity in $\BG$; and $F^{-1}:\BG\to\Set$ is $G$-set given by the preimages of $F$ and $(\sh_H,p_F):F^{-1}(y)\defequi \sum_{x:\BH}y=F(x)$ is the base point).  If $y$ is $\sh_G$ we follow our earlier convention of dropping it from the notation.

% Since the families are equivalent we may use $\typesubgroup_G$ or $\typesubgroup_G'$ interchangeably.  There is, however, a little explanation needed in order to see that the type $\typenormal_G$ of normal subgroups is equivalent to $\typenormal'_G$.  We do this by using the intermediate set of surjections from $G$:
% \begin{definition}
%   \label{def:typeepi}
%   If $G$ is a group, then the \emph{set of surjections from $G$} is the set
% $$\typeepi_G\defequi\sum_{G':\typegroup}\sum_{f:\Hom(G,G')}\mathrm{issurj}(f).$$
% \end{definition}

% Note that if $f:\Hom(G,G')$ is a surjective homomorphism and $e:G'=G''$ is an identity of groups, then $(G',f,!)$ and $(G'',f',!)$ are identitified via $e$, where $f':\Hom(G,G'')$ is the homomorphism given by the composite of $f$ and the homomorphism corresponding to $e$.
% To be deleted by BID june 20}%%%%%%%%%%%%%%%%%%%%%%%%%%%%%5

For $x',y':\BG'$, recall the notation $\pathsp{y'}(x')\defequi(y'=x')$.
\begin{definition}
  \label{def:ker2}
  %If $f:\Hom(G,G')$ is a homomorphism and $x,y:\BG$, set $P^f_y(x)\defequi (f(y)=f(x))$.
  Define $$\nor:\typeepi_G\to\typenormal_G$$
  by $\nor(G',f,!)(y)\defequi(\pathsp{f(y)}f,\refl{f(y)},!)$ for $y:\BG$.
\end{definition}

\begin{lemma}
  \label{lem:diagfornormal}
  The diagram
  $$\xymatrix{
  &\typekernel_G\,\,\ar@{>->}[r]^i&\typemono_G\ar[dd]_{\simeq}^{E}\\
  \typeepi_G\ar@{->>}[ur]^{\ker}\ar[dr]_{\nor}&&\\
  &\typenormal_G\,\,\ar@{>->}[r]^i&\typesubgroup_G}
$$
commutes, where the top composite is the image factorization of the kernel and the bottom inclusion is the inclusion of fixed points.
\end{lemma}
\begin{proof}
  Following $(G',f,!):\typeepi_G$ around the top to $\typesubgroup_G$ yields the transitive $G$-set sending $y:\BG$ to the set $\sh_{G'}=f(y)$ together with the point $p_f:\sh_{G'}=f(\sh_G)$ while around the bottom we get the transitive $G$-set sending $y:\BG$ to the set $f(\sh_G)=f(y)$ together with the point $\refl{f(\sh_G)}:f(\sh_G)=f(\sh_G)$.  Hence, precomposition by $p_f$ gives the identity proving that the diagram commutes. 
\end{proof}
We will prove that both $\ker$ and $\nor$ in the diagram of \cref{lem:diagfornormal} are equivalences, leading to the desired conclusion that the equivalence $E:\typemono_G\we\typesubgroup_G$ takes the subset $\typekernel_G$ identically to $\typenormal_G$.
Actually, since $\ker:\typeepi_G\to\typekernel_G$ is a surjection, we only need to know it is an injection, and for this 
% by the uniqueness of the image factorization shown in \cref{lem:uniquenessofimagefactorizationforgroups},
it is enough to show that $\nor$ is an equivalence; we'll spell out the details.

Since it has independent interest, we take a detour via a quotient group construction of \cref{def:normalquotient} which gives us the desired inverse of $\nor$.

We start with a small, but crucial observation.
\begin{lemma}
  \label{lem:evaliseqwhennormal}
  Let $N:\typenormal_G$ be a normal subgroup with $N(y)\oldequiv (X_y,\pt_y,!)$ for $y:\BG$.
  Then for any $y,z:\BG$
  \begin{enumerate}
  \item the evaluation map
$$\mathrm{ev}_{yz}:(X_y=X_z)\to X_z(y),\qquad \mathrm{ev}_{yz}(f)=f_y(\pt_y)$$
is an equivalence and
  \item  the map $X:(y=z)\to(X_y=X_z)$ (given by induction via $X_{\refl y}\defequi\refl{X_y}$) is surjective.
  \end{enumerate}
\end{lemma}
\begin{proof}
To establish the first fact we need to do induction independently on $y:\BG$ and $z:\BG$ in $X_y(z)$ at the same time as we observe that it suffices (since $\BG$ is connected) to show that $\mathrm{ev}_{yy}$ is an equivalence.

% Induction on the index gives rise to the map $X:(y=z)\to(X_y=X_z)$ ($X_{\refl y}\defequi\refl{X_y}$) and t
The composite 
$$\mathrm{ev}_{yy}X:(y=y)\to X_yy$$ is determined by $\mathrm{ev}_{yy}X(\refl y)\oldequiv \pt_y$. 
By transitivity of $X_y$ this composite is surjective, hence $\mathrm{ev}_{yy}$ is surjective too.  

On the other hand, in  \cref{lem:evisinjwhentransitive} we used the transitivity of $X_y$ to deduce that $\mathrm{ev}_{yy}$ was injective.  Consequently $\mathrm{ev}_{yy}$ is an equivalence.  But since $\mathrm{ev}_{yy}$ is an equivalence and $\mathrm{ev}_{yy}X$ is surjective we conclude that $X$ is surjective
\end{proof}
\begin{definition}
\label{def:normalquotient}
Let $N:\typenormal_G$ be a normal subgroup with $N(y)\oldequiv (X_y,\pt_y,!)$ for $y:\BG$.  The \emph{quotient group}\index{quotient group} is 
$$G/N\defequi\aut_{G\text{-}\Set}(X_{\sh_G}).
$$
%the component of the groupoid of $G$-sets containing and pointed at $X_{\sh_G}$.  

The \emph{quotient homomorphism}\index{quotient homomorphism} is the homomorphism $q_N:\Hom(G,G/N)$  defined by $Bq_N(z)=X_z$ (strictly pointed).
By \cref{lem:evaliseqwhennormal}, $q_N$ is an epimorphism and we have defined a map
$$q:\typenormal_G\to\typeepi_G,\qquad q(N)=(G/N,q_N,!).$$
\end{definition}

\begin{remark}
It is instructive to see how the quotient homomorphism $Bq_N:\BG\to \BG/N$ is defined in the torsor interpretation of $\BG$.  If $Y\colon \BG\to\UU$ is a $G$-type we can define the quotient as
$$
Y/N:\BG\to\UU,\qquad Y/N(y)\defequi\sum_{z:\BG}Y(z)\times X_z(y).
$$
We note that in the case $\princ G(y)\defequi (\sh_G=y)$ we get
that 
$
\princ G /N(y)\defequi\sum_{z:\BG}(\sh_G=z)\times X_z(y)
$
is equivalent to $X_{\sh_G}$.  Consequently, if $Y$ is a $G$-torsor, then $Y/N$ is in the component of $X_{\sh_G}$ and we have
$$-/N:\typetorsor_G\defequi (G\text{-set})_{(\princ G)}\to (G\text{-set})_{(X_{\sh_G})}.
$$ Our quotient homomorphism $q_N:\Hom(G,G/N)$ is the composite of the equivalence $\pathsp{}^G:\BG\we\typetorsor_G$ of \cref{lem:BGbytorsor} and the quotient map $-/N$.
\end{remark}
\begin{lemma}
  \label{lem:qeq}
  The map $\nor:\typeepi_G\to\typenormal_G$ is an equivalence with inverse $q:\typenormal_G\to\typeepi_G$.
\end{lemma}
\begin{proof}
  Assume $N:\typenormal_G$ with $N(y)\defequi(X_y,\pt_y,!)$ for $y:\BG$.
  Then $\nor\, q(N):\BG\to\Set$ takes $y:\BG$ to $(\nor\, q(N))(y)\oldequiv(Y_y,\refl{X_y},!)$, where $Y_y(z)\defequi (X_y=X_z)$.
  Noting that the equivalence $\mathrm{ev}_{yz}:(X_y=X_z)\we X_z(y)$ of \cref{lem:evaliseqwhennormal} has $\mathrm{ev}_{yy}(\refl{X_y})\defequi \pt_y$ we see that univalence gives us the desired identity $\nor\, q(N)=N$.\footnote{fix so that it adhers to dogmatic language and naturality in $N$ is clear}

  Conversely, let $f:\Hom(G,G')$ be an epimorphism.
  Recall that the quotient group is $G/\nor(f)\defequi\aut_{G\text{-}\Set}(\pathsp{f(\sh_G)}f)$ and the quotient homomorphism $q_{\nor f}:\Hom(G,G/\nor f)$ is given by sending $y:\BG$ to $\pathsp{f(y)}f:\BG\to\Set$ (strictly pointed -- \ie by $\refl{\pathsp{f(\sh_G)}f}$).
  We define a homomorphism $Q:\Hom(G',G/\nor f)$ by sending $z:\BG'$ to $\pathsp{z}f$ and using the identification $\pathsp{\sh_{G'}}f=\pathsp{f(\sh_G)}f$ induced by $pf:\sh_{G'}=f(\sh_G)$ and notice the definitional equality
  $$Q\,f\oldequiv q_{\nor f}:\Hom(G,G/\nor f).$$
  
We are done if we can show that $Q$ is an isomorphism.
The preimage of the base point $\pathsp{f(\sh_G)}f$ is
$$\sum_{z:\BG'}\prod_{y:\BG}(z=f(y))=(f(\sh_G)=f(y))$$ 
which by 
\cref{lem:epifullyfaithful} is equivalent to
$$\sum_{z:\BG'}\prod_{v:\BG'}(z=v)=(f(\sh_G)=v)$$
which by \cref{lem:pathsptransportiseq} is equivalent to the contractible type $\sum_{z:\BG'}z=f(\sh_G)$.
% \marginnote{{\color{blue}\tiny DELETE BID June
%     Conversely, consider a surjective homomorphism $f:\Hom(G,G')$.  
% For $x:\BG$ and $z:\BG'$ let $Q^f_z(x)\defequi (z=f(x))$.  
% Then the quotient group $G/\nor(f)$ is the component  of the groupoid of $G$-sets containing and pointed at $Q^f_{f(pt_G)}$ and the quotient homomorphism $q_{\nor f}:\BG\to_* \BG/\nor f$ is given by  $q_{\nor f}(y)\defequi Q^f_{f(y)}$ (strictly pointed -- \ie by $\refl{Q^f_{f(\sh_G)}}$). 
% Observe that by using the identification of the basepoint $Q^f_{f(\sh_G)}$ of $\BG/\nor f$ with $Q^f_{\sh_{G'}}$ given by $p_f:\sh_{G'}=f(\sh_G)$ we have defined a homomorphism $Q^f:\Hom(G',G/\nor f$) such that 
%  $$\xymatrix{&\BG\ar[dl]_f\ar[dr]^{q_{\nor f}}&\\
%  \BG'\ar[rr]_{Q^f}&&\BG/\ker'f
% }$$
% commutes.
% We are done if we can show that $Q^f$ is an equivalence.
% The preimage of the base point $Q^f_{f(\sh_G)}$ is
% $$\sum_{z:\BG'}\prod_{y:\BG}(z=f(y))=(f(\sh_G)=f(y))$$ 
% which by 
% \cref{lem:epifullyfaithful} is equivalent to
% $$\sum_{z:\BG'}\prod_{v:\BG'}(z=v)=(f(\sh_G)=v)$$
% which by \cref{lem:pathsptransportiseq} is equivalent to the contractible type $\sum_{z:\BG'}z=f(\sh_G)$.
% }}
\end{proof}

\begin{corollary}
  \label{cor:normalisnormal}
  The kernel $\ker:\typeepi_G\to \typekernel_G$ is an equivalence of sets.
\end{corollary}
\begin{proof} Since $\nor:\typeepi_G\to\typenormal_G$ and $E:\typemono_G\to\typesubgroup_G$ are equivalences, the inclusion of fixed points $i:\typenormal\to\typesubgroup$ is an injection and the diagram
  \marginnote{the diagram in \cref{lem:diagfornormal}
    $$\xymatrix{
  &\typekernel_G\,\,\ar@{>->}[r]^i&\typemono_G\ar[dd]_{\simeq}^{E}\\
  \typeepi_G\ar@{->>}[ur]^{\ker}\ar[dr]_{\nor}^\simeq&&\\
  &\typenormal_G\,\,\ar@{>->}[r]^i&\typesubgroup_G}
$$}
in \cref{lem:diagfornormal} commutes, the surjection $\ker:\typeepi_G\to\typekernel_G$ is also an injection.
%the kernel from $\typeepi_G$ to $\typemono_G$ is an injection.   %Hence, given a monomorphism which is a kernel, the type of epimorphisms of which it is the kernel is contractible.
\end{proof}


Summing up, using the various interpretations of subgroups, we get the following list of equivalent sets all interpreting what a normal subgroup is.  
%The explicit equivalences are left out of the statements.
\begin{lemma}
  \label{lem:characterizations of normal}
  Let $G$ be a group, then the following sets are equivalent
\begin{enumerate}
\item The set $\typeepi_G$ of epimorphisms from $G$,
\item the set $\typekernel_G$ of kernels of epimorphisms from $G$,
\item the set $\typenormal_G$ of fixed points of the $G$-set $\typesubgroup_G$ (aka.~normal subgroups),
\item the set of fixed points of the $G$-set $\typemono_G$,
\item the set of fixed points of the $G$-set of abstract subgroups of $\abstr(G)$ of \cref{lem:conjugationabstractly}.
\end{enumerate}
\end{lemma}


\subsection{The associated kernel}
\label{sec:assker}

With this much effort in proving that different perspectives on the concept of ``normal subgroups'' (in particular, kernels and fixed points) are the same, it can be worthwhile to make the composite equivalence
$$\ker\,q:\typenormal_G\we\typekernel_G$$
explicit -- where the quotient group function $q:\typenormal_G\to\typeepi_G$ is the inverse of $\nor$ constructed in \cref{def:normalquotient} --  and even wite out a simplification.

Let $N:\typenormal_G$ be a normal subgroup with $N(y)\oldequiv (X_y,\pt_y,!)$ for $y:\BG$ with $X_y:\BG\to\Set$, $\pt_y:X_y(y)$ and $!:\mathrm{isTrans}(X_y)$. 
Then
$$\Ker q(N)\defequi\aut_{\sum_{x:\BG}(X_x=X_{\sh_G})}(\sh_G,\refl{X_{\sh_G}})
$$
and with the monomorphism $\incl_{\ker q(N)}:\Hom(\Ker q(N),G)$ given by the first projection from $\sum_{x:\BG}(X_x=X_{\sh_G})$ to $\BG$.
% $$(B\ker q_N)_\div\defequi\sum_{z:\BG}(X_z=X_{\sh_G})$$
% pointed at $(\sh_G,\refl{X_{\sh_G}})$ and with $i_{\ker f}:B\ker q_N\to_*\BG$ given by the first projection.

However, going the other way around the pentagon of \cref{lem:diagfornormal}, we see that $\mathrm{ass}(N)\defequi E^{-1}i(N):\typemono_G$ consists of the group
$$\mathrm{Ass}(N)\defequi\aut_{\sum_{x:\BG}X_{\sh_G}(x)}(\sh_G,\pt_{\sh_G})
$$
and the monomorphism into $G$ given by the first projection (monomorphism because $X_{\sh_G}$ has values in sets).  Since the pentagon commutes we know that $\mathrm{ass}(N)$ is the kernel of $q(N)\colon\typeepi_G$, and the identification $\mathrm{ev}:i\ker q(N)=_{\typemono_G}\mathrm{ass}(N)$ is given via \cref{lem:evaliseqwhennormal} and univalence by the equivalence
$$\mathrm{ev}_{x\,\sh_G}:(X_x=X_{\sh_G})\to X_{\sh_G}(x)% ,\qquad \mathrm{ev}_{x\,\sh_G}(f)=f_x(\pt_x)
.$$



Letting the proposition that $\mathrm{ass}(N)$ is a kernel be invisible in the notation we may summarize the above as follows:

\begin{definition}
  \label{def:associatednormal}
  If $N:\typenormal_G$ is a normal subgroup we call the kernel $\mathrm{ass}(N):\typekernel_G$ the \emph{kernel assocaited to $N$}\index{kernel!associated to}.
\end{definition}
\marginnote{\begin{remark}
    In forming the kernel associated to $N$, where did we use that $N$ was a fixed point of the $G$-set $\typesubgroup_G$?
    If $Y:\BG\to\Set$ is a transitive $G$-set and $\pt:Y(\sh_G)$, then surely we could consider the group
    $$W\defequi\aut_{X:\BG\to\Set}(Y)$$ % defined as the component of the groupoid of $G$-sets containing and pointed at $Y$
    as a substitute for the quotient group (see \cref{sec:Weyl}).
    %
    One problem is that we wouldn't know how to construct a homomorphism from $G$ to $W$ which we then could consider the kernel of.  And even if we tried our hand inventing formulae for the outcomes, ignoring all subscripts, we'd be stuck at the very end where we 
used \cref{lem:evaliseqwhennormal} to show that the evaluation map is an equivalencs; if we only had transitivity we could try to use a variant of \cref{lem:evisinjwhentransitive} to pin down injectivity, but surjectivity needs the extra induction freedom. 
\end{remark}}%marginnote


\begin{lemma}
  \label{lem:normalsarekernels}
  The diagram of equivalences
  $$\xymatrix{&\typekernel_G\,\,\ar@{>->}[r]^i&\typemono_G\ar[dd]^E_\simeq\\
\typeepi_G\ar[ur]^{\ker}_{\simeq}\ar[dr]^{\nor}_\simeq&\\
&\,\typenormal_G'\ar[uu]^{\mathrm{ass}}_\simeq\,\,\ar@{>->}[r]^i&\typesubgroup_G}$$
commutes.
\end{lemma}




%   This can be simplified somewhat:

% \begin{definition}
%   \label{def:associatednormal}
%   Let $N:\typenormal'_G$ be a normal subgroup' with $N(y)\oldequiv (X_y,\pt_y,!)$ for $y:\BG$ with $X_y:\BG\to\Set$, $\pt_y:X_y(y)$ and $!:\mathrm{isTrans}(X_y)$.  Define a subgroup $(\mathrm{ass}(N),i_N,!)$ of $G$, called the \emph{associated normal subgroup}, as follows:
%   \begin{enumerate}
%   \item the connected groupoid $B\mathrm{ass}(N)_\div\defequi\sum_{z:\BG}X_{\sh_G}(z)$,
%   \item together with the point $\pt_N\defequi(\sh_G,\pt_{\sh_G})$,
%   \item the first projection $\Bi:B\mathrm{ass}(N)\to_*\BG$
%   \item together with the assertion that the preimages of $\Bi$ (which are equivalent to $X_{\sh_G}(z)$ for varying $z:\BG$) are sets. 
%   \end{enumerate}
% \end{definition}


% We obviously need to show that the ``normal'' in the name is warranted.
% \begin{lemma}
%   \label{lem:normalsarekernels}
% The map
% $$\mathrm{ev}:B\ker q_N\to B\mathrm{ass}(N),\qquad \mathrm{ev}(z,f)\defequi(z,f(\pt_z))$$ is an equivalence and furthermore commutes with the first projections to $\BG$.  The pointed map $\mathrm{ev}_*\defequi(\mathrm{ev},\refl{(\sh_G,\pt_{\sh_G})}):B\ker q_N\to_* B\mathrm{ass}(N)$
% (well defined since $\mathrm{ev}(\sh_G,\refl{X_{\sh_G}})\defequi (\sh_G,\pt_{\sh_G})$) induces an identification of the subgroups $\ker\,q(N)$ and $\mathrm{ass}(N)$.  

% Provided with this information, the associated normal subgroup $\mathrm{ass}(N)$ \emph{is} a normal subgroup and we get a commuting diagram
% $$\xymatrix{&\typenormal_G\\
% \typeepi_G\ar[ur]^{\ker}_{\simeq}\ar[dr]^{\ker'}_\simeq&\\
% &\,\typenormal_G'.\ar[uu]^{\mathrm{ass}}_\simeq}$$
  
%   % The associated normal subgroup of $N:\typenormal_G'$ is a normal subgroup; more precisely $N(\sh_G)$ is the kernel of the quotient homomorphism $q_N:\Hom(G,G/N)$.
%   % Let $N:\typenormal'_G$ be a normal subgroup' with $N(y)\defequi (X_y,\pt_y,!)$ for $y:\BG$ with $X_y:\BG\to\Set$, $\pt_y:X_y(y)$ and $!:\mathrm{isTrans}(X_y)$.  Define a subgroup $(\mathrm{ass}(N),i_N,!)$ of $G$ as follows:
%   % \begin{enumerate}
%   % \item the connected groupoid $B\mathrm{ass}(N)_\div\defequi\sum_{z:\BG}X_{\sh_G}(z)$,
%   % \item together with the point $\pt_N\defequi(\sh_G,\pt_{\sh_G})$,
%   % \item the first projection $\Bi:B\mathrm{ass}(N)\to_*\BG$
%   % \item together with the assertion that the preimages of $\Bi$ (which are equivalent to $X_{\sh_G}(z)$ for varying $z:\BG$) are sets. 
%   % \end{enumerate}
%   % Then $(\mathrm{ass}(N),i_N,!)$ is a normal subgroup of $G$ in the sense that it is the kernel of a homomorphism.
% \end{lemma}
% \begin{proof}
%   % Let $N(y)\oldequiv (X_y,\pt_y,!)$ for $y:\BG$.  Let $G/N$ be the group defined as the component of the groupoid of $G$-sets containing and pointed in $X_{\sh_G}$.  Let $f:\Hom(G,G/N)$ be the homomorphism defined by $\Bf(z)=X_z$.
%  %  Consider the kernel of the quotient homomorphism $q_N:\Hom(G,G/N)$ of \cref{def:normalquotient},
% % $$(B\ker q_N)_\div\defequi\sum_{z:\BG}(X_z=X_{\sh_G})$$
% % pointed in $(\sh_G,\refl{X_{\sh_G}})$ and with $i_{\ker f}:B\ker q_N\to_*\BG$ given by the first projection.  Consider t

% % We claim that $\mathrm{ev}$ is an equivalence.
% Compared with the proof of $\ker'$ being an equivalence (\cref{lem:qeq}) there are no new ingredients.
% Since $B\mathrm{ass}(N)$ is connected it is enough to show that the preimage of $\mathrm{ev}^{-1}(\sh_G,\pt_{\sh_G})$ is contractible.  
% Since $\mathrm{ev}$ agrees with the projections to $\BG$, the preimage is equivalent to $\sum_{f:X_{\sh_G}=X_{\sh_G}}f(\pt_{\sh_G})=\pt_{\sh_G}$.  We recognize this as the preimage $\mathrm{ev}_{{\sh_G}{\sh_G}}^{-1}(\pt_{\sh_G})$
% of the evaluation map 
% $\mathrm{ev}_{{\sh_G}{\sh_G}}:(X_{\sh_G}=X_{\sh_G})\to X_{\sh_G}(\sh_G)$ which is an equivalence by \cref{lem:evaliseqwhennormal}.  
% % the preimage $\mathrm{ev}_{X_{\sh_G}}^{-1}(\pt_{\sh_G})$
% % of the evaluation map 
% % $\mathrm{ev}_{X_{\sh_G}}:(X_{\sh_G}=X_{\sh_G})\to X_{\sh_G}(\sh_G)$.  
% % In \cref{lem:evisinjwhentransitive} %{lem:conistrans} 
% % we proved that (since $X_{\sh_G}$ is transitive) $\mathrm{ev}_{X_{\sh_G}}$ is an injection.  Hence the preimage is a proposition, but since it contains $(\refl{X_{\sh_G}},\refl{\pt_{\sh_G}})$ it is contractible. 

% Evoking univalence we get an identification of subgroups between the kernel of $f$ and $(\mathrm{ass}(N),i_N,!)$.
% \end{proof}




% The associated normal subgroup defines an equivalence from $\typenormal_G'$ to the type $\typenormal_G$ of kernels of surjective homomorphism. To see this we construct an inverse.

% \begin{definition}
%   \label{def:kerneltofixedpoint}
%   Let $f:\Hom(G,G')$ be a surjective homomorphism.  For $y,z:\BG$ consider the set $X_y^f(z)\defequi (f(z)=f(y))$ and the element $\pt^f_y\defequi\refl{f(y)}:X_y(y)$.  The $G$-set $X_y:\BG\to\Set$ is transitive since $f$ is surjective and so we have a map 
% $$\mathrm{ssa}:\typenormal_G\to\typenormal_G',\qquad \mathrm{ssa}(f)(y)\defequi(X_y^f,\pt_y^f,!).$$
% \end{definition}

% \begin{lemma}
%   \label{lem:characterizations of normal}
%   Let $G$ be a group, then the associated normal subgroup is an equivalence
%   $$\mathrm{ass}:\typenormal_G'\to\typenormal_G$$
% with inverse $\mathrm{ssa}$.  Summing up the following sets are equivalent
% \begin{enumerate}
% \item The set $\typenormal_G$ of kernels of surjections from $G$,
% \item the set $\typenormal_G'$ of fixed points of the $G$-set $\typesubgroup_G'$,
% \item the set of fixed points of the $G$-set $\typesubgroup_G$
% \item the set of fixed points of the $G$-set of abstract subgroups of $\abstr(G)$.
% \end{enumerate}
% \end{lemma}
% \begin{proof}
%   The last three entries are equivalent since they are the fixed points of equivalent $G$-sets, so we only need to comment on the first assertion.

% Let $f:\Hom(G,G')$ be a surjective homomorphism.  
% The kernel $N\oldequiv \ker f$ is then given by the first projection 
% $$\text{pr}:\sum_{z:\BG}\sh_{G'}=\Bf(z)\to_*\BG.$$
% Then $\mathrm{ass\,ssa}(\ker f)$ defined to be
% $$
% (\sum_{z:\BG}\prod_{y:\BG}
% (\pathsp{f(\sh_G)}^{G'}{f(y)}=\pathsp{f(z)}^{G'}{f(y)},
% \text{pr}, (\sh_G,\refl{\pathsp{f(\sh_G)}^{G'}{f(\sh_G)}},!),$$
% where $\pathsp{a}^{G'}{b}\defequi (a=b)$ for $a,b:\BG'$.
% The desired identification between $\ker f$ and $\mathrm{ass\,ssa}(\ker f)$ is then given by composing the identifications
% $\preinv(p_f):(\sh_{G'}=f(z))=(f(\sh_G)=f(z))$ and 
% $$\preinv:(f(\sh_G)=f(z))=
% \prod_{y:\BG}(\pathsp{f(\sh_G)}^{G'}{f(y)}=\pathsp{f(z)}^{G'}{f(y)})$$
% \footnote{((find ref where this was demonstrated: slight modification since we only claim naturality in $G$)).}
% \end{proof}



% There are many valuable constructions to be extracted from the proof of \cref{lem:normalsarekernels}.
% \begin{definition}
%   \label{def:associatedquotient}
% \end{definition}
% \footnote{COMEBACK 190509 Do the converse and elevate constructions to definitions}



\section{The pullback}
\label{sec:pullback}

\begin{definition}
  \label{def:pullback}
  Let $B, C, D$ be types and let $f:B\to D$ and $g:C\to D$ be two maps.  
The \emph{pullback}\index{pullback} of $f$ and $g$ is the type 
$$\prod(f,g)\defequi\sum_{(b,c):B\times C}(f(b)=_Dg(c))$$
together with the two projections $\prod(f,g)\to B$ and $\prod(f,g)\to C$ sending $(b,c,p):\prod(f,g)$ to $b:B$ or $c:C$.  If $f$ and $g$ are clear from the context, we may write $B\times_DC$ instead of $\prod(f,g)$ and summarize the situation by the diagram
$$\xymatrix{B\times_DC\ar[r]\ar[d]&C\ar[d]^g\\B\ar[r]^f&\,D.}$$
\end{definition}
\begin{xca}
  \label{xca:univpropofpullback}
  Let $f:B\to D$ and $g:C\to D$ be two maps with common target.  If $A$ is a type show that 
  \begin{align*}
    (A\to B)\times_{(A\to D)}(A\to C)\to &(A\to B\times_DC)\\ 
(\beta,\gamma,p:f\beta=g\gamma)\,\mapsto\,&(a\mapsto (f(a),g(a),p(a):f\beta(a)=g\gamma(a)))
  \end{align*}
 is an equivalence.
\end{xca}

\begin{example}
  If $g:\bn 1\to D$ has value $d:D$ and $f:B\to D$ is any map, then $\prod(f,g)\oldequiv B\times_D\bn 1$ is equivalent to the preimage $f^{-1}(d)\defequi\sum_{b:B}d=f(b)$.
\end{example}
\begin{example}
  \label{ex:pullbackandgcd}
  Much group theory is hidden in the pullback.  For instance, the greatest common divisor $\gcd(a,b)$ of $a,b:\NN$ is another name for the number of components you get if you pull back the $a$-fold and the $b$-fold cover of the circle: as we will see in \cref{lem:iso2} we have a pullback
$$\xymatrix{S^1\times C_{\gcd(a,b)}\ar[d]\ar[r]& S^1\ar[d]^{(-)^b}\\
S^1\ar[r]^{(-)^a}&\,S^1}
$$ 
(where $C_n$ was the cyclic group of order $n$).
To get a geometric idea, think of the circle as the unit circle in the complex numbers so that the $a$-fold cover is simply taking the $a$-fold power.  With this setup, the pullback should consist of pairs $(z_1,z_2)$ of unit length complex numbers with the property that $z_1^a=z_2^b$.  Let $a=a'G$ and $b=b'G$ where $G=\gcd(a,b)$. Taking an arbitrary unit length complex number $z$, then the pair $(z^{b'},z^{a'})$ is in the pull back (since $a'b=ab'$).  But so is $(\zeta z^{b'},z^{a'})$, where $\zeta$ is any $G$-th root of unity.  Each of the $G$-choices of $\zeta$ contributes in this way to a component of the pullback.  In more detail: identifying the cyclic group $C_G$ of order $G$ with the group of $g$-th roots of unity, the top horizontal map $S^1\times C_G\to S^1$ sends $(z,\zeta)$ to $z^{a'}$ and the left vertical map sends $(z,\zeta)$ to the product $\zeta z^{b'}$.  

Also the least common multiple is hidden in the pullback; in the present example it is demonstrated that the map(s) accross the diagram makes each component of the pullback a copy of the subgroup $a'b\ZZ$ of $\ZZ$.
\end{example}


\begin{definition}
  \label{def:intersectionand unionofsets}
  Let $S$ be a set and consider two subsets $A$ and $B$ of $S$ given by two families of propositions (for $s:S$) $P(s)$ and $Q(s)$.  The \emph{intersection}\index{intersection! of sets} $A\cap B$ of the two subsets is given by the family of propositions $P(s)\times Q(s)$.  The \emph{union}\index{union of sets} $A\cup B$ is given by the set family of propositions $A(s)+B(s)$.  
\end{definition}
\begin{xca}
  \label{xca:intersectionpullbackofsets}
  Given two subsets $A$, $B$ of a set $S$, prove that
  \begin{enumerate}
  \item The pullback $A\times_SB$ maps by an equivalence to the intersection $A\cap B$,
  \item\label{xca:cardinalityintersectionunion} 
    If $S$ is finite, then the sum of the cardinalities of $A$ and $B$ is equal to the sum of the cardinalities of $A\cup B$ and $A\cap B$.
  \end{enumerate}
\end{xca}

\begin{definition}
  \label{def:intersectionofgroups}
  Let $f:\Hom(H,G)$ and $f':\Hom(H',G)$ be two homomorphisms with common target.  The \emph{pullback}\index{pullback!of groups} $H\times_GH'$ is the group obtained as the (pointed) component of 
$$\pt_{H\times_GH'}\defequi(\sh_H,\pt_{H'},p_{f'}p_f^{-1})$$ of the pullback $\BH\times_{\BG}\BH'$ (where $p_f:\sh_G=f(\sh_H)$ is the name we chose for the data displaying $f$ as a pointed map, so that $p_{f'}p_f^{-1}:f(\sh_H)=f'(\pt_{H'})$).

If $(H,f,!)$ and $(H',f',!)$ are monomorphisms into $G$, then the pullback is called the \emph{intersection}\index{intersection! of monomorphisms} and if the context is clear denoted simply $H\cap H'$.
\end{definition}
\begin{example}
  If $a,b:\NN$ are natural number with least common multiple $L$, then $L\ZZ$ is the intesection $a\ZZ\cap b\ZZ$ of the subgroups $a\ZZ$ and $b\ZZ$ of $\ZZ$. 
\end{example}
% \begin{example}this came out wrong DELETE June
%   If $H,K:\typemono_G$ with $X,Y:\BG\to\Set$ being the corresponding transitive $G$-sets under the equivalence $E$, then the intersection of $H$ and $K$ corresponds to the $G$-set $X\times Y:\BG\to\Set$ (with $(X\times Y)(x)\defequi X(x)\times Y(x)$).
% \end{example}

\begin{xca}
  Prove that if $f:\Hom(H,G)$ and $f':\Hom(H',G)$ are homomorphisms, then the pointed version of \cref{xca:univpropofpullback} induces an equivalence
$$\USym H\times_{\USym G}\USym {H'}
\simeq (\pt_{H\times_GH'}=\pt_{H\times_GH'})
$$
(hint: set $A\defequi S^1$, $B\defequi \BH$, $C\defequi \BH'$ and $D\defequi \BG$).  Elevate this equivalence to a statement about abstract groups.
\end{xca}

\begin{xca}
  If $\mathcal G$ is an abstract group and $\mathcal H$ and $\mathcal K$ are abstract subgroups.  Give a definition of the intersection $\mathcal H\cap\mathcal K$ is the abstract subgroup of $\mathcal G$ agreeing with our definition for groups.
\end{xca}
\begin{lemma}
  \label{lem:whatSylow2needs}
  Let $(G',f,!):\typeepi_G$, $N$ be the kernel of $f$  and let $(H,i,!):\typemono_G$.  Then
  %\begin{enumerate}
  %\item 
$N\cap H$ is the kernel of $fi:\Hom(H,G')$.  %a normal subgroup of $H$
%  \item The 
and the induced homomorphism in $\Hom(H/(N\cap H),G')$ is a monomorphism.
  % \item If $H$ and $G'$ are finite with coprime cardinalities, then $H$ is a subgroup of $N$.
%  \end{enumerate}
  \begin{proof}
Now, $N$ is the kernel of the epimorphism $f$, giving an equivalence between $\BN_\div$ and the preimage 
$$(\Bf)^{-1}(\sh_{G'})\defequi\sum_{y:\BG}(\sh_{G'}=\Bf(y)).$$  
Writing out the definition of the pullback (and using that for each $x:\BH$ the type $\sum_{y:\BG}y=\Bi(x)$ is contractible), we get an equivalence between $\BN\times_{\BG}\BH$ and 
$$B(fi)^{-1}(\sh_{G'})\defequi\sum_{x:\BH}\sh_{G'}=B(fi)x,$$  
the preimage of $\sh_{G'}$ of the composite $B(fi):\BH\to \BG'$.
 By definition, the intersection $B(N\cap H)$ is the pointed component of the pullback containing $(\pt_N,\sh_H)$.  Under the equivalence with $B(fi)^{-1}(\sh_{G'})$ the intersection corresponds to the component of $(\sh_H,\Bf(p_i)\,p_f)% :\sum_{x:\BH}\sh_{G'}=B(fi)x
 $.  
Since (by definition of the composite of pointed maps) $p_{fi}\defequi \Bf(p_i)\,p_f$ we get that the intersection $N\cap H$ is identified with the kernel of the composite $fi:\Hom(H,G')$.
%    \item 

Finally, since $N\cap H$ is the kernel of the composite $fi:\Hom(H,G')$, under the equivalence of \cref{lem:countinggps}, $N\cap H$ is equivalent to the kernel of the epimorphism $\prj_{\img(fi)}:\Hom(H,\image (fi))$.  Otherwise said, the quotient group $H/(N\cap H)$ is another name for the image $\Img (fi)$, and $\incl_{\img(fi)}$ is indeed a monomorphism into $G'$.
%    \end{enumerate}
  \end{proof}
\end{lemma}
\begin{xca}
  Write out all the above in terms of the set $\typesubgroup_G$ of subgroups of $G$ instead of in terms of the set $\typemono_G$ of monomorphism into $G$.
\end{xca}


\marginnote{Is the below misplaced?}
Recall that if $X:\BG\to\Set$ is a $G$-set, then the set of fixed points is the set $\prod_{v:\BG}X(v)$, which is a subset of $X(\sh_G)$ via the evaluation map.  If a homomorphism from a group $H$ to $G$ is given by $F:\BH_\div\to \BG_\div$ and $p_F:\sh_G=F(\sh_H)$, then precomposition (``restriction of scalars'') by $F$ gives an $H$-set 
$$F^*X\defequi X\,F:\BH\to\Set.$$  
In the case of inclusions of subgroups (or other situations where the homomorphism is clear from the context) it is not uncommon to talk about ``the $H$-set $X$'' rather than ``$F^*X$''.  
This can be somewhat confusing when it comes to fixed points: the fixed points of $F^*X$ are given by $\prod_{v:\BH}XF(v)$ which evaluates nicely to $XF(\sh_H)$, but in order to considered  these as elements in $X(\sh_G)$ we need to apply $X(p_F^{-1}):X(F(\sh_H))=X(\sh_G)$.  

Consequently, we'll say that $x:X(\sh_G)$ is an \emph{$H$-fixed point} if there is an $f:\prod_{v:\BH}XF(v)$ such that $x=X(p_F^{-1})f(\sh_H)$.



\begin{lemma}
  \label{lem:thereisaconjugate}
  Let $G$ be a group, $X:\BG\to Set$ a $G$-set, $x:X(\sh_G)$, $g:\USym G$ and $H=(H,F,p,!):\typesubgroup_G$ a subgroup of $G$ ($F:\BH_\div\to \BG_\div$ and $p:\sh_G=F(\sh_H)$).  

Then $g\,x$ is a fixed point for the $H$-action on $X$ if and only if $x$ is a fixed point for the action  of the conjugate subgroup $g\,H\defequi(H,F,g^{-1}p_F,!)$ on $X$.
\end{lemma}
\begin{proof}
  Consider an $f:\prod_{v:\BH}XF(v)$.  Then $g\cdot x=X(p_F^{-1})(f(\sh_H))$ if and only if $x=g^{-1}\cdot X(p_F^{-1})(f(\sh_H))\oldequiv X((g^{-1}p_F)^{-1}(f(\sh_H))$.
\end{proof}



% {\color{blue}Remove when lemma above gets stable \begin{lemma}
%   \label{lem:thereisaconjugate}
%   Let $G$ be a group, $X$ a $G$-set, $x:X$ and $H=(H,i,!)$ a subgroup of $G$.  If $y$ is an element in the orbit of $x$ s.t. $H\subseteq Stab_y$  (\ie $y$ is an $H$-fixed point), then there is a conjugate $H'=(H',i',!)$ of $H$ with $H'\subseteq Stab_x$.
% \end{lemma}
% \begin{proof}
%   CLASSICAL: There is a $g:G$ s.t. $y=g\cdot x$ and for all $h:H$ we have $h\cdot y=y$.  Define $H'=g^{-1}Hg$.  If $h':H'$, then $h'=g^{-1}hg$ for a unique $h:H$ and
% $$h'\cdot x=(g^{-1}hg)\cdot x = g^{-1}\cdot(h\cdot (g\cdot x))=g^{-1}\cdot(h\cdot y)=g^{-1}\cdot y = x.$$
% \end{proof}
% }

\section{The Weyl group}
\label{sec:Weyl}

In \cref{def:normalquotient} defined the quotient group of a normal subgroup.
As commented in \cref{def:associatednormal}, the definition itself never used that the subgroup was normal (but the quotient homomorphism did) and is important in this more general context.

Recall the equivalence $E$ between the set $\typemono_G$ of monomorphisms and the set $\typesubgroup_G$ of of subgroups of $G$ (pointed transitive $G$-sets): The subgroup $(X,\pt_X,!):\typesubgroup_G$ where $X:\BG\to\Set$ is a transitive $G$-set and $\pt_X:X(\sh_G)$ corresponds to $(H,i_H,!):\typemono_G$ defined by
$$H\defequi\aut_{\sum_{y:\BG}X(y)}(\sh_G,\pt_X)
$$ together with the first projection from $\sum_{y:\BG}X(y)$ to $\BG$.  Conversely, if $(H,i_H,!):\typemono_G$, then the corresponding transitive $G$-set is $G/H\defequi\coker i_H$ pointed at $|\sh_H,p_{i_H}|:\coker i_H(\sh_G)\defequi\Trunc{\sum_{x:\BH}\sh_G=\Bi_H(x)}_0$.  

For the remainder of the section we'll consider a fixed group $G$, monomorphism $i_H:\Hom(H,G)$ and $(X,\pt_X,!)$ will be the associated pointed transitive $G$-set.
\begin{definition}
  % Let $G$ be a group and consider the subgroup represented by the transitive $G$-set $X:\BG\to\Set$ together with the point $\pt_X:X(\sh_G)$ % (so that $\BH_\div\defequi\sum_{y:\BG}X(y)$ pointed at $\sh_H\defequi (\sh_G,\pt_X)$ together with the first projection $\Bi_H:\BH\to_*\BG$ defines an element in $(H,i_H,!):\typesubgroup_G$)
  % .
  
The \emph{Weyl group} \label{def:Weyl}\index{Weyl group}
$$W_GH\defequi\Aut_{G\text{-set}}(X)$$ is defined by the component $\BW_GH$ of the groupoid of $G$-sets pointed at $X$.

The \emph{normalizer subgroup} \label{def:normalizer}\index{normalizer}
$$N_GH\defequi\Aut_{\sum_{y:\BG}\typesubgroup_G(y)}(\sh_G,X,\pt_X)$$ is defined by the component $\BN_GH$ of the groupoid $\sum_{y:\BG}\typesubgroup_G(y)$ pointed at $(\sh_G,X,\pt_X)$.
\end{definition}

Unpacking, we find that
$$\BN_GH_\div\oldequiv \sum_{y:\BG}\sum_{Y:\BG\to\Set}\sum_{\pt^y_Y:Y(y)} \Trunc{(\sh_G,X,\pt_X)=(y,Y,\pt^y_Y)}.$$
While the projection $((\sh_G,X,\pt_X)=(y,Y,\pt^y_Y))\to (X=Y)$ may not be an equivalence, the transitivity of $X$ tells us that for any $\beta:X=Y$ there is a $g:\sh_G=y$ such that $X(g)\,p^y_Y=\beta^{-1}_y\pt_X$, and so the propositional truncation $\Trunc{(\sh_G,X,\pt_X)=(y,Y,\pt^y_Y)} \to \Trunc{X=Y}$ is an equivalence.
Consequently, the projection
$$\BN_GH_\div\to \sum_{y:\BG}\sum_{Y:\BG\to\Set}Y(y)\times\Trunc{X=Y}$$
is an equivalence.  With an innocent rewriting, we see that we have provided an equivalence 
$$e:\BN_GH_\div\we\sum_{(y\times Y):\BG\times \BW_GH}Y(y)\qquad e(y,Y,\pt^y_Y,!)\defequi (y,Y,\pt^y_Y,!).$$
This formulation has the benefit of simplifying the analysis of the monomorphism 
$$i_{N_GH}:\Hom(N_GH,G)$$
given by $\Bi_{N_GH}(y,Y,\pt^y_Y,!)\defequi y$, the ``projection''
 $$p_G^H:\Hom(N_GH,W_GH)$$
$\Bp_G^H(y,Y,\pt^y_Y,!)\defequi (Y,!)$ and the monomorphism
$$j_H:\Hom(H,N_GH)$$
given by $\Bj_H(y,v)\defequi(y,X,v,!))$.


% \begin{definition}
% The projections from $N_GH$ to $\BG$ and $\BW_GH$ are referred to as the inclusion $i_G^H:\Hom(N_GH,G)$ and the projection $p_G^H:\Hom(N_GH,W_GH)$.
% Define $j_H:\Hom(H,N_GH)$ by the pointed map 
% $$\Bj_H:\sum_{y:\BG}X(y)\to_*\sum_{(y,Y):\BG\times \BW_GH}Y(y),\qquad \Bi_H(y,v)\defequi ((y,X),v).$$
% \end{definition}

% The \emph{normalizer subgroup} $N_GH$ is alternatively defined as
% $$N_GH\defequi(\sum_{(y,Y):\BG\times \BW_GH}Y(y),((\sh_G,X),\pt_X)).$$

\begin{lemma}
  The monomorphism $i_G^H:\Hom(N_GH,G)$ displays the normalizer as a subgroup of $G$ and the projection $p_G^H:\Hom(N_GH,W_GH)$ is an epimorphism.  

The homomorphism $j_H:\Hom(H,N_GH)$ defines $H$ as a normal subgroup of the normalizer,
$$\ker p_G^H=_{\typemono_{N_GH}for}(H,i_H,!)$$
and $i_H=_{\Hom(H,G)}i_G^H\,j_H$.
\end{lemma}
\begin{proof}
  Immediate from (our rewriting of) the definitions.
\end{proof}

The Weyl group $W_GH$ has an important interpretation.  It is defined as symmetries of the transitive $G$-set $X$, and so $\pt_{W_GH}=\pt_{W_GH}$ is nothing but $(X=_{G\text{-set}}X)=\prod_{y:\BG}(X(y)=X(y))$.  On the other hand, $\BH_\div$ is equivalent to $\sum_{y:\BG}X(y)$ and 
$$\prod_{y:\BG}(X(y)=X(y))\simeq \prod_{\sum_{y:\BG}X(y)}X(y),$$ so $\pt_{W_GH}=\pt_{W_GH}$ is equivalent to the set $% (i_H^*X)^H\defequi
\prod_{x:\BH}X\, \Bi_Hx$ of fixed points of $X=G/H$ (regarded as an $H$-set through $i_H$).

Summing up
\begin{lemma}
  \label{lem:WGHisHfixofG/H}
  The map  $e:(X=X)\to \prod_{x:\BH}X\, \Bi_Hx$ with $e(f)(y,v)=f(y)$ defines an equivalence
$$e:(\pt_{W_GH}=\pt_{W_GH})\we (G/H)^H.$$ 
\end{lemma}

{\color{blue}THE REST OF THE CHAPTER CONTAINS KNOWN NONSENSE \tiny Don't actually seem to need at present; hence put on hold
\begin{lemma}
  \label{lem:iso2}
  Let $G$ be a group with subgroups $(H,i_H,!)$ and $(N,i_N,!)$ where $N$ is normal. Let $i_{HN}:Hom(H \vee N,G)$ be the homomorphism from the sum $H\vee N$ of \cref{def:sumofgroup} to $G$ induced by $i_H$ and $i_N$.  Then 
the pullback $\BN\times_{\BG}\BH$
is equivalent to $G/{HN}\times B(H\cap N)$.\footnote{display the equivalence}
\end{lemma}
\begin{proof}
  ((WRITE)) Can alternatively be phrased as the fiber of $B(H\ltimes N)\to \BG$ once semi-direct product has been discussed.
\end{proof}

The below should be deleted


Recall that we defined a normal subgroup as a kernel of a homomorphism (which we may assume is surjective by replacing the target with the image without changing the kernel); we can now give a second characterization:
\begin{lemma}
  \label{lem:normalisfixed}
  Let $G$ be a group.  A subgroup of $G$ is normal if and only if it is a fixed points under the conjugation action.
\end{lemma}
\begin{proof}
  Consider a surjective homomorphism $f:\Hom(G,G')$ and let $\BN\defequi\sum_{z:\BG}(\sh_{G'}=\Bf(z))$ (pointed at $(\sh_G,p_f)$) represent its kernel, with the first projection to $\BG$ representing the injection $i_N:\Hom(N,G)$ (with $\refl{\sh_G}$ the witness that $\sh_G$ is identical to the first projection of $(\sh_G,p_f)$).   Now, by the very representation of $N$, for every $g:\USym G$ we get an equivalence $C^g:\BN\we \BN$ by setting $C^g(z,p)\defequi(z,p\,f(g)^{-1})$ with basepoint identity (from $\pt_N\defequi(\sh_G,p_f)$ to $C^g\sh_G\defequi (\sh_G,p_ff(g)^{-1})$) given by $g^{-1}:\USym G$ and the fact 
$$\xymatrix{\pt_Q\ar@{=}[rr]^{p_f}_\to\ar@{=}[d]_{\refl{\pt_Q}}&&f(\sh_G)\\
\pt_Q\ar@{=}[r]^-{p_f}_-\to&f(\sh_G)&\,f(\sh_G).\ar@{=}[l]^\gets_{f(g)}\ar@{=}[u]^\uparrow_{f(g)}}
$$
Since $C^g$ followed by the first projection is exactly the first projection and also the base points match up (\ie $\refl{pt_G}\circ\mathrm{pr}_1(g^{-1},!)=_{\sh_G=\mathrm{pr}_1(\pt,p_f)}g^{-1}$) we get an identity $(N,\Bi_N,\refl{\sh_G},!)=_{\typesubgroup_G}(N,\Bi_N,g^{-1},!)$, showing that the normal subgroup is a fixed point. 

Conversely, let $(H,i,!)$ be any subgroup of $G$ and consider the pointed component $\BW$ of the type of $G$-sets containing the cokernel $G/H$.  If $X$ is a $G$-torsor, then the orbit $X/H$ is a $G$-set in $\BW$, \footnote{((explain how you transport the $G\times G$-action from $G$ or deloop))}
providing us with a pointed map $f:\BG\to \BW$.
By \cref{lem:aut-orbit} \footnote{((which has yet to be provided with a proof))} the identity type $G/H=G/H$ is 


% Let $(H,F,p,!):\typesubgroup_G$ be a fixed point, \ie for all $g:\USym G$ there is an identity $C^g:H=H$ so that 
% $$\xymatrix{}
% $$
\end{proof}}


% \begin{definition}
%   \label{def:normalizer}
% \footnote{TO BE MOVED TO or AFTER the chapter on symmetry (need Burnside -  a $G$-set splits into orbits - etc) has been covered.}
% An element of the $G$-orbit of a subgroup $(H,i_H,!)$ are called a \emph{conjugate} of $(H,i_H,!)$.   The stabilizer group of a subgroup $(H,i_H,!)$ is called the \emph{normalizer $N_G(H)$ of $H$ in $G$}.\index{normalizer}

% If $(K,i_K,!)$ is another subgroup containing a conjugate of $(H,i_H,!)$, we say that $(H,i_H,!)$ is \emph{subconjugate} to $(K,i_K,!)$.
% \end{definition}

% \begin{lemma}
%   Let $(H,i_H,!):\typesubgroup_G$ and let $N_G(H)$ be the normalizer subgroup of $H$ of \cref{ex:abstrandconj} (considered as a subgroup $(N_G(H),i_{N_G(H)},!)$ of $G$).  Then $H$ is a normal subgroup of $N_G(H)$.  ((come back and prove normal))
% \end{lemma}
% \begin{proof}
%   Remember that $N_G(H)$ is the stabilizer subgroup of $(H,i_H,!)$ under the conjugation action, so to prove that $H$ is a subgroup of $N_GH$ we need to show that if $h:\USym H$, then there is an identity between $(H,i_H,!)$ and $c^{i_H(h)}(H,i_H,!)$.   
% Let $g\defequi i_H(h)$.  If $s:\USym H$, then $c^gi_H(s)=g\,i_H(s)\,g^{-1}=i_H(h)\,i_H(s)\,i_H(h)^{-1}=i_H(h\,s\,h^{-1})=i_H(c^hs)$ (since $i_H$ is a homomorphism).  
% Using the identity $c^h:H=H$ we have obtained an identity $(H,i_H,!)=(H,c^{i_H(h)}i_H,!)$.
% \end{proof}

\section{Historical remarks}
\label{sec:grouphistory}

% Move in place

% \begin{remark}
%   Notice that the last statement  (``More precisely\dots'')  not only asserts that there \emph{exist} inverses, but that there actually is a (preferred and consistent) way to produce them.

% Classically this was in many instances unnecessay to say because there was a unique inverse, and the distinction is not mentioned in introductory texts.  However, then this very point had to be revisited later on.  In our proof relevant setting it is obvious that the ultimate statement will have to go beyond an assertion that inverses exist.
% \end{remark}

%%% Local Variables:
%%% mode: latex
%%% fill-column: 144
%%% TeX-master: "book"
%%% End:


%the below is the illustration used for the n-fold \covering in the deck trafo section.
% Move in place
% \begin{figure}
%   \centering
%   \begin{tikzpicture}
%     \node (A) at (2,2) {$\sqrt[n]X$};
%     \node (B) at (2,-2) {$\bn{n}$};
%     \draw[->] (A) -- node[auto] {$p$} (B);
%     \foreach \y in {-2,0,1,2}
%     { \begin{scope}[shift={(0,\y)}]
%         \foreach \x in {0,...,4}
%         { \node[fill,circle,inner sep=1pt] at (180+72*\x:1 and .3) {}; }
%         \foreach \x in {0,...,3}
%         { \draw[-stealth] (180+72*\x:1 and .3) arc(180+72*\x:252+72*\x:1 and .3); }
%       \end{scope} }
%     \begin{scope}[shift={(0,-2)}]
%       \draw[-stealth] (108:1 and .3) arc(108:180:1 and .3);
%     \end{scope}
%     \foreach \y in {1,2}
%     { \begin{scope}[shift={(0,\y)}]
%         \draw[-stealth] (108:1 and .3)
%         .. controls ++( 5:-.3) and ++(80:.2) .. (-.7,-.4)
%         .. controls ++(80:-.2) and ++(90:.2) .. (-1,-1);
%       \end{scope} }
%     \draw[-stealth] (108:1 and .3)
%     .. controls ++( 5:-.3) and ++(80:.2) .. (-.7,-.4);
%     \node (dz) at (-.7,-.7) {\footnotesize $\vdots$};
%     \begin{scope}[shift={(0,3)}]
%       \draw[-stealth] (-.7,-.4)
%       .. controls ++(80:-.2) and ++(90:.2) .. (-1,-1);
%     \node (da) at (-.7,0) {\footnotesize $\vdots$};
%     \end{scope}
%   \end{tikzpicture}
%   \caption{The $n$'th root of an endomorphism, with projection}
%   \label{fig:rootproj}
% \end{figure}
